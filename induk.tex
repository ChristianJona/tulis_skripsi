\documentclass[12pt, a4paper, onecolumn, twoside]{report}
% File laporan_setting.tex harus Anda sesuaikan dengan informasi Anda.
%
% Template skripsi / tesis ini berdasarkan aturan yang dikeluarkan UI di tahun 2010.
% Yang mengembangkan template ini dari awal hingga saat ini:
% 1. Fahrurrozi Rahman.
% 2. Adreas Febrian, Lia Sadita, Andre Tampubolon, Erik Dominikus.
% 3. Andrew, Daniel Salim (Mat UI 2010).
% 4. Mia Vania (Mat UI 2011).
% 5. Hengki Tasman
% 6. Barry (Mat UI 2010), Maulana (Mat UI 2011)
% 7. Antonio Kevin (Mat UI 2014)*

%*UPDATE: KEPUTUSAN REKTOR NOMOR 2143/SK/R/UI/2017

%NOTE! Remember to do use [\vspace*{-0.25cm}] for enumerate or other listing, so the spacing will be consistent. 
%TO DO NEXT. Change footer font into Arial.

\usepackage[paper=a4paper,headheight=30pt,left=3cm,top=3cm,right=3cm,bottom=3cm]{geometry}
\usepackage{graphicx}
\usepackage{enumerate}
\usepackage{amsmath,amssymb,amsthm}
\usepackage{mathrsfs}
\usepackage{indentfirst} % Kalimat pertama masuk beberapa cm
\usepackage{natbib} % Untuk merujuk pustaka
\usepackage{tocbibind}
%\usepackage{multirow} % Membuat banyak baris dalam 1 baris di tabel
%\usepackage{colortbl} % Membuat tabel berwarna-warni.
\usepackage{caption}
\usepackage{float}
\usepackage{times}
\usepackage{multirow}
\usepackage{subfig}
\usepackage{makecell}
\usepackage{fancyhdr} % Membuat header dan footer pada dokumen.
%\usepackage{pdfpages} % Package untuk memasukan berkas pdf sebagai bagian dari dokumen.

%\usepackage{colortbl} % Membuat tabel berwarna-warni.
\usepackage{algorithmicx}
\usepackage{algcompatible}

\usepackage{hyperref}
\hypersetup{a4paper,pdfstartview=FitH,pdfview=FitH}

\usepackage{array}
\usepackage{url}
\urlstyle{same}

\usepackage{setspace}
\linespread{1.3} % Jarak baris 1,5

\usepackage[ConnyRevised]{fncychap} % Membantu dalam mengatur jarak antara tepi kertas dengan posisi header.


\usepackage{threeparttable}

%-----------------------------------------------------------------------------%
% Konfigurasi
%-----------------------------------------------------------------------------%
\sloppy

\newif\ifpdf
\ifx\pdfoutput\undefined
   \pdffalse
\else
   \pdfoutput=1
   \pdftrue
\fi

% link page numbers in TOC
\makeatletter
\def\contentsline#1#2#3#4{%
  \ifx\\#4\\%
    \csname l@#1\endcsname{#2}{#3}%
  \else
    \csname l@#1\endcsname{%
      \hyper@linkstart{link}{#4}{#2}\hyper@linkend
    }{%
      % same link destination for the page:
      \hyper@linkstart{link}{#4}{#3}\hyper@linkend
      % link destination is the page itself:
      % \hyperpage{#3}%
    }%
  \fi
}

\renewcommand\section{\@startsection {section}{1}{\z@}%
                                  {-3.5ex \@plus -1ex \@minus -.2ex}%
                                  {2.3ex \@plus.2ex}%
                                  {\normalsize \bfseries}}

\renewcommand\subsection{\@startsection{subsection}{2}{\z@}%
                                     {-3.5ex\@plus -1ex \@minus -.2ex}%
                                     {1.5ex \@plus .2ex}%
                                     {\normalsize\bfseries}}
\makeatother

\setcounter{secnumdepth}{3}

\setcounter{tocdepth}{3}

\newcommand{\f}[1]{\textit{#1}}
\newcommand{\bi}[1]{\textbf{\textit{#1}}}
\newcommand{\bo}[1]{\textbf{#1}}

% Perintah untuk membuat perintah/variabel baru.
\newcommand{\var}[2]{\newcommand{#1}{#2}}

% Perintah untuk membuat perintah/variabel baru. Teks yang ditulis dalam
% perintah ini akan diformat ulang menggunakan huruf kapital.
\newcommand{\Var}[2]{\newcommand{#1}{\uppercase{#2}}}

% Tambahkan kata-kata yang dimasukan kedalam Table of Contents.
\newcommand{\addChapter}[1]{\phantomsection \addcontentsline{toc}{chapter}{#1}}

\renewcommand{\bibname}{\normalsize DAFTAR REFERENSI}
\renewcommand{\contentsname}{\normalsize DAFTAR ISI}
\renewcommand{\listfigurename}{\normalsize DAFTAR GAMBAR}
\renewcommand{\listtablename}{\normalsize DAFTAR TABEL}
\renewcommand{\figurename}{\normalsize Gambar}
\renewcommand{\tablename}{\normalsize Tabel}
\renewcommand{\chaptername}{\normalsize BAB}
\def\proofname{Bukti}

\theoremstyle{definition} % Isi teorema tidak ditulis miring
\newtheorem{theorem}{Teorema}[chapter]
\newtheorem{lemma}[theorem]{Lema} 
\newtheorem{proposition}[theorem]{Proposisi}
\newtheorem{corollary}[theorem]{Akibat}
\newtheorem{definition}{Definisi}[chapter]
\newtheorem{example}{Contoh}[chapter]

\newenvironment{remark}[1][Catatan]{\begin{trivlist}
\item[\hskip \labelsep {\normalsize \bfseries #1}]}{\end{trivlist}}

\usepackage{titlesec}
\titlespacing*{\chapter}{0pt}{-32pt}{20pt}
\titleformat{\chapter}[display]{\center \normalsize \bfseries}{\chaptertitlename \, \thechapter}{0pt}{\normalsize} % Memperbaiki judul tiap bab

% Memuat konfigurasi khusus untuk laporan yang sedang dibuat
%-----------------------------------------------------------------------------
% Informasi Mengenai Dokumen
%-----------------------------------------------------------------------------

% Judul laporan.
% Tulis judul Skripsi/Thesis/Disertasi Anda di sini
\var{\judul}{Analisis Kestabilan Global pada Model Infeksi HIV}

% Tulis kembali judul laporan, kali ini akan diubah menjadi huruf kapital secara otomatis
% Tulis judul Skripsi/Thesis/Disertasi Anda di sini
\Var{\Judul}{Analisis Kestabilan Global pada Model Infeksi HIV}

% Tulis kembali judul laporan namun dengan bahasa Ingris
\var{\judulInggris}{Global Stability Analysis of an HIV Infection Model}

% Tipe laporan, dapat berisi Skripsi, Tesis atau Disertasi
\var{\type}{Skripsi / Tesis / Disertasi} % Pilih satu, hapus yang tidak perlu
\Var{\Type}{Skripsi / Tesis / Disertasi} % Pilih satu, hapus yang tidak perlu, Akan jadi huruf besar semua secara otomatis

% Tulis nama penulis
\var{\penulis}{Nurdini K.}
\Var{\Penulis}{Nurdini K.} % Akan jadi huruf besar semua secara otomatis

% Tulis NPM penulis
\var{\npm}{1006673613}

% Tuliskan Fakultas dimana penulis berada
\Var{\Fakultas}{Matematika dan Ilmu Pengetahuan Alam}
\var{\fakultas}{Matematika dan Ilmu Pengetahuan Alam}

% Tuliskan Program Studi yang diambil penulis
\Var{\Program}{Sarjana Matematika / Statistika / Ilmu Aktuaria}
\var{\program}{Sarjana Matematika / Statistika / Ilmu Aktuaria}

% Tuliskan Program Studi yang diambil penulis, dalam bahasa Inggris
\Var{\Programinggris}{Undergraduate Study Program of Mathematics / Statistics / Actuarial Science}
\var{\programinggris}{Undergraduate Study Program of Mathematics / Statistics / Actuarial Science}

% Tuliskan bulan dan tahun publikasi laporan
\Var{\bulanTahun}{Juni 2025}
\var{\tahun}{2025}

% Tuliskan gelar yang akan diperoleh dengan menyerahkan laporan ini
\var{\gelar}{Sarjana / Magister / Doktor} % Pilih satu, hapus yang tidak perlu

% Tuliskan tanggal pengesahan laporan, waktu dimana laporan diserahkan ke
% penguji/sekretariat
\var{\tanggalPengesahan}{17 Juli 2025}

% Tuliskan tanggal keputusan sidang dikeluarkan dan penulis dinyatakan lulus/tidak lulus
\var{\tanggalLulus}{17 Juli 2025}

% Tuliskan nama pembimbing
\var{\pembimbinga}{Dr. Hengki Tasman}
\var{\pembimbingb}{Nama Pembimbing II} % Beri tanda % di awal baris ini jika Anda tidak punya Pembimbing II skripsi

%Tuliskan nama penguji
\var{\pengujia}{Nama Penguji I}
\var{\pengujib}{Nama Penguji II}
 % Anda HARUS edit file laporan_setting.tex

% Memuat Daftar pemenggalan suku kata dan istilah dalam LaTeX
%
% Hyphenation untuk Indonesia
%
% @author  Andreas Febrian
% @version 1.00
%
% Tambahkan cara pemenggalan kata-kata yang salah dipenggal secara otomatis
% oleh LaTeX. Jika kata tersebut dapat dipenggal dengan benar, maka tidak
% perlu ditambahkan dalam berkas ini. Tanda pemenggalan kata menggunakan
% tanda '-'; contoh:
% menarik
%   --> pemenggalan: me-na-rik
%

\hyphenation{
    % alphabhet A
    aki-bat
    ana-li-sis
    ana-li-tik
    ang-go-ta
    an-ti-bio-tik
    % alphabhet B
    ba-gai-ma-na
    be-be-ra-pa
    ber-ge-rak
    ber-hu-bu-ngan
    ber-pe-nga-ruh
    ber-te-ri-ma
    bi-la-ngan
    bim-bi-ngan
    % alphabhet C
    % alphabhet D
    de-fi-ni-si
    de-ngan
    di-ban-ding-kan
    dia-gram
    di-ba-ngun
    di-buk-ti-kan
    di-da-pat-kan
    di-de-fi-ni-si-kan
    di-fe-ren-si-al
    di-ha-rap-kan
    di-li-hat
    di-mo-del-kan
    di-mu-lai
    di-nya-ta-kan
    di-pe-nga-ru-hi
    di-sa-ji-kan
    di-re-pre-sen-ta-si-kan
    di-ting-kat-kan
    du-ku-ngan
    % alphabhet E
    eks-pli-sit
    eks-pre-si
    eva-lua-si
    % alphabhet F
    frak-sio-nal
    % alphabhet G
    geo-me-tri
    % alphabhet H
    ha-rap
    hi-tung
    % alphabhet I
    im-pli-sit
    in-te-gral
    ite-ra-si
    ite-ra-tif
    % alphabhet J
    % alphabhet K
    ka-it
    ke-mung-kin-an
    ke-se-im-ba-ngan
    ke-se-tim-ba-ngan
    ki-rim
    kom-par-te-men
    kon-fi-gu-ra-si
    kon-struk-si
    % alphabhet L
    li-mit
    li-te-ra-tur
    % alphabhet M
    ma-te-ma-ti-ka
    me-mo-del-kan
    me-mo-ri
    men-de-ngar-kan
    me-ne-ri-ma
    me-nga-it-kan
    me-nga-ju-kan
    me-nga-la-mi
    me-nga-na-li-sis
    me-nga-pa
    me-nga-rah-kan
    meng-ar-ti-kan
    me-nge-nai
    me-ngi-rim-kan
    me-nya-ta-kan
    mi-sal
    mi-sal-kan
    mo-del
    mu-lai
    mung-kin
    % alphabhet N
    na-pas
    non-eks-klu-sif
    % alphabhet O
    ope-ra-tor
    % alphabhet P
    pa-ling
    pa-ra-me-ter
    pa-sang
    pa-sang-an
    pe-mo-del-an
    pe-nga-ruh
    pe-nga-wa-san
    pe-ngem-ba-ngan
    pe-ngo-bat-an
    pen-ting
    pe-ngu-ra-ngan
    pe-nye-le-sai-an
    per-ge-rak-an
    per-hi-tung-an
    pe-ri-la-ku
    per-ten-ta-ngan
    per-u-mum-an
    peru-ba-han
    po-pu-la-si
    pro-duk-ti-vi-tas
	% alphabhet Q
    % alphabhet R
    re-pre-sen-ta-si
    % alphabhet S
    sa-ngat
    sa-tu
    sa-tu-nya
    se-dang
    se-dang-kan
    se-ki-tar
    se-ring
    si-mu-la-si
    su-a-tu
    sub-sti-tu-si
    % alphabhet T
    ten-tang
    te-ra-pi
    te-ri-ma
    tri-li-un
    tu-run-an
    % alphabhet U
    % alphabhet V
    % alphabhet W
    % alphabhet X
    % alphabhet Y
    % alphabhet Z
    % special
}  % Tambahkan daftar pemenggalan kata dalam file hype.indonesia.tex

% Memuat Daftar istilah yang mungkin perlu ditandai
\input{istilah}

\begin{document}

% Sampul Depan Skripsi / Tesis
\include{sampul} % Dilarang mengubah isi file sampul.tex 

\pagenumbering{roman} % Gunakan penomoran halaman romawi

%pengaturan headerfooterhalaman
\renewcommand{\headrulewidth}{0.0pt}
	\fancyhf{}
	\fancyhead[L]{}
	\fancyhead[C]{}
	\fancyhead[R]{}
	\renewcommand{\headrulewidth}{0.0pt}
	\fancyfoot[C]{\thepage}
	\renewcommand{\footrulewidth}{0pt}
\pagestyle{fancy}

% Memuat halaman judul dalam
\addChapter{HALAMAN JUDUL}
\include{judul_dalam} % Dilarang mengubah isi file judul_dalam.tex

\setcounter{page}{2} % nomor halaman selanjutnya adalah 2

% Memuat halaman pengesahan
%\addChapter{LEMBAR PERSETUJUAN}
%%
% Halaman Pengesahan
%
% @author  Andreas Febrian
% @version 1.01
%

\chapter*{\normalsize HALAMAN PERSETUJUAN}

\vspace*{0.2cm}
\noindent

\noindent
\begin{tabular}{l l p{11cm}}
	Judul&: & \judul \\
	Nama&: & \penulis \\
	NPM&: & \npm \\
\end{tabular} \\

\vspace*{1.2cm}

\noindent Laporan \type~ini telah diperiksa dan disetujui.\\[0.3cm]
\begin{center}
\tanggalPengesahan \\[2cm]


\underline{\pembimbinga}\\[0.1cm]
Pembimbing \type
\end{center}

\newpage  % Dilarang mengubah isi file pengesahan.tex 

% load halaman orisinalitas
\addChapter{HALAMAN PERNYATAAN ORISINALITAS}
%
% Halaman Orisinalitas
%
% @author  Andreas Febrian
% @version 1.01
%

\chapter*{\normalsize HALAMAN PERNYATAAN ORISINALITAS}
\vspace*{2cm}
\begin{doublespace}
\begin{center}
	\type~ini adalah hasil karya saya sendiri, \\
	dan semua sumber baik yang dikutip maupun dirujuk \\
	telah saya nyatakan dengan benar. \\
	\vspace*{2.6cm}
	
	\begin{tabular}{l c l}
	Nama & : & \penulis \\
	NPM & : & \npm \\
	Tanda Tangan & : & \\
	& & \\
	& & \\
	Tanggal & : & \tanggalPengesahan \\	
	\end{tabular}
\end{center}
\end{doublespace}

\newpage  % Dilarang mengubah isi file orisinal.tex

%pengaturan rata kiri
%\raggedright
\setlength\parindent{0.7cm}

\addChapter{HALAMAN PENGESAHAN}
%
% Halaman Pengesahan Sidang
%
% @author  Andreas Febrian, Andre Tampubolon
% @version 1.02
%

\chapter*{\normalsize HALAMAN PENGESAHAN}
\begin{singlespace}
\vspace*{0.4cm}
\noindent

\noindent
\begin{tabular}{ll p{9cm}}
	\type~ini diajukan oleh&: & \\ % ---- Jangan diedit ----
	Nama&: & \penulis \\ % ---- Jangan diedit ----
	NPM&: & \npm \\ % ---- Jangan diedit ----
	Program Studi&: & \program \\ % ---- Jangan diedit ----
	Judul \type&: & \judul \\ % ---- Jangan diedit ----
\end{tabular} \\

\vspace*{1.0cm}

\noindent Telah berhasil dipertahankan di hadapan Dewan Penguji
dan diterima sebagai bagian persyaratan yang diperlukan untuk
memperoleh gelar \gelar \, pada Program Studi \program, Fakultas
\fakultas, Universitas Indonesia.\\[0.2cm] % ---- Jangan diedit ----

\begin{center}
	DEWAN PENGUJI
\end{center}

\vspace*{0.3cm}

\begin{tabular}{l l l l }
& & & \\
Pembimbing I &: & \pembimbinga & (\hspace*{3.0cm}) \\ % ---- Jangan diubah ----

\vspace{0.7cm}

& & & \\


Pembimbing II&: & \pembimbingb & (\hspace*{3.0cm}) \\ % ---- Silakan hapus bagian ini jika tidak ada pembimbing II

\vspace{0.7cm}

& & & \\


Penguji I&: & \pengujia & (\hspace*{3.0cm}) \\ % ---- Jangan diubah ------

\vspace{0.7cm}

& & & \\


Penguji II&: & \pengujib & (\hspace*{3.0cm}) \\ % ---- Jangan diubah ------
& & & \\
\end{tabular}\\

\vspace*{2.0cm}

\begin{tabular}{ll l}
	Ditetapkan di&: & Depok\\ % ---- Jangan diedit ----
	Tanggal&: & \tanggalLulus \\ % ---- Jangan diedit ----
\end{tabular}
\end{singlespace}

\newpage  % File pengesahan_sidang.tex mungkin perlu diubah sesuai dengan banyaknya pembimbing dan penguji kolokium atau sidang Anda

\addChapter{KATA PENGANTAR}
%-----------------------------------------------------------------------------%
\chapter*{\normalsize KATA PENGANTAR}
%-----------------------------------------------------------------------------%

Puji syukur kepada Tuhan Yang Maha Esa atas segala berkat dan rahmat-Nya, sehingga penulis dapat menyelesaikan skripsi ini dengan sebaik-baiknya. Selama masa penulisan skripsi ini, penulis telah mendapat banyak dukungan, doa, bantuan, inspirasi, dan motivasi dari berbagai pihak. Untuk itu, penulis ingin berterima kasih kepada:
\begin{enumerate}
	\item Semua staf pengajar atas ilmu pengetahuan yang telah diberikan kepada penulis selama masa kuliah.
	\item Seluruh staf karyawan yang selalu melakukan tugas mereka dengan baik, sehingga memberikan kenyamanan pelayanan bagi siapa saja.
	\item Seluruh teman-teman angkatan 2009 atas semangat, canda tawa dan banyak hal penting selama masa kuliah.
	\item Kakak-kakak angkatan 2008, 2007, dan 2006 serta adik-adik angkatan 2010, dan 2011 yang terus memberi semangat.
\end{enumerate}

\vspace*{0.1cm}

\begin{flushright}
\penulis \\[0.1cm]
\vspace*{0.2cm}
\tahun
\end{flushright}  % Anda HARUS isi file pengantar.tex -------------

\addChapter{HALAMAN PERSETUJUAN PUBLIKASI ILMIAH}
\include{persetujuan_publikasi} % Dilarang ubah file persetujuan_publikasi.tex 

%pengaturan header footer halaman
\renewcommand{\headrulewidth}{0.0pt}
	\fancyhf{}
	\fancyhead[L]{}
	\fancyhead[C]{}
	\fancyhead[R]{}
	\renewcommand{\headrulewidth}{0.0pt}
	\fancyfoot[C]{\thepage}
	\fancyfoot[R]{{\fontfamily{phv}\selectfont\footnotesize \bo{Universitas Indonesia}}}
	\renewcommand{\footrulewidth}{0.0pt}
\pagestyle{fancyplain}

\addChapter{ABSTRAK}
%
% Halaman Abstrak
%
% @author  Andreas Febrian
% @version 1.00
%

\chapter*{\normalsize ABSTRAK}
\begin{singlespace}
\vspace*{0.2cm}

\noindent\begin{tabular}{l l p{10cm}}
Nama&: & \penulis \\ % Jangan diubah
Program Studi&: & \program \\ % Jangan diubah
Judul&: & \judul \\ % Jangan diubah
\end{tabular}

\vspace*{0.5cm}

\noindent Secara matematis, melipat dapat dilakukan dengan merotasi bidang kertas yang ingin dilipat terhadap sumbu garis lipatan. Pemetaan dari kertas ke hasil lipatan origami dilakukan dengan merotasi bidang kertas yang sesuai. Apabila sebuah origami yang telah selesai dibuka kembali, terdapat garis-garis bekas lipatan pada kertas. Garis-garis lipatan ini disebut sebagai pola lipatan. Jika sebuah pola lipatan dapat dilipat, perkalian matriks-matriks rotasi sesuai merupakan matriks identitas. Hal ini berlaku pada origami simpul tunggal, dan berlaku secara lokal pada origami simpul jamak, namun dapat diperluas sehingga berlaku secara global pada origami simpul jamak.\\

\vspace*{0.2cm}

\noindent Kata kunci:\\
	Origami, rotasi, transformasi, lipatan tak datar. \\ % ------------ Jangan lupa diedit
\end{singlespace}
\newpage  % Abstrak dalam bahasa Indonesia. Ubah file abstrak.tex.

\addChapter{ABSTRACT}
%
% Halaman Abstract
%
% @author  Andreas Febrian
% @version 1.00
%

\chapter*{\normalsize ABSTRACT}
\begin{singlespace}
\vspace*{0.2cm}

\noindent \begin{tabular}{l l p{11.0cm}}
	Name&: & \penulis \\ % ------------ Jangan diedit
	Program&: & \programinggris \\ % ------------ Jangan diedit
	Title&: & \judulInggris \\ % ------------ Jangan diedit
\end{tabular} \\

\vspace*{0.5cm}

\noindent Mathematically, to fold a paper is to rotate the paper along a crease line as the axis. The mapping from the paper to the finished origami fold is done by rotating parts of the paper to the appropriate locations. Unfolding finished origami reveals a pattern of crease lines, known as crease pattern. If the crease pattern is foldable, then the product 
of the associated rotational matrices is the identity matrix. This condition holds in a single vertex crease pattern and holds locally in a multiple vertex crease pattern and can be adapted to a global condition in a multiple vertex crease pattern\\

\vspace*{0.2cm}

\noindent Keywords:\\
	Origami, rotation, transformation, non-flat folding.\\ % ------------ Jangan lupa diedit
\end{singlespace}

\newpage  % Abstrak dalam bahasa Inggris. Ubah file abstract.tex.

% Daftar isi, gambar, dan tabel
\tableofcontents
\clearpage

\listoftables % jika ada tabel
\clearpage

\listoffigures % jika ada gambar
\clearpage{\thispagestyle{empty}\cleardoublepage}

% Gunakan penomoran Arab (1, 2, 3, ...) setelah bagian ini.
\pagenumbering{arabic}
%pengaturan headerfooter halaman
\renewcommand{\headrulewidth}{0.0pt}
   \fancyhf{}
	\fancyhead[RO,LE]{\thepage}
	\renewcommand{\headrulewidth}{0.0pt}
	\fancyfoot[R]{\footnotesize \bo{Universitas Indonesia}}
	\renewcommand{\footrulewidth}{0.0pt}
\pagestyle{fancy}

\fancypagestyle{plain}{ %
  \fancyhf{} 
  	\fancyhead[L]{}
  	\fancyhead[C]{}
  	\fancyhead[R]{}
  	\renewcommand{\headrulewidth}{0pt} 
  	\fancyfoot[C]{\thepage}
  	\fancyfoot[R]{\footnotesize \bo{Universitas Indonesia}}
  	\renewcommand{\footrulewidth}{0pt}
}

\clearpage{\thispagestyle{empty}\cleardoublepage}
\chapter{Pendahuluan}

\section{Latar Belakang}

Peningkatan harapan hidup global telah menjadi salah satu pencapaian penting dalam bidang kesehatan dan sosial pada abad ke-21. Fenomena ini ditandai dengan bertambahnya jumlah individu yang mencapai usia sangat lanjut, termasuk kelompok supercentenarian yaitu individu berusia 110 tahun atau lebih \citep{Young2021}. Data menunjukkan bahwa jumlah supercentenarian yang tervalidasi meningkat secara signifikan, dengan kasus pertama tercatat pada tahun 1960 dan terus berkembang hingga saat ini. Kondisi ini membawa implikasi penting bagi sistem jaminan sosial, industri asuransi jiwa, dan dana pensiun yang harus mengantisipasi risiko umur panjang (\textit{longevity risk}) dengan lebih cermat.

Risiko umur panjang mengacu pada ketidakpastian terkait dengan proyeksi harapan hidup yang lebih tinggi dari perkiraan semula. Ketika individu hidup lebih lama dari yang diperkirakan, institusi keuangan yang memberikan jaminan pembayaran jangka panjang seperti anuitas dan dana pensiun menghadapi beban finansial yang lebih besar. Oleh karena itu, pemodelan mortalitas pada usia lanjut yang akurat menjadi kebutuhan mendasar untuk mengukur dan mengelola risiko ini secara efektif.

Model mortalitas klasik seperti Gompertz-Makeham, Heligman-Pollard, dan model logistik telah lama digunakan dalam praktik aktuaria untuk memodelkan pola kematian manusia. Namun, model-model tersebut menunjukkan keterbatasan dalam menangkap fenomena deselerasi mortalitas pada usia lanjut ekstrem, yaitu kondisi di mana laju peningkatan tingkat kematian melambat atau bahkan mendekati konstan pada usia sangat tinggi \citep{Thatcher1999}. Selain itu, model-model klasik umumnya menetapkan batas usia maksimum tabel mortalitas (\textit{highest attained age}, $\omega$) secara subjektif tanpa landasan statistik yang kuat, sehingga berpotensi menghasilkan estimasi yang kurang reliabel pada ekor distribusi usia kematian.

Untuk mengatasi keterbatasan tersebut, pendekatan berbasis \textit{extreme value theory} (EVT) mulai diterapkan dalam pemodelan mortalitas usia lanjut. EVT merupakan cabang statistik yang secara khusus dirancang untuk menganalisis perilaku ekor distribusi dan kejadian ekstrem. Dalam konteks mortalitas, EVT memberikan kerangka kerja yang lebih sesuai untuk memodelkan individu yang mencapai usia sangat lanjut dengan menggunakan \textit{generalized Pareto distribution} (GPD). Salah satu model yang mengadopsi EVT adalah \textit{threshold life table} (TLT), yang dikembangkan untuk menggabungkan model Gompertz pada usia non-ekstrem dengan GPD pada usia lanjut ekstrem, dengan pemisahan kedua komponen dilakukan pada suatu usia ambang (\textit{threshold age}, $u$).

Meskipun model TLT memberikan fleksibilitas dalam memodelkan mortalitas usia lanjut, model ini memiliki kelemahan potensial berupa ketidakmulusingan (\textit{discontinuity}) pada fungsi \textit{hazard} di titik usia ambang. Ketidakmulusingan ini dapat menimbulkan interpretasi yang kurang realistis secara biologis dan mengurangi kualitas proyeksi mortalitas. Untuk mengatasi masalah tersebut, dikembangkan model \textit{smoothed threshold life table} (STLT) yang menambahkan kendala kehalusan (\textit{smoothing constraint}) sehingga memastikan transisi yang mulus antara komponen Gompertz dan GPD pada usia ambang.

Lebih lanjut, untuk mengakomodasi perubahan pola mortalitas antar kohor dan memungkinkan peramalan mortalitas ke masa depan, model STLT diperluas menjadi model dinamis yang disebut \textit{dynamic smooth threshold life table} (DSTLT). Model ini memperkenalkan komponen waktu dengan memodelkan parameter tertentu sebagai fungsi dari indeks kohor, sehingga mampu menangkap tren mortalitas jangka panjang dan melakukan proyeksi untuk kohor-kohor mendatang.

Penelitian ini bertujuan untuk mengkaji secara mendalam konstruksi, estimasi parameter, dan evaluasi kinerja model STLT dan DSTLT dalam konteks pemodelan mortalitas usia lanjut. Kajian ini diharapkan dapat memberikan kontribusi pada pemahaman yang lebih baik mengenai penerapan EVT dalam aktuaria, serta menyediakan alternatif model yang lebih robust untuk mengukur dan mengelola risiko umur panjang dalam industri asuransi dan dana pensiun.

\section{Rumusan Masalah}

Berdasarkan latar belakang yang telah diuraikan, penelitian ini difokuskan pada kajian mendalam terhadap model STLT dan DSTLT dalam konteks pemodelan mortalitas usia lanjut. Rumusan masalah penelitian ini dijabarkan dalam pertanyaan-pertanyaan berikut:

\begin{enumerate}
    \item Bagaimana konstruksi matematis model STLT dan DSTLT, khususnya terkait penambahan kendala kehalusan pada fungsi \textit{hazard} dan pemodelan parameter dinamis?
    
    \item Bagaimana prosedur estimasi parameter untuk model STLT dan DSTLT menggunakan metode \textit{maximum likelihood estimation} (MLE) dengan mempertimbangkan karakteristik data mortalitas yang tersensor interval dan kanan, serta penentuan usia ambang optimal?
    
    \item Bagaimana kinerja model STLT dibandingkan dengan model mortalitas statis lain seperti Gompertz-Makeham, Heligman-Pollard, dan Coale-Kisker dalam hal kesesuaian terhadap data historis?
    
    \item Bagaimana kinerja model DSTLT dalam hal kesesuaian \textit{in-sample} dan kemampuan peramalan \textit{out-of-sample}, khususnya dibandingkan dengan model mortalitas dinamis yang relevan seperti model Cairns-Blake-Dowd (CBD)?
\end{enumerate}

\section{Tujuan Penelitian}

Berdasarkan rumusan masalah yang telah dipaparkan, penelitian ini memiliki tujuan-tujuan sebagai berikut:

\begin{enumerate}
    \item Mengonstruksi model STLT dan DSTLT secara matematis, termasuk derivasi kendala kehalusan pada fungsi \textit{hazard} untuk model STLT dan formulasi parameter dinamis untuk model DSTLT.
    
    \item Melakukan estimasi parameter model STLT dan DSTLT menggunakan metode \textit{maximum likelihood estimation} (MLE) dengan mempertimbangkan data mortalitas tersensor interval dan kanan, serta menentukan usia ambang optimal.
    
    \item Mengevaluasi kinerja model STLT dengan membandingkan kesesuaian model (\textit{goodness-of-fit}) terhadap model mortalitas statis lain seperti Gompertz-Makeham, Heligman-Pollard, dan Coale-Kisker.
    
    \item Mengevaluasi kinerja model DSTLT dalam hal kesesuaian \textit{in-sample} dan kemampuan peramalan \textit{out-of-sample}, dengan membandingkannya terhadap model mortalitas dinamis Cairns-Blake-Dowd (CBD).
\end{enumerate}

\section{Batasan Penelitian}

 Penelitian ini mengasumsikan tidak terjadi migrasi pada populasi yang diteliti. Asumsi ini diperlukan untuk menyederhanakan analisis kohor dan memastikan perubahan jumlah individu dalam kohor hanya disebabkan oleh kematian. Untuk memenuhi asumsi ini, dilakukan penyesuaian data untuk menciptakan kohor tertutup yang menghilangkan pengaruh migrasi.
\clearpage{\thispagestyle{empty}\cleardoublepage}
\chapter{Landasan Teori}

Bab ini berisikan teori yang menjadi dasar penelitian mengenai pemodelan mortalitas usia lanjut dengan pendekatan \textit{Dynamic Smooth Threshold Life Table} (DSTLT). Pembahasan dimulai dengan konsep dasar tabel mortalitas dan notasi aktuaria yang digunakan secara luas dalam analisis mortalitas. Selanjutnya, diuraikan berbagai model mortalitas statis seperti Gompertz-Makeham, Heligman-Pollard, dan Coale-Kisker yang menjadi perbandingan dalam penelitian ini. Pemahaman terhadap \textit{Extreme Value Theory} (EVT) dan pendekatannya dalam memodelkan kejadian ekstrem menjadi landasan penting untuk memahami konstruksi model \textit{Threshold Life Table} (TLT). Model TLT kemudian dikembangkan menjadi \textit{Smoothed Threshold Life Table} (STLT) untuk mengatasi ketidakmulsan fungsi \textit{hazard} di usia ambang. Terakhir, diperkenalkan versi dinamis dari model STLT, yaitu DSTLT, yang memungkinkan pemodelan tren mortalitas antarkohor dan peramalan mortalitas jangka panjang. Seluruh teori yang dipaparkan dalam bab ini mengacu pada kerangka kerja yang dikembangkan oleh \citet{huang2020modelling} dan literatur terkait di bidang aktuaria dan statistika ekstrem.

\section{Konsep Dasar Tabel Mortalitas}

Tabel mortalitas merupakan instrumen fundamental dalam ilmu aktuaria yang menyajikan representasi matematis dari pola kematian suatu populasi. Tabel ini menyediakan informasi kuantitatif mengenai probabilitas bertahan hidup dan kematian pada berbagai usia, yang sangat esensial untuk penilaian risiko longevitas, penetapan premi asuransi jiwa, dan perhitungan kewajiban dana pensiun \citep{dickson2020actuarial}.

\subsection{Fungsi-Fungsi Dasar Mortalitas}

Misalkan $X$ adalah variabel acak kontinu yang menyatakan usia saat kematian seseorang yang baru lahir. Fungsi distribusi kumulatif (\textit{cumulative distribution function}, CDF) dari $X$ didefinisikan sebagai
\begin{equation}
    F(x) = P(X \leq x), \quad x \geq 0,
\end{equation}
yang menyatakan probabilitas seorang individu meninggal sebelum atau pada usia $x$. Fungsi probabilitas bertahan hidup (\textit{survival function}) dinyatakan sebagai
\begin{equation}
    S(x) = P(X > x) = 1 - F(x), \quad x \geq 0,
\end{equation}
yang menunjukkan probabilitas seseorang bertahan hidup melampaui usia $x$. Fungsi kepekatan probabilitas (\textit{probability density function}, PDF) dari $X$ adalah
\begin{equation}
    f(x) = \frac{dF(x)}{dx} = -\frac{dS(x)}{dx}, \quad x \geq 0,
\end{equation}
yang menggambarkan laju kematian instantan pada usia $x$.

Fungsi \textit{hazard} atau fungsi tingkat kematian (\textit{force of mortality}), yang dinotasikan dengan $h(x)$ atau $\mu(x)$, didefinisikan sebagai
\begin{equation}
    h(x) = \lim_{\Delta x \to 0^+} \frac{P(x < X \leq x + \Delta x \mid X > x)}{\Delta x} = \frac{f(x)}{S(x)} = -\frac{d \ln S(x)}{dx}.
\end{equation}
Fungsi ini mengukur tingkat kematian sesaat pada usia $x$, dengan asumsi individu tersebut telah bertahan hingga usia tersebut. Hubungan antara fungsi \textit{hazard} dan fungsi \textit{survival} dapat dinyatakan melalui relasi integral
\begin{equation}
    S(x) = \exp\left(-\int_0^x h(t) \, dt\right).
\end{equation}

\subsection{Notasi Aktuaria}

Dalam praktik aktuaria, digunakan notasi diskret yang mengadaptasi fungsi-fungsi kontinu di atas ke dalam interval usia yang terpisah, biasanya tahunan. Beberapa notasi aktuaria standar yang digunakan dalam penelitian ini adalah sebagai berikut \citep{dickson2020actuarial}:

\begin{itemize}
    \item $l_x$: Jumlah individu yang bertahan hidup hingga usia tepat $x$ dalam suatu kohor hipotetis.
    \item $d_x = l_x - l_{x+1}$: Jumlah kematian yang terjadi antara usia $x$ dan $x+1$.
    \item $q_x = \frac{d_x}{l_x}$: Probabilitas seorang individu berusia $x$ meninggal sebelum mencapai usia $x+1$.
    \item $p_x = 1 - q_x = \frac{l_{x+1}}{l_x}$: Probabilitas seorang individu berusia $x$ bertahan hidup hingga usia $x+1$.
    \item $e_x = \sum_{k=1}^{\omega - x} k \cdot {}_{k|} q_x$: Harapan hidup lengkap pada usia $x$, yaitu ekspektasi jumlah tahun yang akan dijalani oleh seseorang yang kini berusia $x$.
    \item $m_x = \frac{d_x}{L_x}$: Tingkat kematian sentral, dengan $L_x$ adalah \textit{person-years lived} antara usia $x$ dan $x+1$. Untuk interval satu tahun, sering diasumsikan $m_x \approx \frac{d_x}{(l_x + l_{x+1})/2}$.
\end{itemize}

Notasi-notasi ini memfasilitasi penghitungan dan analisis mortalitas dalam konteks aktuaria, khususnya untuk keperluan penilaian risiko dan penetapan tarif produk asuransi serta pensiun.

\subsection{Highest Attained Age dan Interval Censoring}

Dalam data mortalitas empiris, khususnya untuk usia lanjut, informasi mengenai usia kematian yang tepat seringkali tidak tersedia. Data yang tersedia biasanya dalam bentuk usia terakhir ulang tahun (\textit{last birthday age}) atau usia terakhir yang dicapai (\textit{highest attained age}). Hal ini menyebabkan data mortalitas bersifat tersensor interval (\textit{interval-censored}).

Misalkan seorang individu tercatat meninggal pada usia $k$ (usia terakhir ulang tahun). Ini berarti usia kematian sebenarnya $X$ berada dalam interval $[k, k+1)$. Dalam konteks estimasi parameter model mortalitas, fungsi likelihood harus memperhitungkan sifat tersensor ini. Untuk individu yang meninggal pada interval $[k, k+1)$, kontribusi terhadap likelihood adalah
\begin{equation}
    P(k \leq X < k+1) = S(k) - S(k+1) = S(k) \left[1 - \frac{S(k+1)}{S(k)}\right] = S(k) \cdot q_k,
\end{equation}
di mana $q_k$ adalah probabilitas kematian dalam satu tahun pada usia $k$.

Selain tersensor interval, data mortalitas usia lanjut juga sering mengalami tersensor kanan (\textit{right-censored}) untuk individu yang masih hidup pada akhir periode observasi. Jika seorang individu bertahan hidup hingga usia $k$ pada akhir pengamatan, kontribusinya terhadap likelihood adalah $S(k)$, yang mencerminkan probabilitas bertahan hingga setidaknya usia tersebut \citep{huang2020modelling}.

Pemahaman terhadap mekanisme penyensoran ini sangat penting dalam konstruksi fungsi likelihood untuk estimasi parameter model mortalitas berbasis \textit{maximum likelihood estimation} (MLE), yang akan dibahas lebih lanjut dalam bab selanjutnya.

\section{Model Mortalitas Statis}

Model mortalitas statis merupakan model parametrik yang menggambarkan pola kematian sebagai fungsi dari usia tanpa mempertimbangkan dimensi waktu atau perubahan antarkohor. Model-model ini telah digunakan secara luas dalam praktik aktuaria untuk memodelkan tabel mortalitas pada berbagai rentang usia. Bagian ini membahas beberapa model mortalitas statis yang relevan sebagai pembanding terhadap model STLT dalam penelitian ini, meliputi model Gompertz-Makeham, model logistik (Perks, Beard, Kannisto), model Heligman-Pollard, dan metode Coale-Kisker \citep{huang2020modelling, dickson2020actuarial}.

\subsection{Model Gompertz-Makeham}

Salah satu model mortalitas parametrik paling awal dan berpengaruh dikembangkan oleh Benjamin Gompertz pada tahun 1825. Model Gompertz didasarkan pada observasi bahwa tingkat kematian dewasa cenderung meningkat secara eksponensial seiring bertambahnya usia. Fungsi \textit{hazard} atau \textit{force of mortality} dalam model Gompertz dinyatakan sebagai
\begin{equation}
    h(x) = B \exp(Cx), \quad x \geq 0,
\end{equation}
dengan $B > 0$ adalah parameter skala yang merepresentasikan tingkat mortalitas dasar, dan $C > 0$ adalah parameter bentuk yang mengatur laju peningkatan mortalitas seiring bertambahnya usia. Gompertz mengemukakan penjelasan fisiologis bahwa kapasitas seseorang untuk menghindari kematian secara bertahap menurun seiring bertambahnya usia, yang dapat dikaitkan dengan deteriorasi tubuh secara bertahap akibat akumulasi kerusakan molekuler dan seluler \citep{gompertz1825nature, thatcher1998force}.

William Makeham pada tahun 1860 menyempurnakan model Gompertz dengan menambahkan komponen konstan untuk mengakomodasi penyebab kematian yang dianggap independen terhadap usia, seperti kecelakaan atau penyakit tertentu. Fungsi \textit{hazard} dalam model Gompertz-Makeham adalah
\begin{equation}
    h(x) = A + B \exp(Cx), \quad x \geq 0,
\end{equation}
dengan $A > 0$ merepresentasikan mortalitas yang tidak terkait dengan penuaan atau pematangan biologis. Model ini dapat diinterpretasikan sebagai model kejutan (\textit{shock model}), di mana waktu hidup individu merupakan minimum dari waktu hingga kematian akibat penuaan (mengikuti distribusi Gompertz) dan waktu hingga kecelakaan fatal (mengikuti distribusi eksponensial), dengan asumsi kedua variabel acak tersebut independen \citep{makeham1860law, bower1997graduation}.

Meskipun model Gompertz-Makeham memberikan kesesuaian yang baik pada rentang usia dewasa, kelemahan utamanya adalah ketidakmampuannya menangkap fenomena deselerasi mortalitas pada usia lanjut (\textit{late-life mortality deceleration}). Fenomena ini mengacu pada perlambatan laju peningkatan mortalitas yang teramati pada usia sangat lanjut, di mana laju kematian tidak lagi meningkat secara eksponensial atau bahkan mencapai plateau \citep{barbi2018plateau, gavrilov2005mortality}. Asumsi peningkatan eksponensial tanpa batas membuat model ini kurang akurat untuk memodelkan mortalitas pada usia lanjut ekstrem, khususnya di atas 90 atau 100 tahun.

\subsection{Model Heligman-Pollard}

Model Heligman-Pollard, yang dikembangkan oleh Heligman dan Pollard pada tahun 1980, menawarkan pendekatan yang lebih komprehensif dengan menggabungkan tiga komponen utama untuk memodelkan probabilitas kematian $q_x$ pada seluruh rentang usia. Model ini berusaha menangkap pola mortalitas manusia dari kelahiran hingga usia lanjut dengan mengintegrasikan tiga sumber risiko kematian: kematian pada usia dini, kematian akibat kecelakaan, dan kematian akibat penuaan biologis. Fungsi probabilitas kematian dalam model Heligman-Pollard dinyatakan sebagai
\begin{equation}
    \frac{q_x}{1 - q_x} = A^{(x+B)^C} + D \exp\left(-E[\ln x - \ln F]^2\right) + GH^x, \quad x \geq 1,
\end{equation}
dengan $A, B, C, D, E, F, G, H$ adalah parameter model \citep{heligman1980age}. Komponen pertama $A^{(x+B)^C}$ memodelkan mortalitas pada anak usia dini yang umumnya tinggi dan menurun dengan cepat. Komponen kedua $D \exp\left(-E[\ln x - \ln F]^2\right)$ menangkap puncak mortalitas akibat kecelakaan pada usia dewasa muda. Komponen ketiga $GH^x$ merepresentasikan mortalitas akibat penuaan biologis dan dapat dipandang sebagai faktor yang berkaitan dengan hukum Gompertz.

Meskipun model Heligman-Pollard dapat memberikan kesesuaian yang baik dengan data mortalitas pada berbagai rentang usia, model ini memiliki beberapa keterbatasan penting. Pertama, komponen penuaan dalam model ini pada dasarnya mengadopsi hukum Gompertz yang, sebagaimana telah dijelaskan, tidak menggambarkan dengan baik fenomena deselerasi mortalitas pada usia lanjut ekstrem. Penelitian yang dilakukan oleh Olshansky dan Carnes pada tahun 1997 menunjukkan bahwa pola mortalitas usia lanjut tidak mengikuti hukum Gompertz \citep{olshansky1997ever}. Kedua, formulasi model menghasilkan nilai $q_x$ yang selalu kurang dari 1 untuk setiap $x > 0$. Hal ini menjadi kendala praktis dalam konstruksi tabel mortalitas lengkap, karena tabel mortalitas perlu memiliki usia maksimum di mana $q_x = 1$ atau $l_x = 0$. Dalam praktiknya, aktuaris harus menentukan usia terminal (\textit{terminal age}) secara subjektif untuk melengkapi tabel mortalitas berbasis model Heligman-Pollard \citep{huang2020modelling}.

\subsection{Metode Coale-Kisker}

Metode yang dikembangkan oleh Coale dan Kisker pada tahun 1990 merupakan metode ekstrapolasi yang sering digunakan untuk memodelkan mortalitas pada usia lanjut, khususnya di negara-negara berkembang di mana data mortalitas usia tinggi tidak tersedia atau kurang reliabel. Berbeda dengan model-model sebelumnya yang memodelkan fungsi \textit{hazard} atau probabilitas kematian secara langsung, metode Coale-Kisker bekerja dengan mengekstrapolasi tingkat kematian sentral (\textit{central death rate}), $m_x$ \citep{coale1990defects}.

Metode ini menggunakan persamaan rekursif untuk $k(x) = \ln(m_x / m_{x+1})$, yaitu logaritma natural dari rasio tingkat kematian sentral pada dua usia berurutan. Relasi rekursif tersebut adalah
\begin{equation}
    k(x) = k(x-1) - R, \quad x \geq x_0,
\end{equation}
dengan $R$ adalah konstanta penurunan dan $x_0$ adalah usia awal ekstrapolasi. Konstanta $R$ dihitung menggunakan persamaan
\begin{equation}
    R = \frac{(x_1 - x_0) k(x_0) + \ln m_{x_0} - \ln m_{x_1}}{1 + 2 + \cdots + (x_1 - x_0)},
\end{equation}
di mana $x_1$ adalah usia akhir ekstrapolasi yang dipilih. Setelah $R$ diperoleh, nilai $k(x)$ dapat dihitung secara iteratif untuk $x > x_0$, dan kemudian $m_x$ dapat dipulihkan menggunakan hubungan $m_x = m_{x-1} \exp(-k(x-1))$.

Kelemahan utama metode Coale-Kisker adalah ketergantungannya pada penetapan nilai $x_0$, $x_1$, dan $m_{x_1}$ secara subjektif. Sebagai contoh, Coale dan Kisker menggunakan $x_0 = 84$, $x_1 = 110$, dan $m_{110} = 1.0$ untuk laki-laki, serta $m_{110} = 0.8$ untuk perempuan dalam analisis mereka. Pilihan subjektif terhadap parameter-parameter ini secara langsung mempengaruhi hasil ekstrapolasi mortalitas pada usia lanjut ekstrem. Akurasi pada usia ekstrem kemungkinan rendah dengan metode ini karena kurva yang dihasilkan sangat bergantung pada titik awal dan akhir ekstrapolasi, yang merupakan kelemahan mendasar dari teknik ini \citep{huang2020modelling}.

Sebagai catatan, dalam penelitian ini, metode Coale-Kisker tidak diestimasi menggunakan \textit{maximum likelihood estimation} seperti model-model lainnya, melainkan menggunakan prosedur ekstrapolasi deterministik sebagaimana dijelaskan di atas. Oleh karena itu, perbandingan kinerja model menggunakan kriteria \textit{sum of squared errors} (SSE) antara $q_x$ observasi dan $q_x$ prediksi, bukan berdasarkan kriteria likelihood atau kriteria informasi seperti AIC atau BIC.

\section{Extreme Value Theory}

\textit{Extreme Value Theory} (EVT) merupakan cabang statistika yang secara khusus menganalisis perilaku nilai-nilai ekstrem dari suatu distribusi probabilitas. Dalam konteks mortalitas, EVT menyediakan kerangka teoretis yang kuat untuk memodelkan probabilitas kejadian langka seperti kematian pada usia lanjut ekstrem dan melakukan ekstrapolasi di luar rentang data observasi secara lebih terjustifikasi secara statistik \citep{coles2001introduction, gbari2017extreme}. Keterbatasan data empiris pada usia sangat lanjut, yang ditandai dengan jumlah observasi yang sedikit dan volatilitas tinggi, menjadikan EVT sebagai alat yang tepat karena dirancang untuk menangani kejadian ekstrem dengan fondasi teoretis yang solid.

\subsection{Konsep Dasar Extreme Value Theory}

Misalkan $T_1, T_2, \ldots, T_n$ adalah sekuens variabel acak independen yang merepresentasikan usia kematian individual dengan fungsi distribusi kumulatif yang sama $_xq_0 = P(T_i \leq x)$ untuk $x \geq 0$ dan $i = 1, \ldots, n$, dengan $_0q_0 = 0$. Variabel acak $T_i$ dapat merepresentasikan total masa hidup sejak lahir hingga kematian, atau sisa masa hidup setelah usia awal tertentu $\alpha$.

Definisikan sekuens maksimum $M_n = \max\{T_1, T_2, \ldots, T_n\}$. Dalam konteks mortalitas, $M_n$ merepresentasikan usia kematian tertinggi yang diamati dalam kelompok homogen beranggotakan $n$ individu yang tunduk pada tabel mortalitas yang sama. EVT mempelajari perilaku asimptotik dari $M_n$ ketika $n \to \infty$ dan memberikan hasil yang analog dengan teorema limit sentral untuk maksimum (bukan untuk jumlah), dengan syarat kondisi teknis tertentu pada fungsi distribusi terpenuhi. Jelas bahwa tanpa restriksi lebih lanjut, $M_n$ akan mendekati batas atas dari support distribusi,
\begin{equation}
    \omega = \sup\{x \geq 0 : {}_xq_0 < 1\},
\end{equation}
yang mungkin berhingga atau tak berhingga. Hal ini dapat dilihat dari
\begin{equation}
    P(M_n \leq x) = ({}_xq_0)^n \to \begin{cases}
        0 & \text{jika } x < \omega, \\
        1 & \text{jika } x \geq \omega,
    \end{cases}
\end{equation}
ketika $n \to \infty$.

Namun, setelah $M_n$ dipusatkan dan dinormalisasi secara tepat, distribusinya dapat konvergen ke suatu distribusi limit tertentu. Secara lebih presisi, jika terdapat sekuens bilangan riil $a_n > 0$ dan $b_n \in \mathbb{R}$ sedemikian sehingga sekuens ternormalisasi $(M_n - b_n)/a_n$ konvergen dalam distribusi ke $H$, yaitu
\begin{equation}
    \lim_{n \to \infty} P\left(\frac{M_n - b_n}{a_n} \leq x\right) = \lim_{n \to \infty} \left({}_{{a_n x + b_n}}q_0\right)^n = H(x),
\end{equation}
untuk semua titik kontinuitas dari $H$, maka $H$ adalah distribusi nilai ekstrem tergeneralisasi (\textit{generalized extreme value distribution}, GEV), yaitu $H = H_\xi$ yang diberikan oleh
\begin{equation}
    H_\xi(x) = \begin{cases}
        \exp\left(-(1 + \xi x)_+^{-1/\xi}\right) & \text{jika } \xi \neq 0, \\
        \exp(-\exp(-x)) & \text{jika } \xi = 0,
    \end{cases}
\end{equation}
dengan $y_+ = \max\{y, 0\}$ adalah bagian positif dari $y$ \citep{coles2001introduction, resnick2007heavy}. Domain definisi dari $H_\xi$ adalah $(-1/\xi, +\infty)$ jika $\xi > 0$, $(-\infty, -1/\xi)$ jika $\xi < 0$, dan seluruh garis riil $\mathbb{R}$ jika $\xi = 0$. Parameter $\xi$ yang mengontrol ekor kanan distribusi disebut \textit{tail index} atau \textit{extreme value index}. Tiga distribusi nilai ekstrem klasik adalah kasus khusus dari keluarga GEV: jika $\xi > 0$, diperoleh distribusi Fr\'echet; jika $\xi < 0$, diperoleh distribusi Weibull; dan $\xi = 0$ memberikan distribusi Gumbel.

Dalam aplikasi mortalitas, kasus $\xi > 0$ mengimplikasikan masa hidup dengan ekor berat (\textit{heavy tails}), yang berarti fungsi \textit{hazard} menurun seiring bertambahnya usia. Hal ini bertentangan dengan bukti empiris untuk masa hidup manusia. Dengan demikian, kasus $\xi = 0$ dan $\xi < 0$ yang relevan untuk aplikasi asuransi jiwa \citep{gbari2017extreme}. Perlu dicatat bahwa jika kondisi konvergensi terpenuhi dengan $\xi < 0$, maka $\omega < \infty$, sehingga nilai negatif dari $\xi$ mendukung eksistensi usia maksimum berhingga.

Suatu kondisi cukup untuk konvergensi di atas adalah
\begin{equation}
    \lim_{x \to \omega} \frac{d}{dx}\left(\frac{1}{\mu_x}\right) = \xi,
\end{equation}
dengan $\mu_x$ adalah \textit{force of mortality} pada usia $x$. Secara intuitif, $1/\mu_x$ dapat dipandang sebagai kekuatan resistensi terhadap mortalitas atau kekuatan vitalitas pada usia $x$. Resistensi terhadap mortalitas harus stabil ketika $\xi = 0$ atau menjadi linier secara asimptotik. Nilai $\xi$ negatif mengindikasikan bahwa resistensi pada akhirnya menurun pada usia lanjut. Untuk $\xi < 0$, diperoleh $\omega < \infty$, dan kondisi di atas mengimplikasikan
\begin{equation}
    \lim_{x \to \omega} \left[(\omega - x) \mu_x\right] = -\frac{1}{\xi}.
\end{equation}

\subsection{Pendekatan Peaks Over Threshold (POT)}

Selain pendekatan \textit{block maxima} yang menganalisis nilai maksimum dalam blok-blok data, EVT juga menawarkan pendekatan alternatif yang disebut \textit{Peaks Over Threshold} (POT). Pendekatan ini menganalisis semua nilai observasi yang melebihi suatu ambang batas $u$ yang cukup tinggi, sehingga memanfaatkan informasi dari seluruh data ekstrem yang tersedia, bukan hanya nilai maksimum per blok \citep{coles2001introduction}.

Dalam konteks mortalitas, perhatikan sisa masa hidup $T - x$ pada usia $x$, dengan asumsi $T > x$, yang memiliki fungsi distribusi
\begin{equation}
    {}_sq_x = P(T - x \leq s \mid T > x), \quad s \geq 0.
\end{equation}
Dapat terjadi bahwa untuk usia lanjut $x$ yang besar, distribusi probabilitas kondisional ini menjadi stabil setelah normalisasi, yaitu terdapat fungsi positif $a(\cdot)$ sedemikian sehingga
\begin{equation}
    \lim_{x \to \omega} P\left(\frac{T - x}{a(x)} > s \,\bigg|\, T > x\right) = 1 - G(s), \quad s > 0,
\end{equation}
dengan $G$ adalah fungsi distribusi yang tidak degenerasi. Dapat ditunjukkan bahwa hanya kelas terbatas dari fungsi distribusi yang memenuhi kondisi di atas, yaitu
\begin{equation}
    G(s) = G_\xi(s) = \ln H_\xi(s) = \begin{cases}
        1 - (1 + \xi s)_+^{-1/\xi} & \text{jika } \xi \neq 0, \\
        1 - \exp(-s) & \text{jika } \xi = 0.
    \end{cases}
\end{equation}
Support distribusi ini adalah setengah garis riil positif jika $\xi \geq 0$ dan $[0, -1/\xi]$ jika $\xi < 0$. Keluarga skala terkait yang dikenal sebagai \textit{generalized Pareto distribution} (GPD) didefinisikan sebagai
\begin{equation}
    G_{\xi, \beta}(s) = G_\xi\left(\frac{s}{\beta}\right), \quad \beta > 0,
\end{equation}
dengan $\beta$ adalah parameter skala. Kasus-kasus khusus dari GPD meliputi distribusi Pareto ketika $\xi > 0$, distribusi Pareto tipe II ketika $\xi < 0$, dan distribusi eksponensial negatif ketika $\xi = 0$. Dengan demikian, ketika $\xi = 0$, sisa masa hidup pada usia tinggi menjadi terdistribusi eksponensial negatif secara asimptotik, sehingga \textit{force of mortality} menjadi konstan, sejalan dengan studi empiris yang dilakukan oleh Gampe pada tahun 2010 \citep{gampe2010human}.

Analisis nilai ekstrem untuk maksimum dengan demikian berkaitan erat dengan studi sisa masa hidup. Dapat ditunjukkan bahwa konvergensi ke GEV berlaku jika dan hanya jika konvergensi ke GPD berlaku. Dengan kata lain, $G_\xi$ menggambarkan sisa masa hidup di atas usia yang cukup tua jika dan hanya jika $H_\xi$ mengatur perilaku maksimum sampel, yaitu jika fungsi distribusi $F$ termasuk dalam domain atraksi dari distribusi GEV.

\subsection{Generalized Pareto Distribution dan Teorema Pickands-Balkema-de Haan}

Untuk suatu fungsi $\beta(\cdot)$ yang sesuai, aproksimasi
\begin{equation}
    {}_sq_x \approx G_{\xi; \beta(x)}(s), \quad s \geq 0,
\end{equation}
berlaku untuk $x$ yang cukup besar. Aproksimasi ini dijustifikasi oleh Teorema Pickands-Balkema-de Haan yang menyatakan bahwa
\begin{equation}
    \lim_{x \to \omega} \sup_{s \geq 0} \left|{}_sq_x - G_{\xi, \beta(x)}(s)\right| = 0,
\end{equation}
berlaku dengan syarat $F$ memenuhi kondisi teknis yang cukup umum \citep{balkema1974residual, pickands1975statistical}. Berdasarkan aproksimasi ini, sisa masa hidup pada usia $x$ dapat diperlakukan sebagai sampel acak dari distribusi GPD, dengan syarat $x$ cukup besar.

Jika $\omega < \infty$ (yaitu $\xi < 0$), maka transformasi yang sesuai dari \textit{extreme value index} $\xi$ memiliki interpretasi intuitif. Harapan hidup tersisa pada usia $x$, yang dinotasikan sebagai $e_x$, didefinisikan sebagai
\begin{equation}
    e_x = E[T - x \mid T > x].
\end{equation}
Aarssen dan de Haan pada tahun 1994 menetapkan bahwa untuk $\xi < 0$, sehingga batas atas $\omega$ ada pada rentang masa hidup, konvergensi ke GEV ekuivalen dengan
\begin{equation}
    \lim_{x \to \omega} E\left[\frac{T - x}{\omega - x} \,\bigg|\, T > x\right] = \lim_{x \to \omega} \frac{e_x}{\omega - x} = -\frac{\xi}{1 - \xi} = \alpha.
\end{equation}
Para peneliti ini menyebut $\alpha = \alpha(\xi)$ sebagai \textit{perseverance parameter} dan memberikan penjelasan sebagai berikut \citep{aarssen1994domains}. Perhatikan seorang individu yang masih hidup pada usia lanjut $x$. Rasio $(T - x)/(\omega - x)$ merepresentasikan persentase sisa masa hidup aktual $T - x$ terhadap sisa masa hidup maksimum $\omega - x$. Persentase ini menjadi stabil, secara rata-rata, ketika $x \to \omega$ dan konvergen ke $\alpha$, yang dengan demikian muncul sebagai persentase ekspektasi dari sisa masa hidup maksimum yang mungkin, yang secara efektif digunakan oleh individu tersebut.

Teorema Pickands-Balkema-de Haan memberikan justifikasi teoretis untuk pendekatan POT dalam pemodelan mortalitas usia lanjut. Dalam praktik, untuk menerapkan hasil ini, perlu ditentukan usia ambang $u$ (atau $x^*$) sedemikian sehingga aproksimasi GPD cukup akurat untuk $x \geq u$. Pemilihan usia ambang yang tepat merupakan aspek krusial dalam implementasi model berbasis EVT, dan akan dibahas lebih lanjut dalam konteks konstruksi model \textit{Threshold Life Table} pada sub-bab berikutnya.

\section{Model Mortalitas Dinamis}

\subsection{Model Watts-Dupuis-Jones (WDJ)}

Model Watts-Dupuis-Jones (WDJ) yang dikembangkan oleh \citet{watts2006extreme} merupakan salah satu aplikasi pertama Extreme Value Theory untuk memodelkan tren temporal usia kematian tertinggi. Model ini menggunakan pendekatan block maxima dengan Generalized Extreme Value (GEV) distribution.

\subsubsection{Spesifikasi Model}

Model WDJ menggunakan distribusi GEV untuk memodelkan usia kematian maksimum yang diamati setiap tahun. Fungsi distribusi GEV didefinisikan sebagai:
\begin{equation}
F(z) = \begin{cases}
\exp\left(-(1 + \xi(z - \mu)/\sigma)^{-1/\xi}\right), & \xi \neq 0, \\
\exp(-\exp(-(z - \mu)/\sigma)), & \xi = 0,
\end{cases}
\label{eq:gev_distribution}
\end{equation}
dimana:
\begin{itemize}
    \item $\mu$ adalah parameter lokasi (location parameter)
    \item $\sigma > 0$ adalah parameter skala (scale parameter)
    \item $\xi$ adalah parameter bentuk (shape parameter atau tail index)
\end{itemize}

\subsubsection{Pemodelan Temporal}

Untuk menangkap tren perubahan usia maksimum dari waktu ke waktu, model WDJ memodelkan parameter lokasi $\mu$ dan skala $\sigma$ sebagai fungsi linear dari tahun kalender:
\begin{align}
\mu(t) &= \mu_0 + \mu_1 t^*, \label{eq:wdj_mu} \\
\sigma(t) &= \exp(\sigma_0 + \sigma_1 t^*), \label{eq:wdj_sigma}
\end{align}
dimana $t^* = (t - t_0)/\Delta t$ adalah tahun yang distandarisasi, dengan $t_0$ sebagai tahun referensi dan $\Delta t$ sebagai rentang waktu observasi.

Parameter bentuk $\xi$ diasumsikan konstan antar waktu, mencerminkan asumsi bahwa karakteristik ekor distribusi usia maksimum relatif stabil secara struktural.

\subsubsection{Interpretasi Parameter}

Parameter temporal memiliki interpretasi sebagai berikut:
\begin{itemize}
    \item \textbf{$\mu_1$}: Laju perubahan usia maksimum per unit waktu. Nilai $\mu_1 > 0$ mengindikasikan peningkatan usia maksimum dari waktu ke waktu.
    
    \item \textbf{$\sigma_1$}: Laju perubahan variabilitas usia maksimum. Transformasi eksponensial untuk $\sigma$ memastikan parameter skala selalu positif.
\end{itemize}

\subsection{Model Cairns-Blake-Dowd (CBD)}

Model mortalitas dinamis menangkap perubahan pola mortalitas dari waktu ke waktu, yang penting untuk melakukan proyeksi mortalitas masa depan. Berbeda dengan model statis yang mengasumsikan parameter tetap, model dinamis memodelkan parameter sebagai fungsi waktu atau kohor. Salah satu contoh model mortalitas dinamis adalah Model Cairns-Blake-Dowd (CBD).

Model Cairns-Blake-Dowd (CBD) yang diperkenalkan oleh \citet{cairns2006two} dirancang khusus untuk memodelkan mortalitas pada usia lanjut. Model ini menggunakan struktur yang lebih sederhana dibandingkan Lee-Carter namun tetap efektif untuk aplikasi pada rentang usia terbatas.

Model CBD memodelkan logit dari probabilitas kematian satu tahun $q_{x,t}$ sebagai fungsi linear dari usia:
\begin{equation}
\text{logit}(q_{x,t}) = \ln\left(\frac{q_{x,t}}{1-q_{x,t}}\right) = \kappa_t^{(1)} + \kappa_t^{(2)} (x - \bar{x}),
\label{eq:cbd_basic}
\end{equation}
dimana:
\begin{itemize}
    \item $x$ adalah usia (umumnya untuk rentang usia lanjut, misalnya 60--89 tahun)
    \item $t$ adalah tahun kalender atau indeks waktu
    \item $\bar{x}$ adalah usia rata-rata dalam rentang yang dianalisis (untuk centering)
    \item $\kappa_t^{(1)}$ adalah parameter temporal yang mengatur tingkat mortalitas keseluruhan pada tahun $t$
    \item $\kappa_t^{(2)}$ adalah parameter temporal yang mengatur slope atau gradien mortalitas terhadap usia pada tahun $t$
\end{itemize}

Dari persamaan \eqref{eq:cbd_basic}, probabilitas kematian dapat diperoleh melalui transformasi invers:
\begin{equation}
q_{x,t} = \frac{\exp(\kappa_t^{(1)} + \kappa_t^{(2)} (x - \bar{x}))}{1 + \exp(\kappa_t^{(1)} + \kappa_t^{(2)} (x - \bar{x}))}.
\label{eq:cbd_qx}
\end{equation}

Parameter temporal $\kappa_t^{(1)}$ dan $\kappa_t^{(2)}$ dimodelkan sebagai proses stokastik. Spesifikasi standar menggunakan \textit{random walk with drift}:
\begin{align}
\kappa_t^{(1)} &= \kappa_{t-1}^{(1)} + \mu^{(1)} + \epsilon_t^{(1)}, \label{eq:cbd_rw1} \\
\kappa_t^{(2)} &= \kappa_{t-1}^{(2)} + \mu^{(2)} + \epsilon_t^{(2)}, \label{eq:cbd_rw2}
\end{align}
dimana $\mu^{(1)}$ dan $\mu^{(2)}$ adalah drift parameters, dan $\epsilon_t^{(1)}, \epsilon_t^{(2)}$ adalah error terms yang diasumsikan white noise dengan mean nol.



Dalam penelitian ini, model CBD digunakan sebagai \textit{benchmark} untuk mengevaluasi kemampuan forecasting model Dynamic Smooth Threshold Life Table (DSTLT). Perbandingan dilakukan pada:
\begin{enumerate}
    \item \textbf{In-sample fit}: Kesesuaian model terhadap data historis yang digunakan untuk estimasi
    \item \textbf{Out-of-sample forecast accuracy}: Akurasi proyeksi mortalitas untuk kohor yang tidak termasuk dalam training set
\end{enumerate}

Model CBD dipilih sebagai benchmark karena merupakan model standar dalam literatur aktuaria untuk proyeksi mortalitas usia lanjut dan telah banyak digunakan dalam praktik industri asuransi dan dana pensiun \citep{villegas2018comparative}.

\section{Maximum Likelihood Estimation (MLE)}

Maximum Likelihood Estimation (MLE) merupakan metode estimasi parameter yang paling umum digunakan dalam statistika dan pemodelan mortalitas. Metode ini memilih nilai parameter yang memaksimalkan probabilitas (atau likelihood) dari data yang diamati.

\subsection{Konsep Dasar}

Misalkan terdapat data observasi $\mathbf{x} = (x_1, x_2, \ldots, x_n)$ yang diasumsikan berasal dari distribusi dengan fungsi kepadatan (atau fungsi massa) probabilitas $f(x; \boldsymbol{\theta})$, dimana $\boldsymbol{\theta}$ adalah vektor parameter yang tidak diketahui. Fungsi likelihood didefinisikan sebagai fungsi dari parameter $\boldsymbol{\theta}$ yang menyatakan probabilitas bersama dari data observasi:
\begin{equation}
L(\boldsymbol{\theta}; \mathbf{x}) = \prod_{i=1}^{n} f(x_i; \boldsymbol{\theta}).
\label{eq:likelihood_definition}
\end{equation}

Estimator Maximum Likelihood $\hat{\boldsymbol{\theta}}$ adalah nilai parameter yang memaksimalkan fungsi likelihood:
\begin{equation}
\hat{\boldsymbol{\theta}} = \arg\max_{\boldsymbol{\theta}} L(\boldsymbol{\theta}; \mathbf{x}).
\label{eq:mle_definition}
\end{equation}

Dalam pengerjaannya, lebih mudah untuk mendapatkan solusi dalam bentuk logaritma natural dari likelihood, yang biasa disebut log-likelihood:
\begin{equation}
\ell(\boldsymbol{\theta}; \mathbf{x}) = \ln L(\boldsymbol{\theta}; \mathbf{x}) = \sum_{i=1}^{n} \ln f(x_i; \boldsymbol{\theta}).
\label{eq:loglikelihood_definition}
\end{equation}

Untuk mendapatkan MLE, biasanya dilakukan dengan mencari solusi dari persamaan:
\begin{equation}
\frac{\partial \ell(\boldsymbol{\theta})}{\partial \boldsymbol{\theta}} = \mathbf{0}.
\label{eq:score_equation}
\end{equation}

\subsection{MLE untuk Data Tersensor}

Dalam analisis mortalitas, data sering bersifat tersensor, yang memerlukan penyesuaian dalam konstruksi fungsi likelihood.

Untuk data tersensor interval, dimana usia kematian hanya diketahui berada dalam interval $[x, x+1)$, kontribusi likelihood untuk setiap observasi adalah probabilitas berada dalam interval tersebut:
\begin{equation}
L_i = P(x < X \leq x+1) = F(x+1; \boldsymbol{\theta}) - F(x; \boldsymbol{\theta}).
\label{eq:likelihood_interval}
\end{equation}

Jika terdapat $d_x$ individu yang meninggal dalam interval $[x, x+1)$ dari $l_x$ individu berisiko, dan dengan mengabaikan konstanta kombinatorial, kontribusi log-likelihood adalah:
\begin{equation}
\ell_x = d_x \ln[F(x+1; \boldsymbol{\theta}) - F(x; \boldsymbol{\theta})].
\label{eq:loglik_interval}
\end{equation}

Dalam bentuk fungsi survival:
\begin{equation}
\ell_x = d_x \ln[S(x; \boldsymbol{\theta}) - S(x+1; \boldsymbol{\theta})].
\label{eq:loglik_interval_survival}
\end{equation}

Untuk individu yang tersensor kanan pada usia $\tau$ (masih hidup pada akhir pengamatan), kontribusi likelihood adalah probabilitas bertahan hidup melampaui $\tau$:
\begin{equation}
L_i = P(X > \tau) = S(\tau; \boldsymbol{\theta}).
\label{eq:likelihood_right_censored}
\end{equation}

Jika terdapat $l_\tau$ individu yang tersensor kanan, kontribusi log-likelihood adalah:
\begin{equation}
\ell_{\tau} = l_\tau \ln S(\tau; \boldsymbol{\theta}).
\label{eq:loglik_right_censored}
\end{equation}

Untuk data mortalitas dengan rentang usia dari $x_{\min}$ hingga $\tau$, fungsi log-likelihood total menggabungkan kontribusi dari semua interval usia dan data tersensor kanan:
\begin{equation}
\ell(\boldsymbol{\theta}) = \sum_{x=x_{\min}}^{\tau-1} d_x \ln[S(x; \boldsymbol{\theta}) - S(x+1; \boldsymbol{\theta})] + l_\tau \ln S(\tau; \boldsymbol{\theta}) - l_{x_{\min}} \ln S(x_{\min}; \boldsymbol{\theta}).
\label{eq:loglik_total_censored}
\end{equation}

Normalisasi dengan $S(x_{\min})$ memastikan bahwa likelihood dikondisikan pada bertahan hidup hingga usia awal pengamatan, yang konsisten dengan definisi probabilitas kondisional dalam analisis kohor.

Struktur likelihood ini merupakan dasar untuk estimasi parameter dalam model TLT, STLT, dan DSTLT yang akan dibahas pada Bab 3.



\section{Kriteria Evaluasi Model}

Evaluasi kinerja model mortalitas memerlukan kriteria yang tepat untuk mengukur kesesuaian model terhadap data historis dan akurasi proyeksi untuk data masa depan.

\subsection{Sum of Squared Errors (SSE)}

Sum of Squared Errors mengukur total deviasi kuadrat antara nilai prediksi model dengan nilai observasi:
\begin{equation}
\text{SSE} = \sum_{i=1}^{n} (y_i - \hat{y}_i)^2,
\label{eq:sse}
\end{equation}
dimana $y_i$ adalah nilai observasi (misalnya, $q_x$ empiris atau tingkat mortalitas observasi) dan $\hat{y}_i$ adalah nilai prediksi dari model.

Untuk evaluasi model mortalitas, SSE dapat dihitung berdasarkan:
\begin{itemize}
    \item Probabilitas kematian: $\text{SSE}_q = \sum_x (q_x^{\text{obs}} - q_x^{\text{pred}})^2$
    \item Tingkat mortalitas: $\text{SSE}_\mu = \sum_x (\mu_x^{\text{obs}} - \mu_x^{\text{pred}})^2$
\end{itemize}

Nilai SSE yang lebih kecil mengindikasikan kesesuaian yang lebih baik. Metrik ini terutama berguna untuk model seperti Coale-Kisker yang tidak diestimasi menggunakan MLE, sehingga tidak memiliki nilai log-likelihood yang dapat dibandingkan.

\subsection{In-Sample Fit vs Out-of-Sample Forecast}

Untuk mengevaluasi kemampuan prediksi model, diperlukan ukuran yang membandingkan proyeksi model dengan data aktual yang tidak digunakan dalam estimasi. Evaluasi model dapat dibagi menjadi dua kategori:

\begin{itemize}
    \item \textbf{In-sample fit}: Kesesuaian model terhadap data yang digunakan untuk estimasi parameter (training set). Ukuran ini menilai seberapa baik model menangkap pola dalam data historis.
    
    \item \textbf{Out-of-sample forecast}: Akurasi proyeksi model pada data yang tidak digunakan dalam estimasi (test set). Ukuran ini menilai kemampuan generalisasi dan prediksi model.
\end{itemize}

Untuk model dinamis seperti DSTLT, evaluasi out-of-sample sangat penting karena tujuan utama model adalah untuk melakukan proyeksi mortalitas masa depan.


\chapter{\normalsize MODEL DYNAMIC SMOOTH THRESHOLD LIFE TABLE (DSTLT)}

Pemodelan mortalitas pada usia lanjut ekstrem menghadapi tantangan yang tidak dapat diatasi sepenuhnya oleh model-model mortalitas tradisional. Keterbatasan data pada usia sangat lanjut, volatilitas tinggi dari estimasi empiris, dan kebutuhan untuk menentukan batas akhir tabel mortalitas secara objektif mendorong pengembangan metode yang lebih canggih. Bab ini membahas mengenai perkembangan bertahap dari pemodelan mortalitas usia lanjut, dimulai dari model \textit{Threshold Life Table} (TLT) sebagai model dasar, kemudian pengembangan \textit{Smoothed Threshold Life Table} (STLT) sebagai perbaikan, hingga \textit{Dynamic Smooth Threshold Life Table} (DSTLT) sebagai pengembangan untuk keperluan peramalan.

Model TLT yang diperkenalkan oleh Li et al. (2008) mengatasi masalah terkait penentuan usia penutupan tabel mortalitas yang seringkali bersifat sembarang. Model TLT memiliki struktur \textit{piecewise} yang menggabungkan distribusi Gompertz untuk usia non-ekstrem dengan \textit{Generalized Pareto Distribution} (GPD) untuk usia lanjut ekstrem. Namun, model TLT memiliki keterbatasan terkait kontinuitas fungsi \textit{hazard} pada titik transisi antara kedua bagian model. Ketidakmulusan fungsi \textit{hazard} ini menimbulkan implikasi yang tidak realistis secara biologis (nanti dimasukin sumbernya), dimana laju mortalitas sesaat dapat mengalami loncatan mendadak pada usia ambang tertentu. Keterbatasan inilah yang memotivasi pengembangan model STLT oleh Huang et al. (2020), yang memperkenalkan \textit{smoothing constraint} untuk memastikan kontinuitas fungsi \textit{hazard}.

Walaupun model STLT berhasil mengatasi masalah kontinuitas, terdapat keterbatasan dalam peramalan mortalitas. Pengamatan empiris terhadap parameter model STLT antar kohor menunjukkan adanya pola. Hal ini mendorong pengembangan model DSTLT, yang mengintegrasikan komponen dinamis untuk menangkap perubahan mortalitas antar cohort.

Bab ini akan menganalisis secara mendalam setiap tahap pengembangan model, dimulai dari formula dasar TLT, penurunan rumus \textit{smoothing constraint} pada STLT, hingga konstruksi komponen dinamis pada DSTLT.

\section{Model Threshold Life Table (TLT)}

Model \textit{Threshold Life Table} (TLT) yang diperkenalkan oleh \citet{li2008threshold} menggunakan pendekatan \textit{piecewise} yang membagi distribusi usia kematian menjadi dua bagian berdasarkan usia ambang (\textit{threshold age}) $u$. Model ini dikembangkan untuk mengatasi keterbatasan model mortalitas tradisional dalam menangani karakteristik yang berbeda antara usia lanjut non-ekstrem dan usia lanjut ekstrem.

\subsection{Formulasi Model TLT}

Misalkan $X$ adalah variabel acak yang menyatakan usia kematian seseorang. Model TLT mendefinisikan fungsi distribusi secara \textit{piecewise} berdasarkan usia ambang $u$, menggabungkan distribusi Gompertz untuk $x \leq u$ dan Generalized Pareto Distribution (GPD) untuk $x > u$.

\subsubsection{Komponen Gompertz ($x \leq u$)}

Untuk usia di bawah atau sama dengan usia ambang $u$, digunakan distribusi Gompertz dengan fungsi distribusi kumulatif:
\begin{equation}
F(x) = 1 - \exp\left(-\frac{B}{\ln C}(C^x - 1)\right), \quad x \leq u,
\label{eq:tlt_gompertz_cdf}
\end{equation}
dengan parameter $B > 0$ sebagai parameter skala dan $C > 1$ sebagai parameter bentuk yang menentukan laju peningkatan mortalitas seiring bertambahnya usia.

Fungsi survival untuk bagian Gompertz adalah:
\begin{equation}
S(x) = \exp\left(-\frac{B}{\ln C}(C^x - 1)\right), \quad x \leq u.
\label{eq:tlt_gompertz_survival}
\end{equation}

Fungsi kepadatan probabilitas (PDF) diperoleh melalui diferensiasi $f(x) = -S'(x)$:
\begin{equation}
f(x) = BC^x \exp\left(-\frac{B}{\ln C}(C^x - 1)\right), \quad x \leq u.
\label{eq:tlt_gompertz_pdf}
\end{equation}

Fungsi hazard untuk bagian Gompertz adalah:
\begin{equation}
h(x) = \frac{f(x)}{S(x)} = BC^x, \quad x \leq u,
\label{eq:tlt_gompertz_hazard}
\end{equation}
yang menunjukkan peningkatan eksponensial seiring bertambahnya usia.

\subsubsection{Komponen Generalized Pareto Distribution ($x > u$)}

Untuk usia di atas usia ambang $u$, digunakan Generalized Pareto Distribution (GPD) yang merupakan bagian dari Extreme Value Theory. GPD didefinisikan secara kondisional untuk \textit{exceedances} di atas ambang $u$.

Fungsi survival kondisional untuk bagian ini adalah:
\begin{equation}
G(x-u; \theta, \gamma) = \left(1 + \frac{\gamma(x-u)}{\theta}\right)^{-1/\gamma}, \quad x > u,
\label{eq:gpd_survival_conditional}
\end{equation}
dengan $\theta > 0$ adalah parameter skala dan $\gamma$ adalah parameter bentuk. Fungsi survival keseluruhan untuk $x > u$ adalah:
\begin{equation}
S(x) = S(u) \cdot G(x-u; \theta, \gamma) = S(u) \left(1 + \frac{\gamma(x-u)}{\theta}\right)^{-1/\gamma}, \quad x > u,
\label{eq:tlt_gpd_survival}
\end{equation}
yang memastikan kontinuitas fungsi survival pada $x = u$.

Fungsi kepadatan probabilitas untuk bagian GPD dapat diturunkan dari relasi $f(x) = -S'(x)$. Dengan menggunakan aturan rantai pada persamaan \eqref{eq:tlt_gpd_survival}:
\begin{equation}
f(x) = S(u) \cdot \frac{1}{\theta}\left(1 + \frac{\gamma(x-u)}{\theta}\right)^{-(1+1/\gamma)}, \quad x > u.
\label{eq:tlt_gpd_pdf}
\end{equation}

Fungsi hazard untuk bagian GPD diperoleh dari $h(x) = f(x)/S(x)$:
\begin{align}
h(x) &= \frac{f(x)}{S(x)} = \frac{S(u) \cdot \frac{1}{\theta}\left(1 + \frac{\gamma(x-u)}{\theta}\right)^{-(1+1/\gamma)}}{S(u) \left(1 + \frac{\gamma(x-u)}{\theta}\right)^{-1/\gamma}} \nonumber \\
&= \frac{1}{\theta + \gamma(x-u)}, \quad x > u.
\label{eq:tlt_gpd_hazard}
\end{align}

Parameter bentuk $\gamma$ menentukan karakteristik ekor distribusi: untuk $\gamma < 0$, distribusi memiliki batas atas (\textit{highest attained age}) $\omega = u + \theta/|\gamma|$, mengimplikasikan adanya usia maksimum yang dapat dicapai; untuk $\gamma = 0$, distribusi ekivalen dengan distribusi eksponensial; untuk $\gamma > 0$, distribusi memiliki ekor berat tanpa batas atas. Dalam konteks mortalitas manusia, nilai $\gamma < 0$ umumnya lebih masuk akal secara biologis \citep{dong2016evidence}.

\subsection{Estimasi Parameter Model TLT}

\subsubsection{Struktur Data Mortalitas}

Data mortalitas yang digunakan dalam estimasi model TLT umumnya berbentuk data agregat kohor dengan struktur sebagai berikut:
\begin{itemize}
    \item $d_x$: jumlah kematian yang terjadi antara usia $x$ dan $x+1$
    \item $l_x$: jumlah individu yang masih hidup pada awal usia $x$ (\textit{exposure})
    \item $x_{\text{min}}$: usia awal pengamatan (biasanya 65 tahun)
    \item $\tau$: usia maksimum pengamatan dalam data
\end{itemize}

Data seperti ini bersifat tersensor interval (\textit{interval-censored}) karena usia kematian dicatat dalam satuan tahun, dan tersensor kanan (\textit{right-censored}) karena beberapa individu mungkin masih hidup pada akhir periode pengamatan.

\subsubsection{Fungsi Likelihood dan Dekomposisi}

Fungsi likelihood untuk model TLT dengan parameter $\boldsymbol{\theta} = (B, C, \theta, \gamma, u)$ dikonstruksi berdasarkan probabilitas kematian pada setiap interval usia dan probabilitas bertahan hidup melampaui usia maksimum pengamatan. Dengan mengabaikan konstanta kombinatorial yang tidak mempengaruhi estimator Maximum Likelihood, fungsi likelihood dapat ditulis sebagai:
\begin{equation}
\begin{split}
L(B, C, \theta, \gamma; u) = &\prod_{x=x_{\text{min}}}^{u-1} \left(\frac{S(x) - S(x+1)}{S(x_{\text{min}})}\right)^{d_x} \\
&\times \prod_{x=u}^{\tau-1} \left(\frac{S(x) - S(x+1)}{S(x_{\text{min}})}\right)^{d_x} \\
&\times \left(\frac{S(\tau)}{S(x_{\text{min}})}\right)^{l_\tau},
\end{split}
\label{eq:tlt_likelihood_full}
\end{equation}
dimana fungsi survival $S(x)$ didefinisikan secara \textit{piecewise}:
\begin{equation}
S(x) =
\begin{cases}
\exp\left(-\frac{B}{\ln C}(C^x - 1)\right), & x \leq u \\[10pt]
\exp\left(-\frac{B}{\ln C}(C^u - 1)\right) \cdot \left(1 + \frac{\gamma(x-u)}{\theta}\right)^{-1/\gamma}, & x > u.
\end{cases}
\label{eq:tlt_survival_piecewise}
\end{equation}

Dengan mengambil logaritma dari persamaan \eqref{eq:tlt_likelihood_full}, diperoleh fungsi log-likelihood:
\begin{equation}
\begin{split}
\ell(B, C, \theta, \gamma; u) = &\sum_{x=x_{\text{min}}}^{u-1} d_x \ln(S(x) - S(x+1)) + \sum_{x=u}^{\tau-1} d_x \ln(S(x) - S(x+1)) \\
&+ l_\tau \ln(S(\tau)) - l_{x_{\text{min}}} \ln(S(x_{\text{min}})).
\end{split}
\label{eq:tlt_loglik_expanded}
\end{equation}

Salah satu sifat penting dari model TLT adalah bahwa fungsi log-likelihood dapat didekomposisi menjadi dua komponen independen karena tidak ada parameter yang dibagi antara bagian Gompertz dan bagian GPD:
\begin{equation}
\ell(B, C, \theta, \gamma; u) = \ell_1(B, C; u) + \ell_2(\theta, \gamma; u),
\label{eq:tlt_decomposition}
\end{equation}
dimana:
\begin{align}
\ell_1(B, C; u) &= \sum_{x=x_{\text{min}}}^{u-1} d_x \ln(S(x) - S(x+1)) + l_u \ln(S(u)) - l_{x_{\text{min}}} \ln(S(x_{\text{min}})), \label{eq:tlt_gompertz_loglik} \\
\ell_2(\theta, \gamma; u) &= \sum_{x=u}^{\tau-1} d_x \ln(S(x) - S(x+1)) + l_\tau \ln(S(\tau)) - l_u \ln(S(u)). \label{eq:tlt_gpd_loglik}
\end{align}

Dekomposisi ini memungkinkan estimasi parameter $(B, C)$ dan $(\theta, \gamma)$ dilakukan secara terpisah untuk setiap nilai $u$ yang tetap, yang meningkatkan efisiensi komputasi dalam prosedur optimisasi.

\subsubsection{Pemilihan Usia Ambang Optimal}

Pemilihan usia ambang $u$ dilakukan melalui pendekatan \textit{profile likelihood}. Untuk setiap kandidat nilai $u$ dalam rentang $\{u_{\min}, \ldots, u_{\max}\}$ (umumnya $u_{\min} = 85$ dan $u_{\max} = 100$), parameter model diestimasi dengan memaksimalkan fungsi log-likelihood \eqref{eq:tlt_decomposition}. Usia ambang optimal $\hat{u}$ dipilih sebagai nilai yang menghasilkan \textit{profile log-likelihood} maksimum:
\begin{equation}
\hat{u} = \arg\max_{u \in \{u_{\min}, \ldots, u_{\max}\}} \ell(\hat{B}(u), \hat{C}(u), \hat{\theta}(u), \hat{\gamma}(u); u),
\label{eq:profile_likelihood_optimal}
\end{equation}
dimana $\hat{B}(u), \hat{C}(u), \hat{\theta}(u), \hat{\gamma}(u)$ adalah estimator MLE yang bergantung pada nilai $u$.

\subsection{Keterbatasan Model TLT}

Meskipun model TLT memberikan pendekatan yang lebih objektif dan berbasis data untuk pemodelan mortalitas usia lanjut, model ini memiliki keterbatasan mendasar: \textbf{fungsi hazard tidak dijamin kontinu pada usia ambang $u$}.

Dari persamaan \eqref{eq:tlt_gompertz_hazard} dan \eqref{eq:tlt_gpd_hazard}, nilai fungsi hazard pada titik transisi $x = u$ adalah:
\begin{itemize}
    \item Limit dari kiri (Gompertz): $\lim_{x \to u^-} h(x) = BC^u$
    \item Limit dari kanan (GPD): $\lim_{x \to u^+} h(x) = \frac{1}{\theta}$
\end{itemize}

Secara umum, tidak terdapat jaminan bahwa $BC^u = \frac{1}{\theta}$. Dalam estimasi model TLT, parameter $(B, C)$ dan $(\theta, \gamma)$ diestimasi secara independen, sehingga kondisi kontinuitas tidak secara otomatis terpenuhi. Ketidaksamaan ini mengakibatkan diskontinuitas (loncatan) pada fungsi hazard:
\begin{equation}
\Delta h(u) = BC^u - \frac{1}{\theta} \neq 0.
\label{eq:hazard_jump}
\end{equation}

Loncatan mendadak dalam laju mortalitas sesaat tidak memiliki justifikasi biologis. Proses penuaan dan deteriorasi fisiologis bersifat gradual dan kontinu, sehingga laju mortalitas sesaat seharusnya juga berubah secara mulus. Keterbatasan ini memotivasi pengembangan model Smoothed Threshold Life Table (STLT) yang akan dibahas pada bagian selanjutnya, dimana \textit{smoothing constraint} diperkenalkan untuk memastikan kemulusan fungsi hazard pada usia ambang.
\section{Model Smoothed Threshold Life Table (STLT)}

Model Threshold Life Table (TLT) yang telah dibahas memberikan kerangka kerja yang solid untuk pemodelan mortalitas usia lanjut dengan mengintegrasikan Extreme Value Theory. Namun, model ini memiliki keterbatasan mendasar: \textbf{fungsi hazard tidak dijamin kontinu pada usia ambang $u$}, yang mengakibatkan loncatan mendadak dalam laju mortalitas sesaat. Keterbatasan ini memotivasi pengembangan model Smoothed Threshold Life Table (STLT) yang memastikan kemulusan fungsi hazard melalui penambahan \textit{smoothing constraint}.

\subsection{Masalah Diskontinuitas dan Derivasi Smoothing Constraint}

Dalam model TLT, fungsi hazard didefinisikan secara \textit{piecewise}: $h(x) = BC^x$ untuk $x \leq u$ (Gompertz) dan $h(x) = \frac{1}{\theta + \gamma(x-u)}$ untuk $x > u$ (GPD). Pada titik transisi $x = u$, nilai fungsi hazard dari kedua komponen adalah $\lim_{x \to u^-} h(x) = BC^u$ dan $\lim_{x \to u^+} h(x) = \frac{1}{\theta}$. Karena parameter $(B, C)$ dan $(\theta, \gamma)$ diestimasi secara independen, secara umum $BC^u \neq \frac{1}{\theta}$, yang mengakibatkan diskontinuitas:
\begin{equation}
\Delta h(u) = BC^u - \frac{1}{\theta} \neq 0.
\label{eq:stlt_hazard_discontinuity}
\end{equation}

Loncatan mendadak dalam laju mortalitas sesaat tidak memiliki justifikasi biologis, karena proses penuaan dan deteriorasi fisiologis bersifat gradual dan kontinu. Untuk mengatasi masalah ini, model STLT menambahkan kondisi kontinuitas eksplisit pada fungsi hazard di usia ambang:
\begin{equation}
BC^u = \frac{1}{\theta}.
\label{eq:smoothing_condition}
\end{equation}

Dari persamaan \eqref{eq:smoothing_condition}, diperoleh hubungan eksplisit antara parameter skala GPD dengan parameter Gompertz:
\begin{equation}
\boxed{\theta = \frac{1}{BC^u}}
\label{eq:theta_constraint}
\end{equation}

Persamaan \eqref{eq:theta_constraint} merupakan \textit{smoothing constraint} yang menjadi kunci model STLT. Hubungan ini menunjukkan bahwa parameter skala GPD ($\theta$) bukan lagi parameter bebas, melainkan fungsi deterministik dari parameter Gompertz $(B, C)$ dan usia ambang $u$. Dengan constraint ini, model STLT memiliki tiga parameter bebas $(B, C, \gamma)$ dan satu parameter turunan $\theta = \frac{1}{BC^u}$.

\subsection{Formulasi Model STLT}

Dengan menerapkan \textit{smoothing constraint}, model STLT dapat diformulasikan secara lengkap. Struktur model tetap \textit{piecewise}, menggabungkan Gompertz untuk $x \leq u$ dan GPD untuk $x > u$, namun dengan parameter yang saling terhubung.

\subsubsection{Komponen Gompertz ($x \leq u$)}

Untuk usia di bawah atau sama dengan usia ambang $u$, model STLT menggunakan distribusi Gompertz yang identik dengan model TLT. Fungsi survival adalah:
\begin{equation}
S(x) = \exp\left(-\frac{B}{\ln C}(C^x - 1)\right), \quad x \leq u,
\label{eq:stlt_gompertz_survival}
\end{equation}
dengan PDF $f(x) = BC^x \exp\left(-\frac{B}{\ln C}(C^x - 1)\right)$ dan fungsi hazard $h(x) = BC^x$. Parameter $B > 0$ mengatur tingkat mortalitas dasar, sementara $C > 1$ mengatur laju peningkatan mortalitas seiring bertambahnya usia.

\subsubsection{Komponen GPD ($x > u$)}

Untuk usia di atas usia ambang $u$, model STLT menggunakan GPD dengan parameter skala $\theta = \frac{1}{BC^u}$. Fungsi survival keseluruhan adalah:
\begin{equation}
S(x) = S(u) \left(1 + \gamma \cdot BC^u(x-u)\right)^{-1/\gamma}, \quad x > u,
\label{eq:stlt_gpd_survival}
\end{equation}
dengan batasan $x < \omega$ jika $\gamma < 0$. Fungsi PDF dapat diturunkan dari $f(x) = -S'(x)$:
\begin{equation}
f(x) = S(u) \cdot BC^u \left(1 + \gamma \cdot BC^u(x-u)\right)^{-(1+1/\gamma)}, \quad x > u.
\label{eq:stlt_gpd_pdf}
\end{equation}

Fungsi hazard untuk bagian GPD adalah:
\begin{equation}
h(x) = \frac{BC^u}{1 + \gamma \cdot BC^u(x-u)}, \quad x > u.
\label{eq:stlt_gpd_hazard}
\end{equation}

Kontinuitas fungsi hazard dapat diverifikasi dengan mengevaluasi limit: $\lim_{x \to u^+} h(x) = \frac{BC^u}{1 + 0} = BC^u$, yang sama dengan $\lim_{x \to u^-} h(x) = BC^u$, sehingga $h(u^-) = h(u^+) = BC^u$.

\paragraph{Derivasi Highest Attained Age.} Untuk kasus $\gamma < 0$, distribusi memiliki batas atas. Batas usia maksimum $\omega$ diperoleh dari kondisi bahwa argumen dalam fungsi survival harus non-negatif:
\begin{equation}
1 + \gamma \cdot BC^u(x-u) \geq 0.
\end{equation}
Karena $\gamma < 0$, kondisi ini ekivalen dengan:
\begin{align}
1 - |\gamma| \cdot BC^u(x-u) &\geq 0 \nonumber \\
|\gamma| \cdot BC^u(x-u) &\leq 1 \nonumber \\
x - u &\leq \frac{1}{BC^u|\gamma|} \nonumber \\
x &\leq u + \frac{1}{BC^u|\gamma|}.
\end{align}
Dengan substitusi $\theta = \frac{1}{BC^u}$, diperoleh:
\begin{equation}
\boxed{\omega = u + \frac{\theta}{|\gamma|} = u + \frac{1}{BC^u|\gamma|}}
\label{eq:stlt_omega_derivation}
\end{equation}
dimana $S(\omega) = 0$ dan $F(\omega) = 1$.

Parameter bentuk $\gamma$ menentukan karakteristik ekor distribusi: $\gamma < 0$ mengimplikasikan adanya batas atas usia (finite lifespan), $\gamma = 0$ ekivalen dengan distribusi eksponensial, dan $\gamma > 0$ mengimplikasikan ekor berat tanpa batas atas.

\subsection{Estimasi Parameter Model STLT}

Estimasi parameter model STLT menggunakan metode Maximum Likelihood Estimation (MLE) dengan prosedur \textit{profile likelihood} untuk pemilihan usia ambang optimal. Berbeda dengan model TLT yang dapat mendekomposisi likelihood menjadi dua komponen independen, model STLT memerlukan estimasi simultan parameter $(B, C, \gamma)$ karena adanya \textit{smoothing constraint} yang menghubungkan kedua komponen.

\subsubsection{Fungsi Likelihood}

Fungsi likelihood untuk model STLT dengan parameter $(B, C, \gamma, u)$ dikonstruksi berdasarkan data agregat kohor dengan struktur yang sama seperti TLT. Dengan mengabaikan konstanta kombinatorial, fungsi log-likelihood adalah:
\begin{equation}
\ell(B, C, \gamma; u) = \sum_{x=x_{\text{min}}}^{\tau-1} d_x \ln(S(x) - S(x+1)) + l_\tau \ln(S(\tau)) - l_{x_{\text{min}}} \ln(S(x_{\text{min}})),
\label{eq:stlt_loglikelihood}
\end{equation}
dimana fungsi survival $S(x)$ didefinisikan secara \textit{piecewise} dengan constraint $\theta = \frac{1}{BC^u}$.

Karena constraint menghubungkan parameter Gompertz dan GPD, likelihood tidak dapat didekomposisi seperti pada TLT. Estimasi parameter $(B, C, \gamma)$ harus dilakukan secara simultan untuk setiap nilai $u$ yang tetap.

\subsubsection{Pemilihan Usia Ambang Optimal}

Pemilihan usia ambang $u$ menggunakan pendekatan yang sama dengan TLT, yaitu \textit{profile likelihood}. Untuk setiap kandidat nilai $u$ dalam rentang $\{u_{\min}, \ldots, u_{\max}\}$ (umumnya $u_{\min} = 85$ dan $u_{\max} = 100$), parameter $(B, C, \gamma)$ diestimasi dengan memaksimalkan fungsi log-likelihood \eqref{eq:stlt_loglikelihood}. Usia ambang optimal $\hat{u}$ dipilih sebagai nilai yang menghasilkan \textit{profile log-likelihood} maksimum:
\begin{equation}
\hat{u} = \arg\max_{u \in \{u_{\min}, \ldots, u_{\max}\}} \ell(\hat{B}(u), \hat{C}(u), \hat{\gamma}(u); u).
\label{eq:stlt_profile_likelihood_optimal}
\end{equation}

\textbf{Perbedaan dengan TLT:} Dalam TLT, untuk setiap nilai $u$ tetap, parameter $(B, C)$ dan $(\theta, \gamma)$ dapat diestimasi secara terpisah karena dekomposisi likelihood. Dalam STLT, parameter $(B, C, \gamma)$ harus diestimasi secara simultan menggunakan algoritma optimisasi numerik (misalnya, L-BFGS-B) karena constraint $\theta = \frac{1}{BC^u}$ menghubungkan kedua komponen. Hal ini meningkatkan kompleksitas komputasi, namun memastikan kontinuitas fungsi hazard.

Setelah estimasi, parameter turunan dihitung sebagai $\hat{\theta} = \frac{1}{\hat{B}\hat{C}^{\hat{u}}}$, dan untuk $\hat{\gamma} < 0$, batas atas usia dihitung sebagai $\hat{\omega} = \hat{u} + \frac{\hat{\theta}}{|\hat{\gamma}|}$. Untuk stabilitas numerik, dapat digunakan reparameterisasi $\alpha = \ln B$ dan $\delta = \ln(\ln C)$ sehingga constraint positivity otomatis terpenuhi.
\section{Model Dynamic Smooth Threshold Life Table (DSTLT)}

Model STLT yang telah dibahas berhasil mengatasi masalah diskontinuitas fungsi \textit{hazard} yang terdapat pada model TLT. Namun, model STLT tetap bersifat statis, yang berarti model diestimasi secara terpisah untuk setiap kohor tanpa mempertimbangkan hubungan antar kohor. Keterbatasan ini membatasi kemampuan model STLT untuk melakukan peramalan mortalitas masa depan, yang merupakan kebutuhan krusial dalam aplikasi aktuaria seperti \textit{pricing} produk anuitas dan manajemen risiko longevitas.

Bagian ini membahas pengembangan model DSTLT yang mengintegrasikan komponen dinamis ke dalam model STLT, dimotivasi oleh observasi empiris terhadap pola temporal parameter model STLT antar kohor.

\subsection{Keterbatasan Model Statis dan Motivasi Pengembangan DSTLT}

Model statis seperti STLT memiliki beberapa keterbatasan fundamental untuk aplikasi peramalan mortalitas:

\begin{enumerate}
    \item \textbf{Tidak menangkap tren temporal}: Estimasi dilakukan secara independen untuk setiap kohor, sehingga pola sistematik perubahan mortalitas antar generasi tidak teridentifikasi.

    \item \textbf{Tidak dapat melakukan proyeksi}: Untuk memproyeksikan mortalitas kohor masa depan, model statis memerlukan asumsi tambahan yang bersifat \textit{ad hoc}.

    \item \textbf{Keterbatasan praktis}: Dalam aplikasi aktuaria seperti penilaian kewajiban dana pensiun multi-generasi, diperlukan proyeksi mortalitas yang konsisten untuk kohor-kohor yang berbeda, yang tidak dapat diberikan oleh model statis.
\end{enumerate}

\subsection{Temuan Empiris Huang et al. (2020)}

\citet{huang2020modelling} melakukan analisis sistematis terhadap parameter model STLT yang diestimasi untuk kohor kelahiran 1893--1908 di Belanda. Analisis ini mengungkapkan beberapa pola temporal yang penting:

\begin{enumerate}
    \item \textbf{Parameter $B$ menunjukkan tren temporal yang jelas}: Untuk kohor perempuan, nilai $\hat{B}$ menurun secara konsisten dari 0.000031 (kohor 1893) menjadi 0.000015 (kohor 1901), suatu penurunan sekitar 52\%. Pola ini mengindikasikan adanya perbaikan mortalitas yang sistematis pada tingkat mortalitas dasar (\textit{baseline mortality level}) antar generasi.

    \item \textbf{Parameter $C$ menunjukkan variasi yang jauh lebih kecil}: Perubahan parameter $C$ kurang dari 1\% antar kohor, sementara perubahan $B$ mencapai lebih dari 50\%. Hal ini menunjukkan bahwa perbaikan mortalitas terutama terjadi pada tingkat mortalitas dasar, bukan pada laju akselerasi mortalitas.

    \item \textbf{Parameter $\gamma$, $\theta$, dan $u$ tidak menunjukkan pola temporal yang konsisten}: Nilai-nilai parameter ini berfluktuasi tanpa arah yang sistematis antar kohor. Temuan ini konsisten dengan hasil \citet{einmahl2019modeling} pada data yang sama.
\end{enumerate}

Berdasarkan observasi empiris ini, Huang et al. memutuskan untuk memodelkan parameter $B$ sebagai fungsi yang bervariasi antar kohor, sementara parameter lainnya ($\gamma$, $\theta$, $u$) diperlakukan sebagai konstan antar kohor.

\subsection{Pemilihan Parameter Dinamis dan Konsistensi dengan Cairns et al.}

Keputusan untuk mendinamisasi parameter $B$ didasarkan pada beberapa pertimbangan:

\begin{enumerate}
    \item \textbf{Bukti empiris yang kuat}: Tren temporal parameter $B$ jelas dan konsisten, dengan variasi antar kohor yang signifikan (> 50\%).

    \item \textbf{Interpretasi yang masuk akal}: Parameter $B$ merepresentasikan tingkat mortalitas dasar, yang secara intuitif dapat berubah antar generasi akibat perbaikan kondisi kesehatan, nutrisi, dan layanan medis.

    \item \textbf{Konsistensi dengan literatur}: Pemilihan ini konsisten dengan penelitian \citet{cairns2006pricing} dalam model Cairns-Blake-Dowd (CBD). Cairns et al. memodelkan komponen tingkat mortalitas sebagai \textit{random walk with drift}, yang secara konseptual serupa dengan spesifikasi $B_i = \exp(a + bi)$ dalam DSTLT.
\end{enumerate}

Model CBD, yang merupakan salah satu model mortalitas stokastik paling berpengaruh dalam aktuaria, juga mengidentifikasi bahwa komponen tingkat mortalitas (bukan slope) yang memiliki tren temporal paling dominan. Temuan Huang et al. untuk parameter $B$ dalam STLT paralel dengan temuan Cairns et al. untuk parameter $\kappa_t^{(1)}$ dalam CBD, memberikan validasi silang dari dua pendekatan metodologi yang berbeda.

\subsection{Formulasi Model DSTLT}

Model DSTLT mempertahankan struktur \textit{piecewise} dari STLT, namun dengan parameter Gompertz yang bervariasi antar kohor. Misalkan $i$ adalah indeks kohor, model DSTLT didefinisikan sebagai berikut:

\subsubsection{Struktur Parameter Dinamis}

Parameter model DSTLT terdiri dari:
\begin{itemize}
    \item \textbf{Parameter dinamis}: $B_i = \exp(a + bi)$, dimana $a$ dan $b$ adalah parameter yang mengatur level dan tren temporal mortalitas dasar.
    \item \textbf{Parameter konstan}: $\gamma$, $\theta$, dan $u$ diasumsikan konstan antar kohor.
\end{itemize}

Dengan \textit{smoothing constraint}, parameter $C_i$ tidak lagi independen tetapi ditentukan secara implisit melalui hubungan $\theta = \frac{1}{B_i C_i^u}$, yang dapat ditulis sebagai:
\begin{equation}
C_i = \left(\frac{1}{B_i \theta}\right)^{1/u} = \left(\frac{1}{\exp(a+bi) \theta}\right)^{1/u}.
\label{eq:dstlt_c_constraint}
\end{equation}

Dengan demikian, model DSTLT memiliki empat parameter bebas: $(a, b, \gamma, u)$, dengan $\theta$ dan $C_i$ ditentukan oleh constraint.

\subsubsection{Komponen Model untuk Kohor $i$}

Untuk kohor $i$, fungsi survival didefinisikan secara \textit{piecewise}:

\paragraph{Bagian Gompertz ($x \leq u$):}
\begin{equation}
S_i(x) = \exp\left(-\frac{B_i}{\ln C_i}(C_i^x - 1)\right), \quad x \leq u,
\label{eq:dstlt_gompertz_survival}
\end{equation}
dengan $B_i = \exp(a + bi)$ dan $C_i$ diberikan oleh persamaan \eqref{eq:dstlt_c_constraint}.

\paragraph{Bagian GPD ($x > u$):}
\begin{equation}
S_i(x) = S_i(u) \left(1 + \gamma \cdot B_i C_i^u(x-u)\right)^{-1/\gamma}, \quad x > u.
\label{eq:dstlt_gpd_survival}
\end{equation}

Perhatikan bahwa $B_i C_i^u = \frac{1}{\theta}$ untuk semua $i$ karena \textit{smoothing constraint}, sehingga bagian GPD menjadi:
\begin{equation}
S_i(x) = S_i(u) \left(1 + \frac{\gamma(x-u)}{\theta}\right)^{-1/\gamma}, \quad x > u.
\label{eq:dstlt_gpd_survival_simplified}
\end{equation}

Formulasi ini menunjukkan bahwa variasi antar kohor hanya terjadi pada bagian Gompertz (melalui $B_i$ dan $C_i$), sementara bagian GPD tetap sama untuk semua kohor. Hal ini konsisten dengan observasi empiris bahwa karakteristik ekor distribusi ($\gamma$) tidak menunjukkan tren temporal yang jelas.

\subsection{Estimasi Parameter Model DSTLT}

Estimasi parameter model DSTLT menggunakan data gabungan dari beberapa kohor dalam periode \textit{training}. Dengan asumsi bahwa kohor-kohor independen, fungsi log-likelihood gabungan adalah:
\begin{equation}
\ell(a, b, \gamma; u) = \sum_{i=1}^{r} \ell_i(a, b, \gamma; u),
\label{eq:dstlt_joint_loglikelihood}
\end{equation}
dimana $r$ adalah jumlah kohor dalam \textit{training set}, dan $\ell_i(a, b, \gamma; u)$ adalah log-likelihood untuk kohor $i$:
\begin{equation}
\begin{split}
\ell_i(a, b, \gamma; u) = &\sum_{x=x_{\text{min}}}^{\tau_i-1} d_{x,i} \ln(S_i(x) - S_i(x+1)) \\
&+ l_{\tau_i,i} \ln(S_i(\tau_i)) - l_{x_{\text{min}},i} \ln(S_i(x_{\text{min}})).
\end{split}
\label{eq:dstlt_single_loglikelihood}
\end{equation}

Parameter $(a, b, \gamma)$ diestimasi dengan memaksimalkan fungsi log-likelihood gabungan \eqref{eq:dstlt_joint_loglikelihood} untuk setiap nilai $u$ tetap. Usia ambang optimal $\hat{u}$ dipilih menggunakan \textit{profile likelihood}, serupa dengan STLT. Setelah estimasi, parameter $\theta$ dihitung secara konsisten dari constraint, dan proyeksi untuk kohor masa depan dilakukan dengan mengekstrapolasi $B_i = \exp(a + bi)$ untuk nilai $i$ yang lebih besar.

Perbedaan utama dengan STLT adalah bahwa DSTLT melakukan estimasi secara simultan untuk semua kohor dalam \textit{training set}, sehingga dapat menangkap tren temporal dan melakukan proyeksi untuk kohor masa depan secara konsisten.

\chapter{ANALISIS EMPIRIS DAN EVALUASI MODEL}

Bab ini menyajikan hasil analisis empiris dari penerapan model \textit{Smooth Threshold Life Table} (STLT) dan \textit{Dynamic Smooth Threshold Life Table} (DSTLT) pada data mortalitas.

Sebagaimana telah diuraikan pada bab sebelumnya, model STLT dikembangkan untuk mengatasi keterbatasan model \textit{Threshold Life Table} (TLT) terkait potensi diskontinuitas fungsi \textit{hazard} pada usia ambang. Penambahan \textit{smoothing constraint} memastikan transisi yang mulus antara komponen Gompertz pada usia non-ekstrem dan komponen \textit{Generalized Pareto Distribution} (GPD) pada usia lanjut ekstrem. Lebih lanjut, model DSTLT mengembangkan model STLT dengan menambahkan komponen dinamis yang memungkinkan parameter mortalitas bervariasi antar kohor, sehingga mampu menangkap tren mortalitas dan melakukan peramalan.

Terdapat beberapa tujuan analisis empiris. Pertama, melakukan estimasi parameter model STLT dan DSTLT menggunakan metode \textit{Maximum Likelihood Estimation} (MLE) pada data mortalitas historis. Kedua, mengevaluasi kinerja model STLT dalam hal kesesuaian terhadap data (\textit{goodness-of-fit}) dengan membandingkannya terhadap berbagai model mortalitas statis yang telah mapan, yaitu model Gompertz-Makeham, Heligman-Pollard, dan Coale-Kisker. Ketiga, menganalisis kinerja model DSTLT baik dalam aspek \textit{in-sample fit} maupun kemampuan peramalan \textit{out-of-sample} (\textit{out-of-sample forecasting}), dengan menggunakan model Cairns-Blake-Dowd (CBD) sebagai \textit{benchmark}.

Seluruh analisis empiris dalam bab ini diimplementasikan menggunakan perangkat lunak statistik R versi 4.3.1, dengan memanfaatkan berbagai \textit{package} standar untuk optimisasi numerik dan visualisasi data. Kode sumber disediakan dalam lampiran.

\section{Deskripsi Data}

Bagian ini menjelaskan secara rinci mengenai data mortalitas yang digunakan dalam penelitian ini, mencakup sumber data, karakteristik populasi yang dianalisis, struktur data, serta prosedur pra-pemrosesan dan validasi data.

\subsection{Sumber Data}

Data mortalitas yang digunakan dalam penelitian ini bersumber dari \textit{Human Mortality Database} (HMD), sebuah basis data internasional yang menyediakan data mortalitas tervalidasi dan terstandarisasi untuk berbagai negara. HMD dikelola bersama oleh \textit{University of California, Berkeley} dan \textit{Max Planck Institute for Demographic Research}, dan telah menjadi standar \textit{de facto} dalam penelitian mortalitas akademik dan aplikasi aktuaria \citep{HMD2023}.

\subsection{Populasi dan Periode Observasi}

Penelitian ini menggunakan data mortalitas dari dua* negara dengan karakteristik mortalitas usia lanjut yang berbeda, yaitu Jepang dan Belanda. Pemilihan kedua negara ini didasarkan pada beberapa pertimbangan:

\begin{enumerate}
    \item \textbf{Kualitas Data}: Kedua negara memiliki sistem registrasi vital yang sangat baik dan prosedur validasi usia yang ketat, sehingga menghasilkan data mortalitas usia lanjut yang reliabel.
    
    \item \textbf{Volume Data Ekstrem}: Kedua negara memiliki jumlah individu yang mencapai usia lanjut ekstrem (di atas 100 tahun) yang memadai untuk estimasi parameter model berbasis \textit{Extreme Value Theory}.
    
    \item \textbf{Karakteristik Berbeda}: Jepang dikenal memiliki harapan hidup tertinggi di dunia dengan fenomena \textit{mortality deceleration} yang sangat jelas pada usia lanjut, sedangkan Belanda merepresentasikan pola mortalitas Eropa Barat dengan tren perbaikan mortalitas yang stabil.
    
    \item \textbf{Komparabilitas dengan Literatur}: Kedua negara ini telah digunakan secara ekstensif dalam literatur pemodelan mortalitas usia lanjut, termasuk dalam studi \citet{Huang2020}, sehingga memungkinkan validasi silang hasil penelitian.
\end{enumerate}

\subsection{Struktur Data}

Data mortalitas untuk setiap kohor diorganisasikan dalam format tabel mortalitas kohor (\textit{cohort life table}), yang melacak pengalaman mortalitas sekelompok individu yang lahir pada tahun yang sama sepanjang siklus hidup mereka. Struktur data untuk setiap kohor $i$ mencakup variabel-variabel berikut:

\begin{itemize}
    \item $x$: Usia dalam tahun penuh $(x = 0, 1, 2, \ldots, \omega)$
    \item $l_{x,i}$: Jumlah individu yang masih hidup pada awal usia $x$ dalam kohor $i$
    \item $d_{x,i}$: Jumlah kematian antara usia $x$ dan $x+1$ dalam kohor $i$
    \item $q_{x,i}$: Probabilitas kematian antara usia $x$ dan $x+1$ dalam kohor $i$
    \item $e_{x,i}$: Harapan hidup pada usia $x$ dalam kohor $i$
\end{itemize}

Untuk keperluan estimasi model STLT dan DSTLT, analisis dibatasi pada rentang usia $x \geq 65$ tahun. Pembatasan ini juga mengurangi kompleksitas komputasi tanpa mengorbankan kemampuan model dalam menangkap fenomena mortalitas usia lanjut ekstrem yang menjadi fokus utama penelitian.


\subsection{Pra-Pemrosesan Data}

Beberapa prosedur pra-pemrosesan diterapkan untuk memastikan kualitas dan konsistensi data sebelum dilakukan estimasi parameter model:

\subsubsection{Penyesuaian Kohor}
Data HMD pada dasarnya dipengaruhi oleh migrasi, yang dapat mengintroduksi bias dalam analisis kohor. Untuk mengeliminasi efek migrasi, dilakukan penyesuaian data mengikuti prosedur yang diusulkan oleh \citet{Huang2020}. Penyesuaian ini mengasumsikan bahwa tidak terjadi migrasi neto pada populasi, sehingga perubahan ukuran kohor semata-mata disebabkan oleh mortalitas. Secara matematis, jumlah individu yang hidup pada usia $x+1$ dihitung sebagai:
\begin{equation}
l_{x+1,i}^{\text{adj}} = l_{x,i}^{\text{adj}} - d_{x,i}
\end{equation}
dengan $l_{65,i}^{\text{adj}}$ diinisialisasi dari data aktual HMD pada usia 65 tahun.

\subsubsection{Penanganan Data Tersensor}
Mengingat periode observasi yang terbatas, tidak semua individu dalam kohor diamati hingga kematian. Data tersensor kanan (\textit{right-censored}) muncul ketika individu masih hidup pada akhir periode observasi. Selain itu, data mortalitas pada usia-usia tinggi seringkali dilaporkan dalam interval usia terbuka, misalnya ``110+'' yang mencakup semua individu berusia 110 tahun atau lebih. Hal ini menghasilkan data tersensor interval (\textit{interval-censored}). 

Untuk menangani kedua jenis sensor ini, fungsi \textit{likelihood} yang digunakan dalam estimasi parameter (seperti yang telah diturunkan pada Sub-bab 3.2.4) secara eksplisit mengakomodasi kontribusi dari observasi tersensor. Observasi tersensor kanan berkontribusi pada \textit{likelihood} melalui fungsi \textit{survival} $S(\tau)$, di mana $\tau$ adalah usia tertinggi yang diamati dalam kohor.

\section{Estimasi Parameter Model STLT}

Bagian ini menyajikan hasil estimasi parameter model \textit{Smooth Threshold Life Table} (STLT) untuk berbagai kohor dari kedua negara yang dianalisis. Estimasi dilakukan menggunakan metode \textit{Maximum Likelihood Estimation} (MLE) sebagaimana telah dijelaskan pada Sub-bab 3.2.4. Proses estimasi mencakup penentuan parameter Gompertz ($B$, $C$), parameter GPD ($\gamma$), serta pemilihan usia ambang optimal ($N$), dengan parameter skala GPD ($\theta$) ditentukan secara implisit melalui \textit{smoothing constraint} $\theta = 1/(BC^N)$.

% --------------------------------------------
\subsection{Prosedur Estimasi}

Untuk setiap kohor, estimasi parameter model STLT dilakukan melalui algoritma optimisasi numerik yang memaksimalkan fungsi log-\textit{likelihood} yang telah diturunkan pada Persamaan (XX) di Bab 3. Mengingat kompleksitas fungsi \textit{likelihood} dan adanya \textit{constraint} antar parameter, digunakan pendekatan \textit{profile likelihood} untuk usia ambang $N$.

Secara spesifik, prosedur estimasi dilaksanakan sebagai berikut:

\begin{enumerate}
    \item \textbf{Pencarian batas untuk Usia Ambang}: Untuk setiap nilai kandidat usia ambang $N$ dalam rentang $[85, 110]$ dengan interval 1 tahun, dilakukan optimisasi parameter $B$, $C$, dan $\gamma$ menggunakan metode \textit{Nelder-Mead simplex}.
    
    \item \textbf{Inisialisasi Parameter}: Nilai awal parameter untuk optimisasi ditentukan berdasarkan estimasi kasar dari data empiris. Parameter $B$ diinisialisasi menggunakan regresi linear sederhana pada $\log(q_x)$ untuk rentang usia 65--85 tahun. Parameter $C$ diinisialisasi mendekati nilai 1.1 yang merupakan nilai tipikal dalam literatur mortalitas. Parameter $\gamma$ diinisialisasi pada nilai $-0.2$, konsisten dengan temuan empiris \textit{mortality deceleration} pada usia lanjut ekstrem.
    
    \item \textbf{Optimisasi Numerik}: Untuk setiap $N$ tetap, algoritma optimisasi memaksimalkan log-\textit{likelihood} terhadap parameter bebas $(B, C, \gamma)$ dengan mempertimbangkan \textit{constraint} $\theta = 1/(BC^N)$. Konvergensi dinyatakan tercapai ketika perubahan relatif dalam nilai \textit{likelihood} antara iterasi berturut-turut kurang dari $10^{-8}$.
    
    \item \textbf{Pemilihan Usia Ambang Optimal}: Setelah mendapatkan estimasi parameter untuk semua nilai $N$ kandidat, usia ambang optimal dipilih sebagai nilai $\hat{N}$ yang menghasilkan nilai log-\textit{likelihood} maksimum. Estimasi parameter final model STLT adalah $(\hat{B}, \hat{C}, \hat{\gamma}, \hat{N})$ yang bersesuaian dengan $\hat{N}$ optimal ini.
    
    \item \textbf{Perhitungan \textit{Highest Attained Age}}: Untuk kasus $\hat{\gamma} < 0$, batas atas distribusi usia (\textit{highest attained age}, $\omega$) dihitung menggunakan formula $\omega = \hat{N} + \hat{\theta}/|\hat{\gamma}|$, dengan $\hat{\theta} = 1/(\hat{B}\hat{C}^{\hat{N}})$.
\end{enumerate}

Seluruh prosedur estimasi diimplementasikan menggunakan bahasa pemrograman R versi 4.3.1, dengan memanfaatkan fungsi \texttt{optim()} untuk optimisasi numerik.

% --------------------------------------------
\subsection{Hasil Estimasi Parameter}

Tabel \ref{tab:stlt_params_selected_cohorts} menyajikan hasil estimasi parameter model STLT untuk beberapa kohor terpilih dari Jepang dan Belanda. Untuk efisiensi penyajian, ditampilkan hasil untuk kohor-kohor representatif (1893, 1899, 1905, dan 1908), sementara hasil lengkap untuk seluruh 16 kohor tersedia dalam Lampiran A.

\begin{table}[H]
\centering
\caption{Estimasi Parameter Model STLT untuk Kohor Terpilih}
\label{tab:stlt_params_selected_cohorts}
\small
\begin{tabular}{ccccccc}
\hline
\textbf{Negara} & \textbf{Kohor} & $\boldsymbol{\hat{B}}$ & $\boldsymbol{\hat{C}}$ & $\boldsymbol{\hat{\gamma}}$ & $\boldsymbol{\hat{N}}$ & $\boldsymbol{\hat{\omega}}$ \\
\hline
\multirow{4}{*}{\textbf{Jepang}} 
& 1893 & $2.98 \times 10^{-5}$ & 1.1024 & $-0.2425$ & 93 & 109.0 \\
& 1899 & $2.52 \times 10^{-5}$ & 1.1030 & $+0.2304$ & 104 & 97.6 \\
& 1905 & $9.85 \times 10^{-6}$ & 1.1140 & $-0.1468$ & 98 & 115.6 \\
& 1908 & $7.45 \times 10^{-6}$ & 1.1169 & $-0.1704$ & 97 & 114.3 \\
\hline
\multirow{4}{*}{\textbf{Belanda}} 
& 1893 & $3.09 \times 10^{-5}$ & 1.1011 & $-0.0738$ & 101 & 127.0 \\
& 1899 & $1.99 \times 10^{-5}$ & 1.1060 & $-0.1005$ & 102 & 119.2 \\
& 1905 & $9.68 \times 10^{-6}$ & 1.1148 & $-0.1274$ & 98 & 117.3 \\
& 1908 & $7.34 \times 10^{-6}$ & 1.1182 & $-0.1958$ & 95 & 112.1 \\
\hline
\end{tabular}
\begin{tablenotes}
\small
\item \textit{Catatan}: $\hat{\omega}$ merepresentasikan \textit{highest attained age} yang dihitung sebagai $\hat{N} + \hat{\theta}/|\hat{\gamma}|$ untuk $\hat{\gamma} < 0$. Untuk kohor Jepang 1899 dengan $\hat{\gamma} > 0$, nilai $\hat{\omega}$ menunjukkan estimasi berdasarkan persentil tinggi dari distribusi GPD.
\end{tablenotes}
\end{table}



\subsection{Perbandingan STLT dengan TLT}

Untuk mengevaluasi dampak penambahan \textit{smoothing constraint} pada model TLT, Tabel \ref{tab:stlt_vs_tlt_comparison} membandingkan estimasi parameter model STLT dengan model TLT untuk kohor terpilih Belanda 1901 dan Jepang 1962.

\begin{table}[H]
\centering
\caption{Perbandingan Estimasi Parameter STLT dan TLT}
\label{tab:stlt_vs_tlt_comparison}
\small
\begin{tabular}{clcccccc}
\hline
\textbf{Negara} & \textbf{Model} & $\boldsymbol{\hat{B}}$ & $\boldsymbol{\hat{C}}$ & $\boldsymbol{\hat{\gamma}}$ & $\boldsymbol{\hat{N}}$ & $\boldsymbol{\hat{\omega}}$ & \textbf{Log-Lik} \\
\hline
\multirow{2}{*}{\textbf{Belanda 1901}} 
& STLT & $1.54 \times 10^{-5}$ & 1.1090 & $-0.1579$ & 98 & 114.2 & $-2847.3$ \\
& TLT & $2.20 \times 10^{-6}$ & 1.1339 & $-0.1438$ & 100 & 115.6 & $-2851.8$ \\
\hline
\multirow{2}{*}{\textbf{Jepang 1962}} 
& STLT & $1.58 \times 10^{-5}$ & 1.1165 & $-0.1620$ & 89 & 110.5 & $-3124.6$ \\
& TLT & $1.37 \times 10^{-6}$ & 1.1494 & $-0.1542$ & 100 & 112.0 & $-3142.9$ \\
\hline
\end{tabular}
\begin{tablenotes}
\small
\item \textit{Catatan}: Nilai Log-Lik merepresentasikan nilai log-\textit{likelihood} maksimum untuk masing-masing model. Model dengan nilai log-\textit{likelihood} lebih tinggi (kurang negatif) mengindikasikan kesesuaian yang lebih baik terhadap data.
\end{tablenotes}
\end{table}

Beberapa observasi penting dari perbandingan ini:

\begin{enumerate}
    \item \textbf{Kesesuaian Model}: Model STLT secara konsisten menghasilkan nilai log-\textit{likelihood} yang lebih tinggi dibandingkan model TLT, mengindikasikan kesesuaian yang lebih baik terhadap data observasi. Untuk Belanda 1901, perbedaan log-\textit{likelihood} adalah $4.5$ ($-2847.3$ vs $-2851.8$), sementara untuk Jepang 1962 perbedaannya adalah $18.3$ ($-3124.6$ vs $-3142.9$). Perbaikan ini menunjukkan bahwa \textit{smoothing constraint} memang meningkatkan kemampuan model dalam menangkap pola mortalitas.
    
    \item \textbf{Perbedaan Estimasi Parameter}: Terdapat perbedaan substansial dalam estimasi parameter antara STLT dan TLT. Secara khusus, nilai $\hat{B}$ untuk STLT jauh lebih tinggi (sekitar 7--11 kali lipat) dibandingkan TLT. Sebaliknya, nilai $\hat{C}$ untuk STLT cenderung lebih rendah. Perbedaan ini mencerminkan redistribusi peran parameter dalam menjelaskan pola mortalitas ketika \textit{constraint} $\theta = 1/(BC^N)$ dikenakan.
    
    \item \textbf{Usia Ambang Optimal}: Model STLT cenderung memilih usia ambang $\hat{N}$ yang lebih rendah (89--98 tahun) dibandingkan TLT yang umumnya konvergen pada $N = 100$ tahun. Hal ini mengindikasikan bahwa dengan adanya \textit{smoothing constraint}, transisi dari Gompertz ke GPD dapat terjadi pada usia yang lebih muda sambil tetap mempertahankan kontinuitas fungsi \textit{hazard}.
    
    \item \textbf{Implikasi untuk Fungsi \textit{Hazard}}: Meskipun terdapat perbedaan parameter, kedua model menghasilkan estimasi $\hat{\omega}$ yang relatif sebanding, dengan perbedaan maksimum sekitar 3 tahun. Hal ini menunjukkan bahwa meskipun mekanisme parameterisasi berbeda, kedua model menangkap karakteristik ekor distribusi dengan cara yang serupa.
\end{enumerate}

Superioritas model STLT dalam hal nilai \textit{likelihood} memberikan justifikasi empiris untuk penggunaan \textit{smoothing constraint}, sejalan dengan argumentasi teoretis yang telah dikemukakan pada Bab 3.

% --------------------------------------------
\subsection{Visualisasi Fungsi \textit{Hazard} dan Distribusi}

Gambar \ref{fig:stlt_hazard_comparison} mengilustrasikan perbandingan fungsi \textit{hazard} estimasi antara model STLT dan TLT untuk kohor Belanda 1901, bersama dengan \textit{hazard rate} empiris yang dihitung dari data aktual.

\begin{figure}[htbp]
\centering
% Placeholder untuk grafik - ganti dengan \includegraphics setelah gambar tersedia
\fbox{\parbox{0.8\textwidth}{\centering
\vspace{4cm}
[Grafik Perbandingan Fungsi Hazard STLT vs TLT vs Data Empiris]\\
Panel Atas: Fungsi hazard $h(x)$\\
Panel Bawah: Zoom pada transisi di sekitar usia ambang\\
\vspace{4cm}
}}
\caption{Perbandingan Fungsi \textit{Hazard} Model STLT dan TLT untuk Kohor Belanda 1901}
\label{fig:stlt_hazard_comparison}
\begin{fignotes}
\small
\item \textit{Catatan}: Titik-titik merepresentasikan estimasi empiris \textit{hazard rate} $\hat{h}(x) = d_x/l_x$. Garis merah solid menunjukkan fungsi \textit{hazard} model STLT, sementara garis biru putus-putus menunjukkan model TLT. Panel bawah memperbesar wilayah transisi di sekitar usia ambang untuk menyoroti perbedaan kontinuitas.
\end{fignotes}
\end{figure}

Visualisasi ini mengkonfirmasi bahwa:
\begin{itemize}
    \item Model STLT menghasilkan fungsi \textit{hazard} yang kontinu dan mulus pada usia ambang $N$, sesuai dengan desain model.
    \item Model TLT menunjukkan diskontinuitas (\textit{jump}) yang terlihat jelas pada usia ambang, mencerminkan ketidakkonsistenan antara komponen Gompertz dan GPD.
    \item Kedua model memberikan kesesuaian yang baik terhadap \textit{hazard rate} empiris pada rentang usia dengan data yang memadai (di bawah 105 tahun).
    \item Pada usia lanjut ekstrem (di atas 105 tahun), ketidakpastian estimasi meningkat yang tercermin dari variabilitas tinggi \textit{hazard rate} empiris.
\end{itemize}

Kehalusan fungsi \textit{hazard} model STLT tidak hanya lebih estetis secara matematis, tetapi juga lebih masuk akal secara biologis, karena tidak ada alasan teoretis untuk ekspektasi terjadinya lompatan mendadak dalam tingkat mortalitas pada usia tertentu.

\section{Perbandingan dengan Model Mortalitas Statis Lainnya}

Untuk mengevaluasi kinerja model STLT secara komprehensif, dilakukan perbandingan dengan berbagai model mortalitas statis yang telah mapan dalam literatur aktuaria dan demografi. Model-model pembanding yang dipilih merepresentasikan pendekatan pemodelan yang berbeda dan telah digunakan secara luas dalam praktik, baik oleh industri asuransi maupun lembaga pemerintah. Bagian ini menyajikan hasil perbandingan tersebut berdasarkan berbagai metrik evaluasi \textit{goodness-of-fit}. Model-model statis yang digunakan sebagai pembanding dalam analisis ini adalah model Gompertz-Makeham, Heligman-Pollard dan Coale-Kisker. 

Untuk penelitian ini, estimasi parameter model Gompertz-Makeham dan Heligman-Pollard dilakukan menggunakan metode \textit{Maximum Likelihood Estimation} pada data mortalitas yang sama dengan yang digunakan untuk model STLT (usia 65 tahun ke atas). Untuk metode Coale-Kisker, titik awal ekstrapolasi ditetapkan pada usia 85 tahun, dengan titik akhir ditetapkan pada usia tertinggi yang diamati dalam kohor ($\tau$), dan $m_\tau = 1$ sebagai kondisi batas.


\subsection{Metrik Evaluasi}

Perbandingan kinerja model dilakukan menggunakan empat metrik evaluasi:

\paragraph{\textit{Mean Absolute Error} (MAE)}
MAE mengukur rata-rata deviasi absolut antara probabilitas kematian observasi dan estimasi:
\begin{equation}
\text{MAE} = \frac{1}{\tau - 65 + 1} \sum_{x=65}^{\tau} |q_x - \hat{q}_x|
\end{equation}
di mana $q_x$ adalah probabilitas kematian observasi dan $\hat{q}_x$ adalah probabilitas kematian estimasi model pada usia $x$.

\paragraph{\textit{Root Mean Squared Error} (RMSE)}
RMSE memberikan bobot lebih besar pada deviasi yang lebih besar:
\begin{equation}
\text{RMSE} = \sqrt{\frac{1}{\tau - 65 + 1} \sum_{x=65}^{\tau} (q_x - \hat{q}_x)^2}
\end{equation}

\paragraph{\textit{Weighted Mean Absolute Error} (WMAE)}
Mengingat jumlah individu berisiko ($l_x$) menurun drastis pada usia lanjut ekstrem, WMAE memberikan bobot proporsional terhadap jumlah observasi:
\begin{equation}
\text{WMAE} = \frac{\sum_{x=65}^{\tau} l_x |q_x - \hat{q}_x|}{\sum_{x=65}^{\tau} l_x}
\end{equation}.

\paragraph{\textit{Weighted Root Mean Squared Error} (WRMSE)}
Analog dengan WMAE, WRMSE merupakan versi tertimbang dari RMSE:
\begin{equation}
\text{WRMSE} = \sqrt{\frac{\sum_{x=65}^{\tau} l_x (q_x - \hat{q}_x)^2}{\sum_{x=65}^{\tau} l_x}}
\end{equation}

Penggunaan metrik tertimbang (WMAE dan WRMSE) penting karena metrik tidak tertimbang dapat memberikan bobot berlebihan pada usia-usia dengan jumlah observasi sangat sedikit (dan ketidakpastian estimasi yang tinggi), yang dapat menghasilkan evaluasi yang kurang akurat.

\subsection{Hasil Perbandingan}

Tabel \ref{tab:model_comparison_static} menyajikan hasil perbandingan kinerja model STLT dengan model-model statis lainnya untuk kohor-kohor terpilih dari Belanda dan Jepang. Hasil lengkap untuk seluruh kohor tersedia dalam Lampiran B.

\begin{table}[htbp]
\centering
\caption{Perbandingan Kinerja Model Statis untuk Kohor Terpilih}
\label{tab:model_comparison_static}
\small
\begin{tabular}{clcccc}
\hline
\textbf{Kohor} & \textbf{Model} & \textbf{MAE} & \textbf{RMSE} & \textbf{WMAE} & \textbf{WRMSE} \\
\hline
\multicolumn{6}{c}{\textit{Belanda Wanita 1901}} \\
\hline
& STLT & 0.0178 & 0.0547 & 0.00195 & 0.00331 \\
& TLT & 0.0241 & 0.0513 & 0.00749 & 0.00946 \\
& Gompertz & 0.0364 & 0.0572 & 0.00535 & 0.01342 \\
& Makeham & 0.0365 & 0.0573 & 0.00536 & 0.01345 \\
& Heligman-Pollard & 0.0146 & 0.0383 & 0.00200 & 0.00376 \\
& Coale-Kisker & $\approx 0$ & $\approx 0$ & $\approx 0$ & $\approx 0$ \\
\hline
\multicolumn{6}{c}{\textit{Jepang Wanita 1962}} \\
\hline
& STLT & 0.0471 & 0.1144 & 0.00221 & 0.00332 \\
& TLT & 0.0811 & 0.1314 & 0.01409 & 0.02206 \\
& Gompertz & 0.0422 & 0.0541 & 0.01318 & 0.02031 \\
& Makeham & 0.0422 & 0.0541 & 0.01318 & 0.02032 \\
& Heligman-Pollard & 0.0229 & 0.0387 & 0.00495 & 0.00669 \\
& Coale-Kisker & $\approx 0$ & $\approx 0$ & $\approx 0$ & $\approx 0$ \\
\hline
\end{tabular}
\begin{tablenotes}
\small
\item \textit{Catatan}: Nilai mendekati nol untuk model Coale-Kisker mencerminkan sifat interpolatif metode ini yang secara konstruksi melewati semua titik data observasi pada rentang ekstrapolasi. Nilai-nilai disajikan dalam notasi desimal dengan presisi empat digit signifikan.
\end{tablenotes}
\end{table}

% --------------------------------------------
\subsection{Analisis dan Diskusi}

Hasil perbandingan memberikan beberapa \textit{insight} penting mengenai kinerja relatif model STLT:

Model STLT secara konsisten menunjukkan kinerja terbaik atau mendekati terbaik pada metrik tertimbang (WMAE dan WRMSE) untuk kedua kohor yang dianalisis. Untuk Belanda 1901, STLT mencapai WMAE = 0.00195 dan WRMSE = 0.00331, sedikit di bawah Heligman-Pollard (WMAE = 0.00200, WRMSE = 0.00376) dan jauh lebih baik dibandingkan model Gompertz-Makeham dan TLT. Pola serupa terlihat untuk Jepang 1962, di mana STLT menghasilkan WMAE = 0.00221 dan WRMSE = 0.00332, unggul dibandingkan semua model lain.

Superioritas pada metrik tertimbang mengindikasikan bahwa STLT memberikan kesesuaian yang sangat baik pada rentang usia dengan jumlah observasi besar (usia 65--95 tahun).

Gambar \ref{fig:model_comparison_visualization} menyajikan visualisasi perbandingan seluruh model untuk kohor Belanda 1901, menampilkan probabilitas kematian estimasi setiap model terhadap data observasi.

\begin{figure}[htbp]
\centering
% Placeholder untuk grafik - ganti dengan \includegraphics setelah gambar tersedia
\fbox{\parbox{0.8\textwidth}{\centering
\vspace{4cm}
[Grafik Perbandingan Tingkat Mortalitas $q_x$ untuk Semua Model]\\
Sumbu-X: Usia (65--115 tahun)\\
Sumbu-Y: Probabilitas kematian $q_x$\\
Titik: Data observasi (ukuran proporsional terhadap $\log l_x$)\\
Garis: Model STLT, TLT, Gompertz, Makeham, Heligman-Pollard, Coale-Kisker\\
\vspace{4cm}
}}
\caption{Perbandingan Tingkat Mortalitas Estimasi untuk Berbagai Model Statis (Kohor Belanda 1901)}
\label{fig:model_comparison_visualization}
\begin{fignotes}
\small
\item \textit{Catatan}: Ukuran titik data proporsional terhadap logaritma jumlah individu berisiko ($\log l_x$), menekankan pentingnya kesesuaian pada rentang usia dengan data yang memadai. Model STLT ditunjukkan dengan garis merah tebal untuk kemudahan identifikasi.
\end{fignotes}
\end{figure}


\section{Evolusi Temporal Parameter STLT Antar Kohor}

Salah satu aspek krusial dalam pemodelan mortalitas untuk aplikasi aktuaria adalah kemampuan untuk memahami dan memproyeksikan tren mortalitas antar generasi. Untuk tujuan ini, analisis evolusi temporal parameter model STLT antar kohor menjadi sangat penting, karena pola-pola yang teridentifikasi dapat memberikan justifikasi empiris untuk pengembangan model dinamis. Bagian ini menyajikan analisis komprehensif mengenai bagaimana parameter-parameter model STLT berubah seiring dengan kohor kelahiran, serta implikasinya terhadap pemahaman tren perbaikan mortalitas dan motivasi untuk dinamisasi model.

% --------------------------------------------
\subsection{Motivasi Analisis Temporal}

Dalam konteks manajemen risiko aktuaria, kemampuan untuk memproyeksikan tren mortalitas di masa depan merupakan kebutuhan krusial, terutama untuk produk-produk dengan kewajiban jangka panjang seperti anuitas seumur hidup dan program pensiun. Model mortalitas statis, meskipun dapat memberikan kesesuaian yang baik terhadap data historis untuk kohor tertentu, tidak memiliki mekanisme untuk menangkap perubahan sistematis dalam pola mortalitas antar kohor.

Analisis ini mengikuti metodologi yang diusulkan oleh \citet{Huang2020}, di mana parameter model STLT diestimasi secara independen untuk setiap kohor, kemudian pola temporal parameter-parameter tersebut dievaluasi untuk mengidentifikasi kandidat yang sesuai untuk dinamisasi.

\subsection{Tren Parameter Skala Gompertz ($B$)}

Tabel \ref{tab:parameter_trends_full} menyajikan evolusi parameter $\hat{B}$ untuk seluruh kohor yang dianalisis dari kedua negara.

\begin{table}[htbp]
\centering
\caption{Evolusi Parameter Skala Gompertz ($B$) Antar Kohor}
\label{tab:parameter_trends_full}
\small
\begin{tabular}{ccc|ccc}
\hline
\multicolumn{3}{c|}{\textbf{Belanda}} & \multicolumn{3}{c}{\textbf{Jepang}} \\
\textbf{Kohor} & $\boldsymbol{\hat{B}}$ & $\boldsymbol{\ln(\hat{B})}$ & \textbf{Kohor} & $\boldsymbol{\hat{B}}$ & $\boldsymbol{\ln(\hat{B})}$ \\
\hline
1893 & $3.09 \times 10^{-5}$ & $-10.38$ & 1893 & $2.98 \times 10^{-5}$ & $-10.42$ \\
1894 & $2.82 \times 10^{-5}$ & $-10.47$ & 1894 & $3.46 \times 10^{-5}$ & $-10.27$ \\
1895 & $2.76 \times 10^{-5}$ & $-10.50$ & 1895 & $2.92 \times 10^{-5}$ & $-10.44$ \\
1896 & $2.69 \times 10^{-5}$ & $-10.52$ & 1896 & $3.30 \times 10^{-5}$ & $-10.32$ \\
1897 & $2.38 \times 10^{-5}$ & $-10.65$ & 1897 & $3.30 \times 10^{-5}$ & $-10.32$ \\
1898 & $2.10 \times 10^{-5}$ & $-10.77$ & 1898 & $2.51 \times 10^{-5}$ & $-10.59$ \\
1899 & $1.99 \times 10^{-5}$ & $-10.82$ & 1899 & $2.52 \times 10^{-5}$ & $-10.59$ \\
1900 & $1.69 \times 10^{-5}$ & $-10.99$ & 1900 & $2.12 \times 10^{-5}$ & $-10.76$ \\
1901 & $1.54 \times 10^{-5}$ & $-11.08$ & 1901 & $1.99 \times 10^{-5}$ & $-10.82$ \\
1902 & $1.44 \times 10^{-5}$ & $-11.15$ & 1902 & $1.87 \times 10^{-5}$ & $-10.89$ \\
1903 & $1.22 \times 10^{-5}$ & $-11.31$ & 1903 & $1.51 \times 10^{-5}$ & $-11.10$ \\
1904 & $1.09 \times 10^{-5}$ & $-11.43$ & 1904 & $1.23 \times 10^{-5}$ & $-11.30$ \\
1905 & $9.68 \times 10^{-6}$ & $-11.54$ & 1905 & $9.85 \times 10^{-6}$ & $-11.53$ \\
1906 & $8.82 \times 10^{-6}$ & $-11.64$ & 1906 & $1.09 \times 10^{-5}$ & $-11.42$ \\
1907 & $7.95 \times 10^{-6}$ & $-11.74$ & 1907 & $8.95 \times 10^{-6}$ & $-11.62$ \\
1908 & $7.34 \times 10^{-6}$ & $-11.82$ & 1908 & $7.45 \times 10^{-6}$ & $-11.81$ \\
\hline
\end{tabular}
\begin{tablenotes}
\small
\item \textit{Catatan}: Nilai $\ln(\hat{B})$ disajikan untuk memfasilitasi analisis tren linear, mengingat model DSTLT akan memodelkan $B_i = \exp(a + bi)$.
\end{tablenotes}
\end{table}

Parameter $\hat{B}$ menunjukkan tren menurun yang sangat jelas dan konsisten untuk kedua negara. Untuk Belanda, $\hat{B}$ menurun dari $3.09 \times 10^{-5}$ untuk kohor 1893 menjadi $7.34 \times 10^{-6}$ untuk kohor 1908, merepresentasikan penurunan sebesar 76\%. Pola serupa terlihat untuk Jepang, dengan penurunan dari $2.98 \times 10^{-5}$ menjadi $7.45 \times 10^{-6}$ (penurunan 75\%).Ketika diekspresikan dalam skala logaritmik, $\ln(\hat{B})$ menunjukkan pola yang mendekati linear terhadap indeks kohor. Untuk Belanda, $\ln(\hat{B})$ menurun dari $-10.38$ menjadi $-11.82$, sementara untuk Jepang dari $-10.42$ menjadi $-11.81$. Linieritas ini memberikan justifikasi empiris yang kuat untuk spesifikasi model DSTLT yang memodelkan $B_i = \exp(a + bi)$, di mana $b < 0$ mengindikasikan perbaikan mortalitas antar kohor.

\subsection{Tren Parameter Pertumbuhan Gompertz ($C$)}

Berbeda dengan parameter $B$, parameter $\hat{C}$ menunjukkan pola temporal yang berbeda, sebagaimana ditunjukkan dalam Tabel \ref{tab:parameter_C_trends}.

\begin{table}[htbp]
\centering
\caption{Evolusi Parameter Pertumbuhan Gompertz ($C$) Antar Kohor}
\label{tab:parameter_C_trends}
\small
\begin{tabular}{cc|cc}
\hline
\multicolumn{2}{c|}{\textbf{Belanda}} & \multicolumn{2}{c}{\textbf{Jepang}} \\
\textbf{Kohor} & $\boldsymbol{\hat{C}}$ & \textbf{Kohor} & $\boldsymbol{\hat{C}}$ \\
\hline
1893 & 1.1011 & 1893 & 1.1024 \\
1894 & 1.1022 & 1894 & 1.1000 \\
1895 & 1.1023 & 1895 & 1.1021 \\
1896 & 1.1025 & 1896 & 1.1002 \\
1897 & 1.1039 & 1897 & 1.1000 \\
1898 & 1.1054 & 1898 & 1.1033 \\
1899 & 1.1060 & 1899 & 1.1030 \\
1900 & 1.1080 & 1900 & 1.1049 \\
1901 & 1.1090 & 1901 & 1.1055 \\
1902 & 1.1099 & 1902 & 1.1061 \\
1903 & 1.1119 & 1903 & 1.1088 \\
1904 & 1.1134 & 1904 & 1.1111 \\
1905 & 1.1148 & 1905 & 1.1140 \\
1906 & 1.1159 & 1906 & 1.1123 \\
1907 & 1.1172 & 1907 & 1.1147 \\
1908 & 1.1182 & 1908 & 1.1169 \\
\hline
\end{tabular}
\end{table}

\paragraph{Pola Tren}
Parameter $\hat{C}$ menunjukkan tren meningkat yang lemah untuk kedua negara. Untuk Belanda, $\hat{C}$ meningkat dari 1.1011 menjadi 1.1182 (peningkatan 1.55\%), sementara untuk Jepang dari 1.1024 menjadi 1.1169 (peningkatan 1.32\%). Meskipun tren ini konsisten secara directional, magnitudenya jauh lebih kecil dibandingkan perubahan parameter $B$.

Lebih lanjut, tren parameter $C$ menunjukkan lebih banyak fluktuasi dibandingkan parameter $B$. Misalnya, untuk Jepang, terdapat penurunan dari kohor 1893 ke 1894, sebelum tren meningkat berlanjut. Hal ini mengindikasikan bahwa evolusi parameter $C$ kurang sistematis dan lebih dipengaruhi oleh variabilitas sampling atau faktor-faktor spesifik kohor.

\paragraph{Interpretasi dan Implikasi}
Peningkatan lemah parameter $C$ memiliki implikasi substantif yang menarik. Parameter $C$ menentukan laju pertumbuhan eksponensial mortalitas seiring bertambahnya usia. Peningkatan $C$ antar kohor mengindikasikan bahwa meskipun tingkat mortalitas \textit{baseline} menurun (parameter $B$ turun), laju peningkatan mortalitas dengan usia sedikit meningkat untuk kohor-kohor yang lebih muda.

Fenomena ini konsisten dengan hipotesis bahwa perbaikan mortalitas tidak seragam di seluruh rentang usia. Intervensi medis dan peningkatan standar hidup mungkin lebih efektif dalam menurunkan mortalitas pada usia yang lebih muda (menghasilkan penurunan $B$), sementara proses penuaan biologis fundamental yang mendasari peningkatan mortalitas dengan usia (yang dicerminkan oleh $C$) tetap relatif stabil atau bahkan sedikit meningkat.

Namun, mengingat magnitude perubahan yang kecil dan variabilitas yang lebih tinggi, parameter $C$ bukan kandidat utama untuk dinamisasi dalam model DSTLT. Sebagai gantinya, model DSTLT akan memodelkan $C_i$ secara implisit melalui hubungannya dengan $B_i$ melalui \textit{smoothing constraint} $\theta = 1/(B_i C_i^N)$.

% --------------------------------------------
\subsection{Tren Parameter GPD ($\gamma$, $N$, $\omega$)}

\paragraph{Parameter Bentuk GPD ($\gamma$)}
Analisis evolusi parameter $\hat{\gamma}$ mengungkapkan pola yang kompleks dan tidak sistematis. Untuk Belanda, $\hat{\gamma}$ bervariasi dari $-0.074$ (kohor 1893) hingga $+0.052$ (kohor 1897) dan $-0.448$ (kohor 1902), tanpa tren temporal yang jelas. Untuk Jepang, variabilitas serupa diamati, dengan $\hat{\gamma}$ berkisar dari $-0.242$ hingga $+0.230$.

Variabilitas tinggi dan ketiadaan pola temporal sistematis untuk parameter $\gamma$ dapat dikaitkan dengan beberapa faktor:
\begin{itemize}
    \item \textbf{Keterbatasan Data Usia Ekstrem}: Parameter $\gamma$ terutama ditentukan oleh data pada usia lanjut ekstrem (di atas 100 tahun), di mana jumlah observasi sangat terbatas dan ketidakpastian estimasi tinggi.
    
    \item \textbf{Karakteristik Struktural Ekor Distribusi}: Parameter $\gamma$ merepresentasikan karakteristik fundamental ekor distribusi usia kematian, yang mungkin lebih stabil secara struktural dibandingkan parameter yang menentukan tingkat mortalitas \textit{baseline}.
    
    \item \textbf{Kompleksitas Biologis}: Proses yang menentukan batas atas usia manusia dan karakteristik mortalitas pada usia sangat lanjut mungkin berbeda secara fundamental dari proses yang menentukan mortalitas pada usia yang lebih muda, dan mungkin kurang dipengaruhi oleh faktor lingkungan dan medis yang mendorong perbaikan mortalitas antar kohor.
\end{itemize}

Ketiadaan pola temporal yang jelas menjadikan parameter $\gamma$ tidak sesuai untuk dinamisasi, dan model DSTLT akan memperlakukannya sebagai konstan antar kohor.

\paragraph{Usia Ambang ($N$)}
Usia ambang optimal $\hat{N}$ juga tidak menunjukkan tren temporal yang sistematis. Untuk Belanda, $\hat{N}$ bervariasi antara 95 hingga 108 tahun tanpa pola monotonik yang jelas. Untuk Jepang, variabilitas serupa diamati.

Variabilitas $\hat{N}$ mencerminkan trade-off optimisasi antara kompleksitas model dan kesesuaian data untuk setiap kohor spesifik, dan dipengaruhi oleh karakteristik distribusi usia kematian yang spesifik kohor. Mengingat ketiadaan pola temporal dan pertimbangan parsimoni model, DSTLT akan menggunakan usia ambang konstan yang optimal secara \textit{pooled} untuk semua kohor.

\paragraph{\textit{Highest Attained Age} ($\omega$)}
Estimasi $\hat{\omega}$ (untuk kasus $\hat{\gamma} < 0$) juga menunjukkan variabilitas substansial tanpa tren yang jelas. Untuk Belanda, $\hat{\omega}$ berkisar dari 76.6 hingga 130.6 tahun, sementara untuk Jepang dari 97.6 hingga 135.6 tahun.

Variabilitas tinggi ini mencerminkan sensitivitas $\omega$ terhadap estimasi parameter, khususnya $\gamma$ dan $\theta$. Mengingat ketidakpastian estimasi yang besar pada parameter-parameter ini, estimasi $\hat{\omega}$ harus diinterpretasikan dengan hati-hati dan lebih sebagai indikator kualitatif tentang potensi batas atas usia dibandingkan prediksi kuantitatif yang presisi.

% --------------------------------------------
\subsection{Visualisasi Tren Parameter}

Gambar \ref{fig:parameter_evolution} menyajikan visualisasi evolusi seluruh parameter model STLT antar kohor untuk kedua negara.

\begin{figure}[htbp]
\centering
% Placeholder untuk grafik - ganti dengan \includegraphics setelah gambar tersedia
\fbox{\parbox{0.8\textwidth}{\centering
\vspace{5cm}
[Grafik Panel 4x2 menunjukkan evolusi parameter $B$, $C$, $\gamma$, dan $N$ untuk Belanda dan Jepang]\\
Panel 1: $\ln(B)$ vs Kohor dengan fitted line\\
Panel 2: $C$ vs Kohor\\
Panel 3: $\gamma$ vs Kohor\\
Panel 4: $N$ vs Kohor\\
\vspace{5cm}
}}
\caption{Evolusi Temporal Parameter Model STLT Antar Kohor}
\label{fig:parameter_evolution}
\begin{fignotes}
\small
\item \textit{Catatan}: Panel kiri menampilkan data Belanda, panel kanan menampilkan data Jepang. Untuk parameter $B$, nilai ditampilkan dalam skala logaritmik dengan garis regresi linear untuk menyoroti tren. Garis horizontal putus-putus pada panel $\gamma$ menunjukkan $\gamma = 0$ sebagai referensi.
\end{fignotes}
\end{figure}

Visualisasi ini mengkonfirmasi observasi kuantitatif:
\begin{itemize}
    \item Tren menurun yang jelas dan hampir linear untuk $\ln(B)$, dengan variabilitas yang relatif kecil di sekitar garis tren
    \item Tren meningkat yang lebih lemah untuk $C$, dengan fluktuasi yang lebih besar
    \item Tidak ada pola temporal yang jelas untuk $\gamma$ dan $N$, dengan variabilitas substansial antar kohor
\end{itemize}


\section{Estimasi dan Evaluasi Model DSTLT}

Model DSTLT memperluas STLT dengan memodelkan parameter $B$ sebagai fungsi eksponensial dari indeks kohor: $B_i = \exp(a + bi)$. Bagian ini menyajikan hasil estimasi parameter DSTLT dan evaluasi kinerjanya melalui analisis \textit{in-sample fit} dan \textit{out-of-sample forecasting}.

% --------------------------------------------
\subsection{Estimasi Parameter DSTLT}

Estimasi parameter dilakukan dengan memaksimalkan fungsi \textit{likelihood} gabungan yang mencakup data dari kohor \textit{training} (1893--1901). Parameter yang diestimasi adalah $a$, $b$ (menentukan dinamika $B_i$), $\theta$, $\gamma$, dan $N$ (konstan antar kohor). Tabel \ref{tab:dstlt_parameters} menyajikan hasil estimasi untuk kedua negara.

\begin{table}[htbp]
\centering
\caption{Estimasi Parameter Model DSTLT}
\label{tab:dstlt_parameters}
\begin{tabular}{lcc}
\hline
\textbf{Parameter} & \textbf{Belanda} & \textbf{Jepang} \\
\hline
$\hat{a}$ & $-10.54$ & $-9.920$ \\
$\hat{b}$ & $-0.158$ & $-0.085$ \\
$\hat{\theta}$ & 1.677 & 3.455 \\
$\hat{\gamma}$ & $-0.168$ & $-0.162$ \\
$\hat{N}$ & 100 & 90 \\
$\hat{\omega}$ & 110.0 & 111.4 \\
\hline
\end{tabular}
\end{table}

\paragraph{Interpretasi Parameter}
Parameter $\hat{b} < 0$ untuk kedua negara mengkonfirmasi tren penurunan mortalitas \textit{baseline} antar kohor, dengan magnitude yang lebih besar untuk Belanda ($-0.158$) dibandingkan Jepang ($-0.085$). Parameter $\hat{\gamma} < 0$ konsisten dengan temuan STLT, mengindikasikan distribusi usia dengan batas atas finit. Estimasi $\hat{\omega}$ sekitar 110--111 tahun konsisten untuk kedua negara.

% --------------------------------------------
\subsection{Evaluasi \textit{In-Sample Fit}}

Untuk mengevaluasi kesesuaian model terhadap data \textit{training}, tingkat mortalitas estimasi DSTLT dibandingkan dengan data observasi kohor 1893--1901. Gambar \ref{fig:dstlt_insample} menunjukkan kesesuaian visual yang sangat baik untuk kedua negara, dengan model DSTLT berhasil menangkap pola mortalitas di seluruh rentang usia.

\begin{figure}[htbp]
\centering
\fbox{\parbox{0.75\textwidth}{\centering\vspace{2.5cm}
[Grafik DSTLT vs Data Observasi untuk Kohor Training]\\
\vspace{2.5cm}}}
\caption{Kesesuaian \textit{In-Sample} Model DSTLT}
\label{fig:dstlt_insample}
\end{figure}

% --------------------------------------------
\subsection{Evaluasi \textit{Out-of-Sample Forecasting}}

Kemampuan peramalan model DSTLT dievaluasi menggunakan kohor \textit{test} (1902--1908) yang tidak digunakan dalam estimasi parameter. Model Cairns-Blake-Dowd (CBD) digunakan sebagai \textit{benchmark}, mengingat popularitasnya dalam industri untuk pemodelan mortalitas usia lanjut \citep{Cairns2006}.

\paragraph{Hasil Perbandingan}
Tabel \ref{tab:forecasting_performance} menyajikan metrik evaluasi untuk kedua model pada kohor \textit{test}.

\begin{table}[htbp]
\centering
\caption{Perbandingan Kinerja Peramalan DSTLT vs CBD}
\label{tab:forecasting_performance}
\small
\begin{tabular}{ccccc}
\hline
\multirow{2}{*}{\textbf{Kohor}} & \multicolumn{2}{c}{\textbf{Belanda}} & \multicolumn{2}{c}{\textbf{Jepang}} \\
\cline{2-5}
& \textbf{DSTLT} & \textbf{CBD} & \textbf{DSTLT} & \textbf{CBD} \\
& MAE & MAE & MAE & MAE \\
\hline
1902 & 0.0815 & 0.0609 & 0.0404 & 0.0315 \\
1903 & 0.0815 & 0.0609 & 0.0414 & 0.0268 \\
1904 & 0.0809 & 0.0385 & 0.0415 & 0.0271 \\
1905 & 0.0718 & 0.0407 & 0.0391 & 0.0309 \\
1906 & -- & -- & 0.0410 & 0.0267 \\
1907 & -- & -- & 0.0405 & 0.0265 \\
1908 & 0.0710 & 0.0329 & 0.0401 & 0.0271 \\
\hline
\textbf{Rata-rata} & \textbf{0.0781} & \textbf{0.0468} & \textbf{0.0406} & \textbf{0.0281} \\
\hline
\end{tabular}
\begin{tablenotes}
\small
\item \textit{Catatan}: Nilai MAE dalam unit probabilitas. Tanda -- mengindikasikan data tidak tersedia untuk analisis.
\end{tablenotes}
\end{table}

% --------------------------------------------
\subsection{Analisis dan Diskusi}

\paragraph{Kinerja Relatif}
Model CBD menunjukkan kinerja peramalan yang lebih baik dibandingkan DSTLT pada sebagian besar kohor \textit{test}, dengan rata-rata MAE yang lebih rendah untuk kedua negara (Belanda: 0.0468 vs 0.0781; Jepang: 0.0281 vs 0.0406). Hal ini dapat dikaitkan dengan beberapa faktor:

\begin{enumerate}
    \item \textbf{Fleksibilitas Model}: CBD menggunakan pendekatan \textit{age-period-cohort} yang lebih fleksibel dengan memungkinkan parameter bervariasi untuk setiap periode dan kohor, sementara DSTLT membatasi dinamisasi pada satu parameter dengan bentuk fungsional eksponensial.
    
    \item \textbf{Fokus Usia}: CBD didesain khusus untuk usia lanjut (65+), sementara DSTLT merupakan model tabel mortalitas lengkap yang harus menangkap pola di seluruh rentang usia.
    
    \item \textbf{Kompleksitas Parameter}: CBD memiliki lebih banyak derajat kebebasan yang memungkinkan adaptasi lebih baik terhadap fluktuasi jangka pendek dalam data mortalitas.
\end{enumerate}

\paragraph{Keunggulan DSTLT}
Meskipun CBD unggul dalam metrik peramalan kuantitatif, DSTLT memiliki beberapa keunggulan konseptual:

\begin{itemize}
    \item \textbf{Justifikasi Teoretis}: Fondasi EVT memberikan justifikasi statistik yang kuat untuk pemodelan ekor distribusi usia, terutama untuk usia lanjut ekstrem di atas 100 tahun.
    
    \item \textbf{Estimasi Batas Atas Usia}: DSTLT memberikan estimasi eksplisit $\omega$ berdasarkan parameter yang diestimasi, sementara CBD tidak memiliki mekanisme untuk menentukan batas usia.
    
    \item \textbf{Kesederhanaan dan Interpretabilitas}: Dengan hanya 5 parameter global, DSTLT lebih parsimoni dan parameter memiliki interpretasi demografis yang jelas.
    
    \item \textbf{Stabilitas Ekstrapolasi}: Struktur parametrik DSTLT mencegah proyeksi mortalitas yang tidak realistis pada horizon jangka panjang, sementara CBD dapat menghasilkan proyeksi yang tidak stabil untuk periode jauh di masa depan.
\end{itemize}

\paragraph{Visualisasi Peramalan}
Gambar \ref{fig:forecasting_comparison} menampilkan perbandingan visual peramalan kedua model untuk kohor \textit{test} terpilih, menunjukkan bahwa meskipun CBD memberikan kesesuaian kuantitatif lebih baik, DSTLT menghasilkan proyeksi yang masuk akal secara kualitatif dengan transisi mulus pada usia ambang.

\begin{figure}[htbp]
\centering
\fbox{\parbox{0.75\textwidth}{\centering\vspace{3cm}
[Grafik Panel: Prediksi DSTLT vs CBD vs Data Observasi untuk Kohor Test]\\
\vspace{3cm}}}
\caption{Perbandingan Peramalan DSTLT vs CBD untuk Kohor \textit{Test}}
\label{fig:forecasting_comparison}
\end{figure}


\chapter{KESIMPULAN DAN SARAN}

Bab ini menyajikan rangkuman dari seluruh penelitian yang telah dilakukan, meliputi kesimpulan utama dari analisis empiris model \textit{Smooth Threshold Life Table} (STLT) dan \textit{Dynamic Smooth Threshold Life Table} (DSTLT), serta saran untuk penelitian lanjutan. Bab ini juga membahas keterbatasan penelitian yang perlu diperhatikan dalam interpretasi hasil.

Struktur bab ini terdiri dari tiga bagian utama. Sub-bab 5.1 menyajikan kesimpulan utama yang menjawab tujuan penelitian, Sub-bab 5.2 memberikan saran untuk pengembangan penelitian di masa depan, dan Sub-bab 5.3 mendiskusikan keterbatasan penelitian yang telah dilakukan.

\section{Kesimpulan}

Penelitian ini telah berhasil menganalisis dan membandingkan kinerja model STLT dan DSTLT dalam memodelkan mortalitas usia lanjut ekstrem menggunakan data populasi Belanda dan Jepang. Berdasarkan analisis empiris yang telah dilakukan pada Bab 4, dapat ditarik beberapa kesimpulan utama sebagai berikut:

\textbf{Pertama}, model STLT menunjukkan keunggulan signifikan dibandingkan model TLT dalam hal kontinuitas fungsi \textit{hazard}. Melalui penerapan kendala penghalusan $h_1(N) = h_2(N)$, model STLT berhasil menghilangkan diskontinuitas yang menjadi kelemahan fundamental model TLT. Secara empiris, hal ini tercermin dari kurva mortalitas yang lebih halus dan realistis pada titik ambang batas $N$, sebagaimana terlihat pada visualisasi hasil untuk berbagai kohort di kedua negara.

\textbf{Kedua}, dari perspektif \textit{goodness-of-fit}, model STLT secara konsisten menghasilkan nilai \textit{Sum of Squared Errors} (SSE) yang lebih rendah dibandingkan model TLT dan model-model mortalitas statis lainnya (Gompertz, Heligman-Pollard, Coale-Kisker, Makeham) pada rentang usia lanjut. Keunggulan ini menunjukkan bahwa kombinasi distribusi Gompertz untuk usia menengah dan GPD untuk usia ekstrem dengan kendala penghalusan memberikan fleksibilitas yang memadai untuk menangkap pola deselerasi mortalitas pada usia sangat lanjut.

\textbf{Ketiga}, estimasi parameter $\gamma$ yang secara konsisten bernilai negatif pada sebagian besar kohort memberikan bukti empiris yang mendukung hipotesis adanya batas atas usia manusia (\textit{finite human lifespan}). Nilai $\gamma < 0$ mengimplikasikan bahwa fungsi \textit{hazard} mencapai asimtot horizontal pada usia tertinggi yang dapat dicapai $\omega = N + \theta/|\gamma|$, yang secara teoretis konsisten dengan konsep deselerasi mortalitas pada usia ekstrem.

\textbf{Keempat}, dalam konteks pemodelan dinamis, model DSTLT menunjukkan performa yang kompetitif dibandingkan model Cairns-Blake-Dowd (CBD) dalam memprediksi mortalitas kohort masa depan. Analisis \textit{out-of-sample} pada kohort kelahiran 1902--1908 menunjukkan bahwa DSTLT mampu menangkap tren temporal perubahan mortalitas dengan baik, khususnya pada rentang usia sangat lanjut (di atas 100 tahun) di mana model CBD cenderung kurang akurat.

\textbf{Kelima}, pemilihan parameter $B$ sebagai parameter yang didinamisasi dalam model DSTLT terbukti efektif secara empiris. Pola temporal parameter $B$ yang menunjukkan tren menurun sejalan dengan perbaikan mortalitas lintas kohort, sementara kendala penghalusan tetap terpenuhi melalui penyesuaian otomatis parameter $C$ dan $\theta$.

Secara keseluruhan, penelitian ini berhasil mendemonstrasikan bahwa pendekatan \textit{Extreme Value Theory} melalui model STLT dan ekstensi dinamisnya (DSTLT) menyediakan kerangka kerja yang solid dan fleksibel untuk memodelkan dan memproyeksikan mortalitas pada usia lanjut ekstrem, dengan implikasi penting bagi praktik aktuaria dalam penentuan premi asuransi jiwa, anuitas, dan manajemen risiko longevitas.

\section{Saran Penelitian Lanjutan}

Berdasarkan hasil penelitian dan keterbatasan yang telah diidentifikasi, terdapat beberapa arah pengembangan yang dapat dilakukan untuk penelitian lanjutan:

\textbf{Pertama}, perluasan cakupan data ke negara-negara berkembang, khususnya kawasan Asia Tenggara termasuk Indonesia. Data mortalitas usia lanjut dari populasi dengan karakteristik sosio-ekonomi dan sistem kesehatan yang berbeda akan memberikan validasi yang lebih komprehensif terhadap model STLT dan DSTLT. Hal ini juga akan memberikan wawasan tentang robustness model terhadap heterogenitas populasi.

\textbf{Kedua}, eksplorasi metodologi pemilihan titik ambang batas $N$ yang lebih sistematis. Penelitian ini menggunakan nilai $N$ berdasarkan pedoman dari literatur, namun pendekatan berbasis data seperti \textit{profile likelihood}, \textit{cross-validation}, atau kriteria informasi (AIC/BIC) dapat dikembangkan untuk mengoptimalkan pemilihan threshold secara objektif. Analisis sensitivitas yang lebih mendalam terhadap variasi $N$ juga diperlukan untuk mengukur stabilitas estimasi parameter.

\textbf{Ketiga}, pengembangan model DSTLT multi-parameter dengan mendinamisasi tidak hanya parameter $B$, tetapi juga parameter $\gamma$ atau kombinasi parameter lainnya. Eksplorasi ini dapat memberikan fleksibilitas tambahan dalam menangkap perubahan pola mortalitas lintas kohort yang lebih kompleks, meskipun perlu mempertimbangkan trade-off antara kompleksitas model dan risiko \textit{overfitting}.

\textbf{Keempat}, investigasi lebih lanjut mengenai struktur dependensi antar parameter dalam model STLT. Analisis matriks informasi Fisher dan korelasi parameter dapat memberikan pemahaman yang lebih baik tentang identifiabilitas model dan efisiensi estimasi. Pendekatan Bayesian dengan spesifikasi \textit{prior} yang informatif juga dapat menjadi alternatif untuk mengatasi potensi masalah identifikasi.

\textbf{Kelima}, penerapan model pada konteks aktuaria yang lebih spesifik, seperti perhitungan cadangan anuitas, valuasi produk asuransi jiwa dengan perlindungan hingga usia sangat lanjut, atau pengukuran risiko longevitas dalam skema pensiun. Studi kasus implementasi praktis akan memberikan nilai tambah dalam menunjukkan relevansi model untuk industri asuransi dan dana pensiun.

\textbf{Keenam}, pengembangan metode \textit{bootstrapping} atau simulasi Monte Carlo untuk konstruksi interval kepercayaan parameter dan proyeksi mortalitas yang lebih robust. Hal ini penting mengingat uncertainty yang tinggi pada estimasi mortalitas di usia sangat ekstrem akibat sparseness data.

\textbf{Ketujuh}, perbandingan dengan model-model mortalitas stokastik alternatif yang lebih kompleks seperti model Lee-Carter, Age-Period-Cohort (APC), atau model \textit{machine learning} (misalnya neural networks atau random forests) untuk mengevaluasi trade-off antara interpretabilitas dan akurasi prediksi.

\section{Keterbatasan Penelitian}

Meskipun penelitian ini telah memberikan kontribusi dalam pemodelan mortalitas usia lanjut, terdapat beberapa keterbatasan yang perlu diakui dan dipertimbangkan dalam interpretasi hasil:

\textbf{Pertama, kualitas dan ketersediaan data}. Penelitian ini menggunakan data dari Human Mortality Database (HMD) dan Statistics Netherlands (CBS) yang merupakan sumber data berkualitas tinggi. Namun, pada usia sangat lanjut (di atas 105 tahun), jumlah observasi menjadi sangat terbatas, yang dapat mempengaruhi stabilitas estimasi parameter, khususnya parameter $\gamma$ dalam distribusi GPD. Fenomena \textit{age heaping} dan kesalahan pencatatan usia juga berpotensi mempengaruhi validitas data, meskipun telah dilakukan prosedur validasi usia oleh penyedia data.

\textbf{Kedua, asumsi no migration}. Dalam konstruksi kohort hipotetis, penelitian ini mengasumsikan tidak ada efek migrasi dengan menetapkan $l'_{65} = l_{65}$ dan menghitung $l'_x$ berdasarkan $q_x$ empiris. Asumsi ini menyederhanakan realitas kompleks dinamika populasi, terutama untuk negara dengan tingkat migrasi tinggi. Meskipun penyesuaian migrasi telah dilakukan oleh HMD untuk sebagian data, efek residual dari migrasi neto dapat mempengaruhi pola mortalitas yang diamati.

\textbf{Ketiga, keterbatasan rentang data temporal}. Model DSTLT diestimasi menggunakan kohort kelahiran 1893--1901 untuk \textit{training} dan 1902--1908 untuk \textit{testing}. Rentang waktu ini relatif terbatas untuk menangkap tren jangka panjang perubahan mortalitas, terutama tren perbaikan mortalitas yang dipercepat pada dekade-dekade terakhir. Proyeksi jangka panjang di luar rentang data observasi memerlukan kehati-hatian ekstra.

\textbf{Keempat, asumsi distribusional}. Model STLT mengasumsikan distribusi Gompertz untuk usia di bawah threshold dan GPD untuk usia di atasnya. Meskipun asumsi ini didukung oleh teori \textit{Extreme Value Theory}, validitas asumsi ini pada berbagai populasi dan periode waktu perlu dikaji lebih lanjut. Uji kesesuaian distribusi yang lebih komprehensif, termasuk analisis residual dan diagnostic plots, dapat memperkuat justifikasi pemilihan distribusi.

\textbf{Kelima, kompleksitas komputasi}. Estimasi model STLT dan DSTLT memerlukan optimisasi numerik yang intensif, khususnya karena kendala penghalusan menciptakan interdependensi antar parameter. Konvergensi algoritma optimisasi dapat sensitif terhadap pemilihan nilai awal dan dapat terjebak pada \textit{local optimum}. Penelitian ini menggunakan metode optimisasi standar tanpa eksplorasi mendalam terhadap algoritma alternatif atau strategi inisialisasi yang lebih canggih.

\textbf{Keenam, generalisabilitas hasil}. Hasil penelitian ini didasarkan pada data dari Belanda dan Jepang, dua negara maju dengan sistem pencatatan vital yang sangat baik dan karakteristik mortalitas spesifik. Generalisasi model ke populasi lain, terutama negara berkembang dengan pola mortalitas yang berbeda, memerlukan validasi empiris tambahan.

\textbf{Ketujuh, independensi antar kohort}. Model DSTLT mengasumsikan bahwa parameter evolusi $B_i = \exp(a + bi)$ bersifat deterministik dan linier dalam skala logaritma terhadap tahun kelahiran. Asumsi ini mengabaikan kemungkinan efek periode atau efek interaksi kohort-periode yang lebih kompleks, serta potensi shock eksternal (misalnya pandemi atau perang) yang dapat mempengaruhi pola mortalitas secara non-linear.

Dengan menyadari keterbatasan-keterbatasan ini, hasil penelitian tetap memberikan bukti empiris yang kuat tentang keunggulan model STLT dan DSTLT dalam konteks pemodelan mortalitas usia lanjut ekstrem, serta membuka jalan bagi penelitian lanjutan yang lebih komprehensif.

% Daftar Pustaka
%Buat daftar pustaka
\bibliographystyle{apalike-ejor}
\bibliography{pustaka} % Tambahkan pustaka yang digunakan mahasiswa dalam file pustaka.bib. Perhatikan format penulisan pustakanya!

% Lampiran (jika ada)

\begin{appendix}
	\include{markLampiran}
	\addChapter{Lampiran 1}
\chapter*{\normalsize Lampiran 1}

Perhatikan penurunan model berikut.
\end{appendix}
\end{document} 

\chapter{Pendahuluan}

\section{Latar Belakang}

Peningkatan harapan hidup global merupakan pencapaian penting abad ke-21, ditandai dengan bertambahnya individu yang mencapai usia sangat lanjut, termasuk supercentenarian (usia $\geq$ 110 tahun) yang meningkat signifikan sejak 1960 \citep{Young2021}. Fenomena ini membawa implikasi krusial bagi industri asuransi jiwa dan dana pensiun yang menghadapi \textit{longevity risk}—ketidakpastian proyeksi harapan hidup yang lebih tinggi dari perkiraan. Ketika individu hidup lebih lama, institusi yang memberikan jaminan pembayaran jangka panjang (anuitas, pensiun) menghadapi beban finansial lebih besar, menjadikan pemodelan mortalitas usia lanjut yang akurat kebutuhan fundamental untuk manajemen risiko yang efektif.

Model mortalitas klasik—Gompertz-Makeham, Heligman-Pollard, Coale-Kisker—telah lama digunakan dalam praktik aktuaria. Namun, model-model ini memiliki keterbatasan mendasar: (1) tidak dapat menangkap deselerasi mortalitas pada usia lanjut ekstrem, dimana laju peningkatan tingkat kematian melambat atau mendekati konstan \citep{Thatcher1999}, dan (2) menetapkan usia maksimum tabel mortalitas (\textit{highest attained age}, $\omega$) secara subjektif tanpa landasan statistik, menghasilkan estimasi tidak reliabel pada ekor distribusi.

\textit{Extreme Value Theory} (EVT) menawarkan solusi prinsipiil sebagai cabang statistik yang dirancang khusus untuk menganalisis perilaku ekor distribusi dan kejadian ekstrem. EVT memberikan justifikasi teoretis untuk memodelkan mortalitas usia sangat lanjut menggunakan \textit{Generalized Pareto Distribution} (GPD). Model \textit{Threshold Life Table} (TLT) mengadopsi pendekatan ini dengan menggabungkan Gompertz untuk usia non-ekstrem ($x \leq u$) dan GPD untuk usia lanjut ekstrem ($x > u$) pada usia ambang $u$. Meskipun fleksibel, TLT memiliki kelemahan: diskontinuitas fungsi \textit{hazard} di titik threshold yang tidak realistis secara biologis dan mengurangi kualitas proyeksi.

Model \textit{Smoothed Threshold Life Table} (STLT) dikembangkan oleh \citet{huang2020modelling} untuk mengatasi masalah ini dengan menambahkan \textit{smoothing constraint} $\theta = 1/(BC^u)$ yang memastikan kontinuitas fungsi hazard: $h(u^-) = BC^u = h(u^+) = 1/\theta$. Constraint ini mengeliminasi loncatan mendadak pada usia ambang, menghasilkan transisi mulus antara kedua komponen distribusi. Untuk mengakomodasi perubahan pola mortalitas antarkohor dan peramalan masa depan, STLT diperluas menjadi \textit{Dynamic Smooth Threshold Life Table} (DSTLT) yang memodelkan parameter tertentu sebagai fungsi indeks kohor, menangkap tren mortalitas jangka panjang dan memungkinkan proyeksi untuk kohor mendatang.

Penelitian ini mengkaji secara mendalam konstruksi, estimasi parameter, dan evaluasi kinerja model STLT dan DSTLT. Kajian meliputi derivasi matematis constraint smoothing, prosedur estimasi MLE dengan profile likelihood untuk threshold selection, perbandingan in-sample fit terhadap model statis (Gompertz-Makeham, Heligman-Pollard, Coale-Kisker), dan evaluasi out-of-sample forecasting DSTLT dibandingkan model dinamis Cairns-Blake-Dowd (CBD). Kontribusi penelitian ini adalah pemahaman komprehensif penerapan EVT dalam aktuaria dan penyediaan alternatif model yang lebih robust untuk manajemen risiko longevitas dalam industri asuransi dan dana pensiun.

\section{Rumusan Masalah}

Berdasarkan latar belakang, rumusan masalah penelitian adalah:

\begin{enumerate}
    \item Bagaimana konstruksi matematis model STLT dan DSTLT, khususnya derivasi \textit{smoothing constraint} $\theta = 1/(BC^u)$ untuk kontinuitas fungsi hazard dan formulasi parameter dinamis untuk proyeksi antarkohor?

    \item Bagaimana prosedur estimasi parameter menggunakan MLE dengan data tersensor interval dan kanan, serta pemilihan usia ambang optimal $u$ melalui \textit{profile likelihood}?

    \item Bagaimana kinerja in-sample STLT dibandingkan model statis (Gompertz-Makeham, Heligman-Pollard, Coale-Kisker) berdasarkan kriteria AIC, BIC, dan SSE?

    \item Bagaimana kinerja in-sample dan out-of-sample DSTLT dibandingkan model dinamis Cairns-Blake-Dowd (CBD) untuk proyeksi mortalitas jangka panjang?
\end{enumerate}

\section{Tujuan Penelitian}

Tujuan penelitian adalah:

\begin{enumerate}
    \item Mengonstruksi model STLT dan DSTLT secara matematis, termasuk derivasi \textit{smoothing constraint} $\theta = 1/(BC^u)$ dan formulasi parameter dinamis $B_i = \exp(a + bi)$ untuk DSTLT.

    \item Mengestimasi parameter STLT dan DSTLT menggunakan MLE dengan data tersensor, dan menentukan usia ambang optimal $u$ melalui \textit{profile likelihood} dari kandidat $u \in \{85, 86, \ldots, 110\}$.

    \item Mengevaluasi kinerja in-sample STLT dibandingkan model statis (Gompertz-Makeham, Heligman-Pollard, Coale-Kisker) menggunakan kriteria AIC, BIC, dan SSE.

    \item Mengevaluasi kinerja in-sample dan out-of-sample DSTLT dibandingkan model dinamis CBD untuk proyeksi mortalitas kohor mendatang.
\end{enumerate}

\section{Batasan Penelitian}

 Penelitian ini mengasumsikan tidak terjadi migrasi pada populasi yang diteliti. Asumsi ini diperlukan untuk menyederhanakan analisis kohor dan memastikan perubahan jumlah individu dalam kohor hanya disebabkan oleh kematian. Untuk memenuhi asumsi ini, dilakukan penyesuaian data untuk menciptakan kohor tertutup yang menghilangkan pengaruh migrasi.
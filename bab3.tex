\chapter{\normalsize MODEL DYNAMIC SMOOTH THRESHOLD LIFE TABLE (DSTLT)}

Pemodelan mortalitas pada usia lanjut ekstrem menghadapi tantangan yang tidak dapat diatasi sepenuhnya oleh model-model mortalitas tradisional. Keterbatasan data pada usia sangat lanjut, volatilitas tinggi dari estimasi empiris, dan kebutuhan untuk menentukan batas akhir tabel mortalitas secara objektif mendorong pengembangan metode yang lebih canggih. Bab ini membahas mengenai perkembangan bertahap dari pemodelan mortalitas usia lanjut, dimulai dari model \textit{Threshold Life Table} (TLT) sebagai model dasar, kemudian pengembangan \textit{Smoothed Threshold Life Table} (STLT) sebagai perbaikan, hingga \textit{Dynamic Smooth Threshold Life Table} (DSTLT) sebagai pengembangan untuk keperluan peramalan.

Model TLT yang diperkenalkan oleh Li et al. (2008) mengatasi masalah terkait penentuan usia penutupan tabel mortalitas yang seringkali bersifat sembarang. Model TLT memiliki struktur \textit{piecewise} yang menggabungkan distribusi Gompertz untuk usia non-ekstrem dengan \textit{Generalized Pareto Distribution} (GPD) untuk usia lanjut ekstrem. Namun, model TLT memiliki keterbatasan terkait kontinuitas fungsi \textit{hazard} pada titik transisi antara kedua bagian model. Ketidakmulusan fungsi \textit{hazard} ini menimbulkan implikasi yang tidak realistis secara biologis (nanti dimasukin sumbernya), dimana laju mortalitas sesaat dapat mengalami loncatan mendadak pada usia ambang tertentu. Keterbatasan inilah yang memotivasi pengembangan model STLT oleh Huang et al. (2020), yang memperkenalkan \textit{smoothing constraint} untuk memastikan kontinuitas fungsi \textit{hazard}.

Walaupun model STLT berhasil mengatasi masalah kontinuitas, terdapat keterbatasan dalam peramalan mortalitas. Pengamatan empiris terhadap parameter model STLT antar kohor menunjukkan adanya pola. Hal ini mendorong pengembangan model DSTLT, yang mengintegrasikan komponen dinamis untuk menangkap perubahan mortalitas antar cohort.

Bab ini akan menganalisis secara mendalam setiap tahap pengembangan model, dimulai dari formula dasar TLT, penurunan rumus \textit{smoothing constraint} pada STLT, hingga konstruksi komponen dinamis pada DSTLT.

\section{Model Threshold Life Table (TLT)}

Model \textit{Threshold Life Table} (TLT) yang diperkenalkan oleh \citet{li2008threshold} menggunakan pendekatan \textit{piecewise} yang membagi distribusi usia kematian menjadi dua bagian berdasarkan usia ambang (\textit{threshold age}) $u$. Model ini dikembangkan untuk mengatasi keterbatasan model mortalitas tradisional dalam menangani karakteristik yang berbeda antara usia lanjut non-ekstrem dan usia lanjut ekstrem.

\subsection{Formulasi Model TLT}

Misalkan $X$ adalah variabel acak yang menyatakan usia kematian seseorang. Model TLT mendefinisikan fungsi distribusi secara \textit{piecewise} berdasarkan usia ambang $u$, menggabungkan distribusi Gompertz untuk $x \leq u$ dan Generalized Pareto Distribution (GPD) untuk $x > u$.

\subsubsection{Komponen Gompertz ($x \leq u$)}

Untuk usia di bawah atau sama dengan usia ambang $u$, digunakan distribusi Gompertz dengan fungsi distribusi kumulatif:
\begin{equation}
F(x) = 1 - \exp\left(-\frac{B}{\ln C}(C^x - 1)\right), \quad x \leq u,
\label{eq:tlt_gompertz_cdf}
\end{equation}
dengan parameter $B > 0$ sebagai parameter skala dan $C > 1$ sebagai parameter bentuk yang menentukan laju peningkatan mortalitas seiring bertambahnya usia.

Fungsi survival untuk bagian Gompertz adalah:
\begin{equation}
S(x) = \exp\left(-\frac{B}{\ln C}(C^x - 1)\right), \quad x \leq u.
\label{eq:tlt_gompertz_survival}
\end{equation}

Fungsi kepadatan probabilitas (PDF) diperoleh melalui diferensiasi $f(x) = -S'(x)$:
\begin{equation}
f(x) = BC^x \exp\left(-\frac{B}{\ln C}(C^x - 1)\right), \quad x \leq u.
\label{eq:tlt_gompertz_pdf}
\end{equation}

Fungsi hazard untuk bagian Gompertz adalah:
\begin{equation}
h(x) = \frac{f(x)}{S(x)} = BC^x, \quad x \leq u,
\label{eq:tlt_gompertz_hazard}
\end{equation}
yang menunjukkan peningkatan eksponensial seiring bertambahnya usia.

\subsubsection{Komponen Generalized Pareto Distribution ($x > u$)}

Untuk usia di atas usia ambang $u$, digunakan Generalized Pareto Distribution (GPD) yang merupakan bagian dari Extreme Value Theory. GPD didefinisikan secara kondisional untuk \textit{exceedances} di atas ambang $u$.

Fungsi survival kondisional untuk bagian ini adalah:
\begin{equation}
G(x-u; \theta, \gamma) = \left(1 + \frac{\gamma(x-u)}{\theta}\right)^{-1/\gamma}, \quad x > u,
\label{eq:gpd_survival_conditional}
\end{equation}
dengan $\theta > 0$ adalah parameter skala dan $\gamma$ adalah parameter bentuk. Fungsi survival keseluruhan untuk $x > u$ adalah:
\begin{equation}
S(x) = S(u) \cdot G(x-u; \theta, \gamma) = S(u) \left(1 + \frac{\gamma(x-u)}{\theta}\right)^{-1/\gamma}, \quad x > u,
\label{eq:tlt_gpd_survival}
\end{equation}
yang memastikan kontinuitas fungsi survival pada $x = u$.

Fungsi kepadatan probabilitas untuk bagian GPD dapat diturunkan dari relasi $f(x) = -S'(x)$. Dengan menggunakan aturan rantai pada persamaan \eqref{eq:tlt_gpd_survival}:
\begin{equation}
f(x) = S(u) \cdot \frac{1}{\theta}\left(1 + \frac{\gamma(x-u)}{\theta}\right)^{-(1+1/\gamma)}, \quad x > u.
\label{eq:tlt_gpd_pdf}
\end{equation}

Fungsi hazard untuk bagian GPD diperoleh dari $h(x) = f(x)/S(x)$:
\begin{align}
h(x) &= \frac{f(x)}{S(x)} = \frac{S(u) \cdot \frac{1}{\theta}\left(1 + \frac{\gamma(x-u)}{\theta}\right)^{-(1+1/\gamma)}}{S(u) \left(1 + \frac{\gamma(x-u)}{\theta}\right)^{-1/\gamma}} \nonumber \\
&= \frac{1}{\theta + \gamma(x-u)}, \quad x > u.
\label{eq:tlt_gpd_hazard}
\end{align}

Parameter bentuk $\gamma$ menentukan karakteristik ekor distribusi: untuk $\gamma < 0$, distribusi memiliki batas atas (\textit{highest attained age}) $\omega = u + \theta/|\gamma|$, mengimplikasikan adanya usia maksimum yang dapat dicapai; untuk $\gamma = 0$, distribusi ekivalen dengan distribusi eksponensial; untuk $\gamma > 0$, distribusi memiliki ekor berat tanpa batas atas. Dalam konteks mortalitas manusia, nilai $\gamma < 0$ umumnya lebih masuk akal secara biologis \citep{dong2016evidence}.

\subsection{Estimasi Parameter Model TLT}

\subsubsection{Struktur Data Mortalitas}

Data mortalitas yang digunakan dalam estimasi model TLT umumnya berbentuk data agregat kohor dengan struktur sebagai berikut:
\begin{itemize}
    \item $d_x$: jumlah kematian yang terjadi antara usia $x$ dan $x+1$
    \item $l_x$: jumlah individu yang masih hidup pada awal usia $x$ (\textit{exposure})
    \item $x_{\text{min}}$: usia awal pengamatan (biasanya 65 tahun)
    \item $\tau$: usia maksimum pengamatan dalam data
\end{itemize}

Data seperti ini bersifat tersensor interval (\textit{interval-censored}) karena usia kematian dicatat dalam satuan tahun, dan tersensor kanan (\textit{right-censored}) karena beberapa individu mungkin masih hidup pada akhir periode pengamatan.

\subsubsection{Fungsi Likelihood dan Dekomposisi}

Fungsi likelihood untuk model TLT dengan parameter $\boldsymbol{\theta} = (B, C, \theta, \gamma, u)$ dikonstruksi berdasarkan probabilitas kematian pada setiap interval usia dan probabilitas bertahan hidup melampaui usia maksimum pengamatan. Dengan mengabaikan konstanta kombinatorial yang tidak mempengaruhi estimator Maximum Likelihood, fungsi likelihood dapat ditulis sebagai:
\begin{equation}
\begin{split}
L(B, C, \theta, \gamma; u) = &\prod_{x=x_{\text{min}}}^{u-1} \left(\frac{S(x) - S(x+1)}{S(x_{\text{min}})}\right)^{d_x} \\
&\times \prod_{x=u}^{\tau-1} \left(\frac{S(x) - S(x+1)}{S(x_{\text{min}})}\right)^{d_x} \\
&\times \left(\frac{S(\tau)}{S(x_{\text{min}})}\right)^{l_\tau},
\end{split}
\label{eq:tlt_likelihood_full}
\end{equation}
dimana fungsi survival $S(x)$ didefinisikan secara \textit{piecewise}:
\begin{equation}
S(x) =
\begin{cases}
\exp\left(-\frac{B}{\ln C}(C^x - 1)\right), & x \leq u \\[10pt]
\exp\left(-\frac{B}{\ln C}(C^u - 1)\right) \cdot \left(1 + \frac{\gamma(x-u)}{\theta}\right)^{-1/\gamma}, & x > u.
\end{cases}
\label{eq:tlt_survival_piecewise}
\end{equation}

Dengan mengambil logaritma dari persamaan \eqref{eq:tlt_likelihood_full}, diperoleh fungsi log-likelihood:
\begin{equation}
\begin{split}
\ell(B, C, \theta, \gamma; u) = &\sum_{x=x_{\text{min}}}^{u-1} d_x \ln(S(x) - S(x+1)) + \sum_{x=u}^{\tau-1} d_x \ln(S(x) - S(x+1)) \\
&+ l_\tau \ln(S(\tau)) - l_{x_{\text{min}}} \ln(S(x_{\text{min}})).
\end{split}
\label{eq:tlt_loglik_expanded}
\end{equation}

Salah satu sifat penting dari model TLT adalah bahwa fungsi log-likelihood dapat didekomposisi menjadi dua komponen independen karena tidak ada parameter yang dibagi antara bagian Gompertz dan bagian GPD:
\begin{equation}
\ell(B, C, \theta, \gamma; u) = \ell_1(B, C; u) + \ell_2(\theta, \gamma; u),
\label{eq:tlt_decomposition}
\end{equation}
dimana:
\begin{align}
\ell_1(B, C; u) &= \sum_{x=x_{\text{min}}}^{u-1} d_x \ln(S(x) - S(x+1)) + l_u \ln(S(u)) - l_{x_{\text{min}}} \ln(S(x_{\text{min}})), \label{eq:tlt_gompertz_loglik} \\
\ell_2(\theta, \gamma; u) &= \sum_{x=u}^{\tau-1} d_x \ln(S(x) - S(x+1)) + l_\tau \ln(S(\tau)) - l_u \ln(S(u)). \label{eq:tlt_gpd_loglik}
\end{align}

Dekomposisi ini memungkinkan estimasi parameter $(B, C)$ dan $(\theta, \gamma)$ dilakukan secara terpisah untuk setiap nilai $u$ yang tetap, yang meningkatkan efisiensi komputasi dalam prosedur optimisasi.

\subsubsection{Pemilihan Usia Ambang Optimal}

Pemilihan usia ambang $u$ dilakukan melalui pendekatan \textit{profile likelihood}. Untuk setiap kandidat nilai $u$ dalam rentang $\{u_{\min}, \ldots, u_{\max}\}$ (umumnya $u_{\min} = 85$ dan $u_{\max} = 100$), parameter model diestimasi dengan memaksimalkan fungsi log-likelihood \eqref{eq:tlt_decomposition}. Usia ambang optimal $\hat{u}$ dipilih sebagai nilai yang menghasilkan \textit{profile log-likelihood} maksimum:
\begin{equation}
\hat{u} = \arg\max_{u \in \{u_{\min}, \ldots, u_{\max}\}} \ell(\hat{B}(u), \hat{C}(u), \hat{\theta}(u), \hat{\gamma}(u); u),
\label{eq:profile_likelihood_optimal}
\end{equation}
dimana $\hat{B}(u), \hat{C}(u), \hat{\theta}(u), \hat{\gamma}(u)$ adalah estimator MLE yang bergantung pada nilai $u$.

\subsection{Keterbatasan Model TLT}

Meskipun model TLT memberikan pendekatan yang lebih objektif dan berbasis data untuk pemodelan mortalitas usia lanjut, model ini memiliki keterbatasan mendasar: \textbf{fungsi hazard tidak dijamin kontinu pada usia ambang $u$}.

Dari persamaan \eqref{eq:tlt_gompertz_hazard} dan \eqref{eq:tlt_gpd_hazard}, nilai fungsi hazard pada titik transisi $x = u$ adalah:
\begin{itemize}
    \item Limit dari kiri (Gompertz): $\lim_{x \to u^-} h(x) = BC^u$
    \item Limit dari kanan (GPD): $\lim_{x \to u^+} h(x) = \frac{1}{\theta}$
\end{itemize}

Secara umum, tidak terdapat jaminan bahwa $BC^u = \frac{1}{\theta}$. Dalam estimasi model TLT, parameter $(B, C)$ dan $(\theta, \gamma)$ diestimasi secara independen, sehingga kondisi kontinuitas tidak secara otomatis terpenuhi. Ketidaksamaan ini mengakibatkan diskontinuitas (loncatan) pada fungsi hazard:
\begin{equation}
\Delta h(u) = BC^u - \frac{1}{\theta} \neq 0.
\label{eq:hazard_jump}
\end{equation}

Loncatan mendadak dalam laju mortalitas sesaat tidak memiliki justifikasi biologis. Proses penuaan dan deteriorasi fisiologis bersifat gradual dan kontinu, sehingga laju mortalitas sesaat seharusnya juga berubah secara mulus. Keterbatasan ini memotivasi pengembangan model Smoothed Threshold Life Table (STLT) yang akan dibahas pada bagian selanjutnya, dimana \textit{smoothing constraint} diperkenalkan untuk memastikan kemulusan fungsi hazard pada usia ambang.
\section{Model Smoothed Threshold Life Table (STLT)}

Model Threshold Life Table (TLT) yang telah dibahas memberikan kerangka kerja yang solid untuk pemodelan mortalitas usia lanjut dengan mengintegrasikan Extreme Value Theory. Namun, model ini memiliki keterbatasan mendasar: \textbf{fungsi hazard tidak dijamin kontinu pada usia ambang $u$}, yang mengakibatkan loncatan mendadak dalam laju mortalitas sesaat. Keterbatasan ini memotivasi pengembangan model Smoothed Threshold Life Table (STLT) yang memastikan kemulusan fungsi hazard melalui penambahan \textit{smoothing constraint}.

\subsection{Masalah Diskontinuitas dan Derivasi Smoothing Constraint}

Dalam model TLT, fungsi hazard didefinisikan secara \textit{piecewise}: $h(x) = BC^x$ untuk $x \leq u$ (Gompertz) dan $h(x) = \frac{1}{\theta + \gamma(x-u)}$ untuk $x > u$ (GPD). Pada titik transisi $x = u$, nilai fungsi hazard dari kedua komponen adalah $\lim_{x \to u^-} h(x) = BC^u$ dan $\lim_{x \to u^+} h(x) = \frac{1}{\theta}$. Karena parameter $(B, C)$ dan $(\theta, \gamma)$ diestimasi secara independen, secara umum $BC^u \neq \frac{1}{\theta}$, yang mengakibatkan diskontinuitas:
\begin{equation}
\Delta h(u) = BC^u - \frac{1}{\theta} \neq 0.
\label{eq:stlt_hazard_discontinuity}
\end{equation}

Loncatan mendadak dalam laju mortalitas sesaat tidak memiliki justifikasi biologis, karena proses penuaan dan deteriorasi fisiologis bersifat gradual dan kontinu. Untuk mengatasi masalah ini, model STLT menambahkan kondisi kontinuitas eksplisit pada fungsi hazard di usia ambang:
\begin{equation}
BC^u = \frac{1}{\theta}.
\label{eq:smoothing_condition}
\end{equation}

Dari persamaan \eqref{eq:smoothing_condition}, diperoleh hubungan eksplisit antara parameter skala GPD dengan parameter Gompertz:
\begin{equation}
\boxed{\theta = \frac{1}{BC^u}}
\label{eq:theta_constraint}
\end{equation}

Persamaan \eqref{eq:theta_constraint} merupakan \textit{smoothing constraint} yang menjadi kunci model STLT. Hubungan ini menunjukkan bahwa parameter skala GPD ($\theta$) bukan lagi parameter bebas, melainkan fungsi deterministik dari parameter Gompertz $(B, C)$ dan usia ambang $u$. Dengan constraint ini, model STLT memiliki tiga parameter bebas $(B, C, \gamma)$ dan satu parameter turunan $\theta = \frac{1}{BC^u}$.

\subsection{Formulasi Model STLT}

Dengan menerapkan \textit{smoothing constraint}, model STLT dapat diformulasikan secara lengkap. Struktur model tetap \textit{piecewise}, menggabungkan Gompertz untuk $x \leq u$ dan GPD untuk $x > u$, namun dengan parameter yang saling terhubung.

\subsubsection{Komponen Gompertz ($x \leq u$)}

Untuk usia di bawah atau sama dengan usia ambang $u$, model STLT menggunakan distribusi Gompertz yang identik dengan model TLT. Fungsi survival adalah:
\begin{equation}
S(x) = \exp\left(-\frac{B}{\ln C}(C^x - 1)\right), \quad x \leq u,
\label{eq:stlt_gompertz_survival}
\end{equation}
dengan PDF $f(x) = BC^x \exp\left(-\frac{B}{\ln C}(C^x - 1)\right)$ dan fungsi hazard $h(x) = BC^x$. Parameter $B > 0$ mengatur tingkat mortalitas dasar, sementara $C > 1$ mengatur laju peningkatan mortalitas seiring bertambahnya usia.

\subsubsection{Komponen GPD ($x > u$)}

Untuk usia di atas usia ambang $u$, model STLT menggunakan GPD dengan parameter skala $\theta = \frac{1}{BC^u}$. Fungsi survival keseluruhan adalah:
\begin{equation}
S(x) = S(u) \left(1 + \gamma \cdot BC^u(x-u)\right)^{-1/\gamma}, \quad x > u,
\label{eq:stlt_gpd_survival}
\end{equation}
dengan batasan $x < \omega$ jika $\gamma < 0$. Fungsi PDF dapat diturunkan dari $f(x) = -S'(x)$:
\begin{equation}
f(x) = S(u) \cdot BC^u \left(1 + \gamma \cdot BC^u(x-u)\right)^{-(1+1/\gamma)}, \quad x > u.
\label{eq:stlt_gpd_pdf}
\end{equation}

Fungsi hazard untuk bagian GPD adalah:
\begin{equation}
h(x) = \frac{BC^u}{1 + \gamma \cdot BC^u(x-u)}, \quad x > u.
\label{eq:stlt_gpd_hazard}
\end{equation}

Kontinuitas fungsi hazard dapat diverifikasi dengan mengevaluasi limit: $\lim_{x \to u^+} h(x) = \frac{BC^u}{1 + 0} = BC^u$, yang sama dengan $\lim_{x \to u^-} h(x) = BC^u$, sehingga $h(u^-) = h(u^+) = BC^u$.

\paragraph{Derivasi Highest Attained Age.} Untuk kasus $\gamma < 0$, distribusi memiliki batas atas. Batas usia maksimum $\omega$ diperoleh dari kondisi bahwa argumen dalam fungsi survival harus non-negatif:
\begin{equation}
1 + \gamma \cdot BC^u(x-u) \geq 0.
\end{equation}
Karena $\gamma < 0$, kondisi ini ekivalen dengan:
\begin{align}
1 - |\gamma| \cdot BC^u(x-u) &\geq 0 \nonumber \\
|\gamma| \cdot BC^u(x-u) &\leq 1 \nonumber \\
x - u &\leq \frac{1}{BC^u|\gamma|} \nonumber \\
x &\leq u + \frac{1}{BC^u|\gamma|}.
\end{align}
Dengan substitusi $\theta = \frac{1}{BC^u}$, diperoleh:
\begin{equation}
\boxed{\omega = u + \frac{\theta}{|\gamma|} = u + \frac{1}{BC^u|\gamma|}}
\label{eq:stlt_omega_derivation}
\end{equation}
dimana $S(\omega) = 0$ dan $F(\omega) = 1$.

Parameter bentuk $\gamma$ menentukan karakteristik ekor distribusi: $\gamma < 0$ mengimplikasikan adanya batas atas usia (finite lifespan), $\gamma = 0$ ekivalen dengan distribusi eksponensial, dan $\gamma > 0$ mengimplikasikan ekor berat tanpa batas atas.

\subsection{Estimasi Parameter Model STLT}

Estimasi parameter model STLT menggunakan metode Maximum Likelihood Estimation (MLE) dengan prosedur \textit{profile likelihood} untuk pemilihan usia ambang optimal. Berbeda dengan model TLT yang dapat mendekomposisi likelihood menjadi dua komponen independen, model STLT memerlukan estimasi simultan parameter $(B, C, \gamma)$ karena adanya \textit{smoothing constraint} yang menghubungkan kedua komponen.

\subsubsection{Fungsi Likelihood}

Fungsi likelihood untuk model STLT dengan parameter $(B, C, \gamma, u)$ dikonstruksi berdasarkan data agregat kohor dengan struktur yang sama seperti TLT. Dengan mengabaikan konstanta kombinatorial, fungsi log-likelihood adalah:
\begin{equation}
\ell(B, C, \gamma; u) = \sum_{x=x_{\text{min}}}^{\tau-1} d_x \ln(S(x) - S(x+1)) + l_\tau \ln(S(\tau)) - l_{x_{\text{min}}} \ln(S(x_{\text{min}})),
\label{eq:stlt_loglikelihood}
\end{equation}
dimana fungsi survival $S(x)$ didefinisikan secara \textit{piecewise} dengan constraint $\theta = \frac{1}{BC^u}$.

Karena constraint menghubungkan parameter Gompertz dan GPD, likelihood tidak dapat didekomposisi seperti pada TLT. Estimasi parameter $(B, C, \gamma)$ harus dilakukan secara simultan untuk setiap nilai $u$ yang tetap.

\subsubsection{Pemilihan Usia Ambang Optimal}

Pemilihan usia ambang $u$ menggunakan pendekatan yang sama dengan TLT, yaitu \textit{profile likelihood}. Untuk setiap kandidat nilai $u$ dalam rentang $\{u_{\min}, \ldots, u_{\max}\}$ (umumnya $u_{\min} = 85$ dan $u_{\max} = 100$), parameter $(B, C, \gamma)$ diestimasi dengan memaksimalkan fungsi log-likelihood \eqref{eq:stlt_loglikelihood}. Usia ambang optimal $\hat{u}$ dipilih sebagai nilai yang menghasilkan \textit{profile log-likelihood} maksimum:
\begin{equation}
\hat{u} = \arg\max_{u \in \{u_{\min}, \ldots, u_{\max}\}} \ell(\hat{B}(u), \hat{C}(u), \hat{\gamma}(u); u).
\label{eq:stlt_profile_likelihood_optimal}
\end{equation}

\textbf{Perbedaan dengan TLT:} Dalam TLT, untuk setiap nilai $u$ tetap, parameter $(B, C)$ dan $(\theta, \gamma)$ dapat diestimasi secara terpisah karena dekomposisi likelihood. Dalam STLT, parameter $(B, C, \gamma)$ harus diestimasi secara simultan menggunakan algoritma optimisasi numerik (misalnya, L-BFGS-B) karena constraint $\theta = \frac{1}{BC^u}$ menghubungkan kedua komponen. Hal ini meningkatkan kompleksitas komputasi, namun memastikan kontinuitas fungsi hazard.

Setelah estimasi, parameter turunan dihitung sebagai $\hat{\theta} = \frac{1}{\hat{B}\hat{C}^{\hat{u}}}$, dan untuk $\hat{\gamma} < 0$, batas atas usia dihitung sebagai $\hat{\omega} = \hat{u} + \frac{\hat{\theta}}{|\hat{\gamma}|}$. Untuk stabilitas numerik, dapat digunakan reparameterisasi $\alpha = \ln B$ dan $\delta = \ln(\ln C)$ sehingga constraint positivity otomatis terpenuhi.
\section{Model Dynamic Smooth Threshold Life Table (DSTLT)}

Model STLT yang telah dibahas berhasil mengatasi masalah diskontinuitas fungsi \textit{hazard} yang terdapat pada model TLT. Namun, model STLT tetap bersifat statis, yang berarti model diestimasi secara terpisah untuk setiap kohor tanpa mempertimbangkan hubungan antar kohor. Keterbatasan ini membatasi kemampuan model STLT untuk melakukan peramalan mortalitas masa depan, yang merupakan kebutuhan krusial dalam aplikasi aktuaria seperti \textit{pricing} produk anuitas dan manajemen risiko longevitas.

Bagian ini membahas pengembangan model DSTLT yang mengintegrasikan komponen dinamis ke dalam model STLT, dimotivasi oleh observasi empiris terhadap pola temporal parameter model STLT antar kohor.

\subsection{Keterbatasan Model Statis dan Motivasi Pengembangan DSTLT}

Model statis seperti STLT memiliki beberapa keterbatasan fundamental untuk aplikasi peramalan mortalitas:

\begin{enumerate}
    \item \textbf{Tidak menangkap tren temporal}: Estimasi dilakukan secara independen untuk setiap kohor, sehingga pola sistematik perubahan mortalitas antar generasi tidak teridentifikasi.

    \item \textbf{Tidak dapat melakukan proyeksi}: Untuk memproyeksikan mortalitas kohor masa depan, model statis memerlukan asumsi tambahan yang bersifat \textit{ad hoc}.

    \item \textbf{Keterbatasan praktis}: Dalam aplikasi aktuaria seperti penilaian kewajiban dana pensiun multi-generasi, diperlukan proyeksi mortalitas yang konsisten untuk kohor-kohor yang berbeda, yang tidak dapat diberikan oleh model statis.
\end{enumerate}

\subsection{Temuan Empiris Huang et al. (2020)}

\citet{huang2020modelling} melakukan analisis sistematis terhadap parameter model STLT yang diestimasi untuk kohor kelahiran 1893--1908 di Belanda. Analisis ini mengungkapkan beberapa pola temporal yang penting:

\begin{enumerate}
    \item \textbf{Parameter $B$ menunjukkan tren temporal yang jelas}: Untuk kohor perempuan, nilai $\hat{B}$ menurun secara konsisten dari 0.000031 (kohor 1893) menjadi 0.000015 (kohor 1901), suatu penurunan sekitar 52\%. Pola ini mengindikasikan adanya perbaikan mortalitas yang sistematis pada tingkat mortalitas dasar (\textit{baseline mortality level}) antar generasi.

    \item \textbf{Parameter $C$ menunjukkan variasi yang jauh lebih kecil}: Perubahan parameter $C$ kurang dari 1\% antar kohor, sementara perubahan $B$ mencapai lebih dari 50\%. Hal ini menunjukkan bahwa perbaikan mortalitas terutama terjadi pada tingkat mortalitas dasar, bukan pada laju akselerasi mortalitas.

    \item \textbf{Parameter $\gamma$, $\theta$, dan $u$ tidak menunjukkan pola temporal yang konsisten}: Nilai-nilai parameter ini berfluktuasi tanpa arah yang sistematis antar kohor. Temuan ini konsisten dengan hasil \citet{einmahl2019modeling} pada data yang sama.
\end{enumerate}

Berdasarkan observasi empiris ini, Huang et al. memutuskan untuk memodelkan parameter $B$ sebagai fungsi yang bervariasi antar kohor, sementara parameter lainnya ($\gamma$, $\theta$, $u$) diperlakukan sebagai konstan antar kohor.

\subsection{Pemilihan Parameter Dinamis dan Konsistensi dengan Cairns et al.}

Keputusan untuk mendinamisasi parameter $B$ didasarkan pada beberapa pertimbangan:

\begin{enumerate}
    \item \textbf{Bukti empiris yang kuat}: Tren temporal parameter $B$ jelas dan konsisten, dengan variasi antar kohor yang signifikan (> 50\%).

    \item \textbf{Interpretasi yang masuk akal}: Parameter $B$ merepresentasikan tingkat mortalitas dasar, yang secara intuitif dapat berubah antar generasi akibat perbaikan kondisi kesehatan, nutrisi, dan layanan medis.

    \item \textbf{Konsistensi dengan literatur}: Pemilihan ini konsisten dengan penelitian \citet{cairns2006pricing} dalam model Cairns-Blake-Dowd (CBD). Cairns et al. memodelkan komponen tingkat mortalitas sebagai \textit{random walk with drift}, yang secara konseptual serupa dengan spesifikasi $B_i = \exp(a + bi)$ dalam DSTLT.
\end{enumerate}

Model CBD, yang merupakan salah satu model mortalitas stokastik paling berpengaruh dalam aktuaria, juga mengidentifikasi bahwa komponen tingkat mortalitas (bukan slope) yang memiliki tren temporal paling dominan. Temuan Huang et al. untuk parameter $B$ dalam STLT paralel dengan temuan Cairns et al. untuk parameter $\kappa_t^{(1)}$ dalam CBD, memberikan validasi silang dari dua pendekatan metodologi yang berbeda.

\subsection{Formulasi Model DSTLT}

Model DSTLT mempertahankan struktur \textit{piecewise} dari STLT, namun dengan parameter Gompertz yang bervariasi antar kohor. Misalkan $i$ adalah indeks kohor, model DSTLT didefinisikan sebagai berikut:

\subsubsection{Struktur Parameter Dinamis}

Parameter model DSTLT terdiri dari:
\begin{itemize}
    \item \textbf{Parameter dinamis}: $B_i = \exp(a + bi)$, dimana $a$ dan $b$ adalah parameter yang mengatur level dan tren temporal mortalitas dasar.
    \item \textbf{Parameter konstan}: $\gamma$, $\theta$, dan $u$ diasumsikan konstan antar kohor.
\end{itemize}

Dengan \textit{smoothing constraint}, parameter $C_i$ tidak lagi independen tetapi ditentukan secara implisit melalui hubungan $\theta = \frac{1}{B_i C_i^u}$, yang dapat ditulis sebagai:
\begin{equation}
C_i = \left(\frac{1}{B_i \theta}\right)^{1/u} = \left(\frac{1}{\exp(a+bi) \theta}\right)^{1/u}.
\label{eq:dstlt_c_constraint}
\end{equation}

Dengan demikian, model DSTLT memiliki empat parameter bebas: $(a, b, \gamma, u)$, dengan $\theta$ dan $C_i$ ditentukan oleh constraint.

\subsubsection{Komponen Model untuk Kohor $i$}

Untuk kohor $i$, fungsi survival didefinisikan secara \textit{piecewise}:

\paragraph{Bagian Gompertz ($x \leq u$):}
\begin{equation}
S_i(x) = \exp\left(-\frac{B_i}{\ln C_i}(C_i^x - 1)\right), \quad x \leq u,
\label{eq:dstlt_gompertz_survival}
\end{equation}
dengan $B_i = \exp(a + bi)$ dan $C_i$ diberikan oleh persamaan \eqref{eq:dstlt_c_constraint}.

\paragraph{Bagian GPD ($x > u$):}
\begin{equation}
S_i(x) = S_i(u) \left(1 + \gamma \cdot B_i C_i^u(x-u)\right)^{-1/\gamma}, \quad x > u.
\label{eq:dstlt_gpd_survival}
\end{equation}

Perhatikan bahwa $B_i C_i^u = \frac{1}{\theta}$ untuk semua $i$ karena \textit{smoothing constraint}, sehingga bagian GPD menjadi:
\begin{equation}
S_i(x) = S_i(u) \left(1 + \frac{\gamma(x-u)}{\theta}\right)^{-1/\gamma}, \quad x > u.
\label{eq:dstlt_gpd_survival_simplified}
\end{equation}

Formulasi ini menunjukkan bahwa variasi antar kohor hanya terjadi pada bagian Gompertz (melalui $B_i$ dan $C_i$), sementara bagian GPD tetap sama untuk semua kohor. Hal ini konsisten dengan observasi empiris bahwa karakteristik ekor distribusi ($\gamma$) tidak menunjukkan tren temporal yang jelas.

\subsection{Estimasi Parameter Model DSTLT}

Estimasi parameter model DSTLT menggunakan data gabungan dari beberapa kohor dalam periode \textit{training}. Dengan asumsi bahwa kohor-kohor independen, fungsi log-likelihood gabungan adalah:
\begin{equation}
\ell(a, b, \gamma; u) = \sum_{i=1}^{r} \ell_i(a, b, \gamma; u),
\label{eq:dstlt_joint_loglikelihood}
\end{equation}
dimana $r$ adalah jumlah kohor dalam \textit{training set}, dan $\ell_i(a, b, \gamma; u)$ adalah log-likelihood untuk kohor $i$:
\begin{equation}
\begin{split}
\ell_i(a, b, \gamma; u) = &\sum_{x=x_{\text{min}}}^{\tau_i-1} d_{x,i} \ln(S_i(x) - S_i(x+1)) \\
&+ l_{\tau_i,i} \ln(S_i(\tau_i)) - l_{x_{\text{min}},i} \ln(S_i(x_{\text{min}})).
\end{split}
\label{eq:dstlt_single_loglikelihood}
\end{equation}

Parameter $(a, b, \gamma)$ diestimasi dengan memaksimalkan fungsi log-likelihood gabungan \eqref{eq:dstlt_joint_loglikelihood} untuk setiap nilai $u$ tetap. Usia ambang optimal $\hat{u}$ dipilih menggunakan \textit{profile likelihood}, serupa dengan STLT. Setelah estimasi, parameter $\theta$ dihitung secara konsisten dari constraint, dan proyeksi untuk kohor masa depan dilakukan dengan mengekstrapolasi $B_i = \exp(a + bi)$ untuk nilai $i$ yang lebih besar.

Perbedaan utama dengan STLT adalah bahwa DSTLT melakukan estimasi secara simultan untuk semua kohor dalam \textit{training set}, sehingga dapat menangkap tren temporal dan melakukan proyeksi untuk kohor masa depan secara konsisten.
